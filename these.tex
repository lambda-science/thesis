% Document de classe yathesis
\documentclass[colophon-location=nowhere]{yathesis}

%
% Chargement manuel de packages (pas déjà chargés par la classe yathesis)
\usepackage[T1]{fontenc}
\usepackage[utf8]{inputenc}
\usepackage{kpfonts}
\usepackage{booktabs}
\usepackage{siunitx}
\usepackage{pgfplots}
\pgfplotsset{compat=1.18} 
\usepackage{caption}
\usepackage{microtype}
\usepackage{varioref}
%\usepackage[xindy,quiet]{imakeidx}
%\usepackage[autostyle]{csquotes}
%\usepackage[backend=biber,safeinputenc]{biblatex}
\usepackage{hyperref}
%\usepackage[xindy,acronyms,symbols]{glossaries}
%
% (Facultatif) Génération de l'index (obligatoire si un package d'index, par
% exemple « imakeidx », est chargé)
% \makeindex
%
% (Facultatif) Spécification de la ou des ressources bibliographiques
% (obligatoire si le package « biblatex » est chargé)
% \addbibresource{auxiliaires/bibliographie.bib}
% \addbibresource{auxiliaires/}
%
% (Facultatif) Génération du glossaire (obligatoire si le package « glossaries »
% est chargé)
% \makeglossaries
%
% (Facultatif) Configuration des styles du glossaire et de la liste d'acronymes
% (à n'utiliser que si le package « glossaries » est chargé)
% \setglossarystyle{indexhypergroup}
% \setacronymstyle{long-sc-short}
%
% (Facultatif) Spécification de la ou des ressources terminologiques
% \loadglsentries{auxiliaires/}
% \loadglsentries{auxiliaires/}
% \loadglsentries{auxiliaires/}
%
% Les réglages figurant habituellement dans le préambule, notamment concernant
% la bibliographie et l'éventuel index, peuvent être saisis dans le fichier
% « thesis.cfg » (situé dans le sous-dossier « configuration ») qui est
% automatiquement importé par la classe yathesis.
%
% Importation manuelle du fichier de macros personnelles
\DeclareUnicodeCharacter{2009}{\,}
\input{configuration/macros}
%
%%%%%%%%%%%%%%%%%%%%%%%%%%%%%%%%%%%%%%%%%%%%%%%%%%%%%%%%%%%%%%%%%%%%%%%%%%%%%%%
%%%%%%%%%%%%%%%%%%%%%%%%%%%%%%%%%%%%%%%%%%%%%%%%%%%%%%%%%%%%%%%%%%%%%%%%%%%%%%%
% Début du document
%%%%%%%%%%%%%%%%%%%%%%%%%%%%%%%%%%%%%%%%%%%%%%%%%%%%%%%%%%%%%%%%%%%%%%%%%%%%%%%
%%%%%%%%%%%%%%%%%%%%%%%%%%%%%%%%%%%%%%%%%%%%%%%%%%%%%%%%%%%%%%%%%%%%%%%%%%%%%%%
\begin{document}
%
%%%%%%%%%%%%%%%%%%%%%%%%%%%%%%%%%%%%%%%%%%%%%%%%%%%%%%%%%%%%%%%%%%%%%%%%%%%%%%%
% Caractéristiques du document
%%%%%%%%%%%%%%%%%%%%%%%%%%%%%%%%%%%%%%%%%%%%%%%%%%%%%%%%%%%%%%%%%%%%%%%%%%%%%%%
%
% Préparation des pages de couverture et de titre
%%%%%%%%%%%%%%%%%%%%%%%%%%%%%%%%%%%%%%%%%%%%%%%%%%%%%%%%%%%%%%%%%%%%%%%%%%%%%%%
% Les caractéristiques de la thèse sont saisies dans le fichier
% « characteristics.tex » (situé dans le dossier « configuration »).
%
% Production des pages de couverture et de titre
%%%%%%%%%%%%%%%%%%%%%%%%%%%%%%%%%%%%%%%%%%%%%%%%%%%%%%%%%%%%%%%%%%%%%%%%%%%%%%%
\maketitle
%
%%%%%%%%%%%%%%%%%%%%%%%%%%%%%%%%%%%%%%%%%%%%%%%%%%%%%%%%%%%%%%%%%%%%%%%%%%%%%%%
% Début de la partie liminaire de la thèse
%%%%%%%%%%%%%%%%%%%%%%%%%%%%%%%%%%%%%%%%%%%%%%%%%%%%%%%%%%%%%%%%%%%%%%%%%%%%%%%
%
% (Facultatif) Production de la page de clause de non-responsabilité
% \makedisclaimer
%
% (Facultatif) Production de la page de mots clés
% \makekeywords
%
% (Facultatif) Production de la page affichant les logo, nom et coordonnées du
% ou des laboratoires (ou unités de recherche) où la thèse a été préparée
% \makelaboratory
%
% (Facultatif) Dédicace(s)
% % Dédicace(s)
\dedication{}
\dedication{}
% Production de la page de dédicace(s)
\makededications


%
% (Facultatif) Épigraphe(s)
% % Épigraphes(s)
\frontepigraph{}{}
\frontepigraph{}{}
% Production de la page de d'épigraphe(s)
\makefrontepigraphs


%
% Résumés succincts
% Résumés (de 1700 caractères maximum, espaces compris) dans la
% langue principale (1re occurrence de l'environnement « abstract »)
% et, facultativement, dans la langue secondaire (2e occurrence de
% l'environnement « abstract »)
\begin{abstract}
Résume court en français
\end{abstract}
\begin{abstract}
Short summary in English
\end{abstract}
\makeabstract


%
% (Facultatif) Chapitre de remerciements
\chapter{Remerciements}
Tout d'abord, je souhaite remercier les membres de mon jury de thèse qui me font l'honneur d'avoir accepté d'évaluer mes travaux de thèse. Merci au Dr Malika Smail-Tabbone, au Dr Antonio Rausell, au Dr Gisèle Bonne et au Pr Cédric Wemmert. \\

J'aimerais remercier l'ensemble des personnes qui ont rendus cette thèse possible, à commencer par l'équipe du CSTB, une équipe que je pourrais qualifier d'atypique, par sa diversité tant sur le plan scientifique que humain. Une équipe qui m'a permis de découvrir le monde de la recherche et de découvrir ce que je souhaitais vraiment faire. Alors merci aux permanents de l'équipe dans leur ensemble. Je souhaiterais remercier en particulier un premier bureau, celui de Laetitia, Arnaud et Nicolas. Pour toi Arnaud la lutte continue, merci pour ces discussions techniques, mais aussi politiques, culturelles et ces \textit{memes} très obscurs. Nicolas je te souhaite soit d'être parmi les prochains, soit de réussir à ouvrir une salle d'escalade dans un autre pays. \\

J'aimerais avoir un mot particulier pour le duo Julie et Odile. Merci, Julie, de m'avoir donné l'opportunité de commencer en stage avec toi, puis de continuer en thèse dans cette équipe et enfin de m'avoir aidé dans cette dernière ligne droite. J'ai toujours admiré ta modestie et ta capacité à aller droit au but. Merci à toi Odile de m'avoir donné l'opportunité d'enseigner à l'ESBS c'était une expérience formidable qui m'a beaucoup aidé et je me suis même découvert une petite passion pour l'enseignement. En dehors du travail, vous êtes deux personnes que j'admire beaucoup sur le plan humain, avec vous, la bio-informatique à Strasbourg est sereine. \\

Je tiens à remercier aussi ceux que je nommerais les précaires, ces stagiaires et doctorants que j'ai pu rencontrer dans l' équipe et avec qui j'ai pu tisser du lien. Merci à Quentin, Tam'si, Nicolas S., Thomas, Romain et Arthur. Merci à Christelle, Hiba et Amani du "bureau d'à côté", vous êtes toutes les trois les prochaines à soutenir et je suis persuadé que cela va bien se passer. Enfin, merci aux personnes qui ont animé et habité le même bureau que moi, par le passé ou par le présent, merci à Célia, Lucille, Dorine et Alix, c'était un plaisir de partager ces moments avec vous qui ont donné lieu à des tranches de vie pleines de rebondissements, mais qui n'en resteront pas moins de bons souvenirs. \\

Merci à Jocelyn Laporte et son équipe ainsi qu'à l'équipe de Norma Romero pour ces collaborations qui ont porté leurs fruits à travers ces travaux de thèse. Vos connaissances extensives des myopathies ont été très stimulantes et nos échanges m'ont toujours permis de repartir l'esprit plein de nouvelles idées à mettre en place. \\

Enfin, je souhaite dédier un remerciement particulier pour mes directeurs de thèses et mes encadrants. Merci à Anne et Pierre pour leurs conseils et expertises en IA (ainsi qu'en philosophie !). Merci à ce duo de choc (c'est le cas de le dire), Olivier et Kirsley, dont l'énergie et l'animation contrebalancent et n'ont d'égales que le calme et l'organisation d'Odile et Julie. Merci à Kirsley, toujours au four et au moulin entre les rédactions de \textit{grant}, le développement et l'encadrement de stagiaires et doctorants, pour ses connaissances techniques et sa capacité à toujours aller plus loin dans les questionnements scientifiques. Merci à Olivier d'avoir été infatigablement derrière cette thèse du début à la fin, même dans les moments de flottement. Ainsi que pour cette capacité de raconter des histoires en continu dès que l'occasion se présente. \\

De manière générale, à tous ceux qui ont commencé comme collègues de travail, mais qui sont devenus bien plus que ça: merci, on sera amené à se revoir. \\

Avant d'être une aventure professionnelle, la thèse est avant tout une aventure personnelle. J'aimerais dédier cette partie à l'ensemble de mes proches qui n'ont pas été impliqués directement dans cette thèse, mais sans qui elle n'aurait jamais vu le jour. \\

À mes frères du groupe \textit{Persepolis} et ADN: Adil, Arthur, Ernest, Keziah, Lucie, Malo et les autres. Merci. Du fond du cœur. Vous me rendez heureux et vous côtoyer tous les jours est un réel plaisir. J'espère que ça va perdurer le plus longtemps possible, on se sait. Le projet Japon est toujours dans un coin de ma tête. \\

À ceux qu'on côtoie moins à cause des trajets de vie, mais avec qui chaque retrouvaille est comme un retour à hier: merci, Bilal, Vincent, Baptiste et Victor. Vous avez vu, j'ai fini, je vais peut-être pouvoir sortir de la salle du temps. \\

Au Discord des doctorants français "\textit{PhD Students}" et à sa communauté. C'est étrange de remercier une entité, mais j'aurais clairement arrêté la thèse sans ce refuge et sans avoir échangé (et râlé en cœur) avec tant de gens géniaux. Merci notamment à Anaïs et Floriane. Et merci à la délégation strasbourgeoise avec qui on a pu échanger tant de bons moments (et qui j'espère vont continuer). Merci au duo Alix et Alix, Emilien, Erin, Maï et Nato. Pour certains d'entre vous aussi, c'est bientôt votre tour. Un mot en particulier pour Alix n°2, camarade de conférences, de \textit{gossip} et de discussions désastreuses, courage pour la suite, tu es bien entourée. \\

À tous ceux qui m'ont permis de sortir la tête de la thèse, souvent par l'escalade, mais pas que, merci. Merci à Louise, Magalie, Morgane, Pierre, Eloïse. \fontencoding{T2A}\selectfont Спасибо, Настя, надеюсь, что в Бретани нам удастся поесть морепродуктов !\fontencoding{T1}\selectfont \\

Enfin, il me reste à remercier trois personnes spéciales, qui constituent ma famille. Je ne suis pas très doué pour le montrer, mais vous savez que je vous aime. Merci, Maman, merci, Olivier, merci, Coraline. Sans votre soutien inconditionnel toutes ces années durant, je ne serais évidemment pas ici aujourd'hui. Je sais que pour vous, le travail de thèse est assez obscur, j'avais peur de ne pas réussir à aller au bout, mais ne vous inquiétez pas après ça, c'est fini. \\

À tous mes proches que je remercie sincèrement à travers ces mots et qui savent que je ne suis pas forcément le plus habile pour ce genre d'exercice, si je devais résumer le fond de ma pensée en trois mots: \\
\begin{center}
\Large \textbf{Vous êtes lumineux.}
\end{center}

%
% (Facultatif) Chapitre d'avertissement
% \include{liminaires/avertissement}

%
% (Facultatif) Liste des symboles
% \printsymbols
%
% (Facultatif) Chapitre d'avant-propos
% \chapter{Avant-Propos}
Placeholder pour l'avant-propos
%
% Sommaire
\tableofcontents[depth=chapter,name=Sommaire]
%
% (Facultatif) Table des figures
\listoffigures
%
% (Facultatif) Liste des tableaux
\listoftables
%
% (Facultatif) Liste des acronymes
% \printacronyms
%
% (Facultatif) Table des listings (nécessite que le package « listings » soit
% chargé)
% \lstlistoflistings
%
%%%%%%%%%%%%%%%%%%%%%%%%%%%%%%%%%%%%%%%%%%%%%%%%%%%%%%%%%%%%%%%%%%%%%%%%%%%%%%%
% Début de la partie principale (du « corps ») de la thèse
%%%%%%%%%%%%%%%%%%%%%%%%%%%%%%%%%%%%%%%%%%%%%%%%%%%%%%%%%%%%%%%%%%%%%%%%%%%%%%%
\chapter*{Resumé de Thèse}
% ...
\subsubsection{Intelligence artificielle et classification par machine-learning}
L’intelligence artificielle (IA) représente un ensemble de techniques permettant de créer des programmes simulant l’intelligence humaine.  Une sous-branche de l’IA nommée Machine-Learning (ML) regroupe l’ensemble des algorithmes permettant à un programme informatique d’accomplir une tâche en apprenant d’un jeu de données. Il peut s’agir par exemple de tâche de régression (prédiction d’une valeur comme une durée ou une espérance de vie) ou bien d’une tâche de classification (prédiction d’une catégorie comme un diagnostic). Le but étant d’être capable de réaliser des prédictions sur de nouvelles données jamais rencontrées lors de l’apprentissage. 

Cet apprentissage est le plus souvent dit « supervisé ». C’est-à-dire que pour chaque point de donnée, le label (valeur ou classe à prédire) est connu et l’algorithme apprend simplement à le reproduire. Il est aussi possible de réaliser un apprentissage non supervisé, souvent nommé clustering, lors duquel les labels des données ne sont pas connus, l’algorithme de ML cherche alors à constituer des clusters, c’est-à-dire des groupes de point de données similaires. La prédiction effectuée sera alors l’appartenance à un cluster spécifique.

L’IA  possède un intérêt particulier dans le domaine médical, car elle pourrait  permettre sur le plan du diagnostic un gain de temps et de précision aux professionnels de santé pour poser un diagnostic ou trouver la cause génétique d’une maladie. Sur le plan de la recherche, l’IA peut aussi permettre de découvrir de nouveaux critères de diagnostic pour les maladies difficiles à diagnostiquer. En effet, les techniques d’IA peuvent identifier et utiliser des critères de classification différents de l’Homme, mettant alors en lumière de potentiels nouveaux facteurs discriminants pour le diagnostic de la maladie.

Cependant le développement de modèle de ML est couteux en temps de travail. Le jeu de données d’entrainement est le point central de cette approche, il doit être de suffisamment grande taille et de suffisamment bonne qualité. Les techniques de ML ne peuvent qu’apprendre que de données sous forme de tableaux où chaque ligne représente un point de donnée et chaque colonne un descripteur de ce point de donnée. La constitution d’un tel jeu de données demande un travail d’annotation et de curation conséquent tout en connaissant à l’avance les descripteurs pertinents qui pourront permettre la création d’un modèle performant. De plus cela ferme la porte à l’exploitation de données plus complexes, comme les images ou les textes libres qui peuvent difficilement être représentés sous la forme d’un tableau de données sans perdre d’information ou sans travail manuel et intensif d’annotation.

\subsubsection{Réseaux de neurones profonds et exploitation de données complexes}

Cette dernière décennie, l’augmentation de la puissance des cartes graphiques (GPU) et l’optimisation des opérations des calculs ont permis la popularisation des réseaux de neurones profonds (deep neural networks, DNN). Les DNN sont une famille d’algorithme de ML reposant sur le concept bio-inspiré des neurones. Ces neurones sont la brique de base des DNN et sont organisés selon une architecture spécifique permettant de mettre en relation plusieurs millions voire milliards de neurones. Les DNN grâce à leur architecture de très grande taille permettent l’exploitation de données complexes sans avoir besoin de connaissance a priori sur les données, c’est-à-dire sans devoir définir des descripteurs pertinents manuellement.
Par exemple, grâce aux architectures de réseau neuronal convolutif (CNN), il est possible d’entrainer un modèle capable de faire la différence entre chien et chat à partir d’images brutes sans avoir à passer par des représentations intermédiaires. 

Grâce aux architectures à base de modules d’attention (Transformers), des réseaux de neurones ont été entrainés pour comprendre du langage naturel (texte libre) pour en extraire de l’information. Ces réseaux sont aussi capables de générer du texte similaire à ce qu’un humain pourrait produire.
Ces méthodes permettent d’explorer de façon rétrospective et multimodale (textes et images) l’ensemble des données acquises sur des patients qui n’étaient pas exploitables jusque là/jusqu’à maintenant’à lors et ceci sans avoir besoin de réaliser un travail manuel d’annotation trop important.

\subsubsection{L’exemple des myopathies congénitales et la difficulté du diagnostic}

Les myopathies congénitales (MC) sont une famille de maladies rares et génétiques. Cette maladie peut être causée par une mutation sur un panel de 35 gènes différents et présente une prévalence d’environ 1,5 pour 100 000, soit environ 1000 patients en France. Actuellement, les myopathies congénitales sont différenciées en cinq sous-types : les myopathies à némaline (NM), avec cores (COM), centronucléaire (CNM), avec disproportion congénitale des fibres (CFTD) et sans précision.

L’examen principal permettant la différenciation de ces sous-types est la biopsie musculaire et l’observation de lames au microscope optique et électronique. Cet examen permet de poser un diagnostic et d’orienter le test génétique vers un groupe de gène candidat. 
Cependant encore aujourd’hui, ce diagnostic est compliqué en raison de l’hétérogénéité des manifestations au niveau du muscle entre patients atteints d’un même sous-type de myopathies congénitales. Mais aussi en raison d’un chevauchement important des manifestations phénotypiques entre des sous-types de myopathies congénitales différents. De fait, 50 % des patients atteints de myopathies congénitales n’ont pas de diagnostic génétique à ce jour.  

\subsubsection{IMPatienT : un outil d’annotation et d’exploration de données multimodales de patients.}
Dans ce contexte, en collaboration avec l’institut de myologie de Paris et l’équipe du Dr Teresinha Evangelista, nous avons développé une plateforme en ligne nommée IMPatienT permettant de numériser et d’explorer les données de patients atteints de myopathies congénitales. Plusieurs centaines de rapports de biopsie musculaires ont été générées ces dernières décennies à l’institut de Myologie de Paris. Cette masse de documents contient un nombre important d’informations sur les critères de différenciation des sous-types de MC. Cependant, ces documents sont sous la forme de texte libre semi-structuré. Nous avons donc utilisé un système d’ontologies pour détecter et extraire les concepts clés dans ces rapports d’histologie et être capables d’en faire l’analyse statistique.

La plateforme IMPatienT a été conçue en quatre modules qui dépendant dépendent les uns des autres.
Le premier est le créateur de vocabulaire standard. Il permet aux utilisateurs de créer leur propre arborescence de termes ou de concepts qui peuvent être annotés ensuite comme présent ou absent dans chaque rapport de patients . Chaque terme possède une définition riche : un identifiant unique, des synonymes optionnels et une traduction en anglais. De plus, au fur et à mesure que la base de données intègre des patients, la définition des termes est enrichie avec les gènes et les diagnostics associés et les autres termes qui co-occurrent chez les patients.

Le deuxième module permet de numériser des rapports d’histologie textuels. Il est possible de déposer un rapport d’histologie au format PDF dont le contenu va être analysé par traitement de langage naturel (NLP)  et comparé au vocabulaire standard établi pour annoter automatiquement les concepts présents dans le rapport. Ce système d’analyse supporte aussi la détection de la négation des phrases pour annoter l’absence de concepts. Ces annotations peuvent être affinées à la main en sélectionnant manuellement l’absence et la présence des termes du vocabulaire standard au sein du rapport. De plus le formulaire est connecté aux ontologies déjà existantes pour pouvoir ajouter des métadonnées liées aux rapports comme des symptômes cliniques (ontologie HPO), un gène muté (ontologie HUGO), une mutation (nomenclature HGVS) et un diagnostic final (ontologie OrphaNet). Enfin, un système d’aide au diagnostic est disponible lors de l’ajout d’un patient dans la base de données qui permet de suggérer un diagnostic sur la base de la similarité du profil du patient avec les patients déjà enregistrés dans la base de données par méthodes bayésiennes.

En plus de la numérisation des rapports histologiques, le troisième module permet l’annotation et la segmentation automatique d’images histologiques. Il est possible d’associer des régions de l’image à des termes issus du vocabulaire standard défini. Ce qui donne ensuite lieu à une segmentation automatique de l’ensemble de l’image sur la base de l’intensité, du contraste et de la texture des régions annotées. Cette segmentation permet de préparer un jeu de données annoté pour le développement d’outils de segmentation automatique par IA ou de réaliser de la quantification des régions segmentées.

Enfin, le dernier module correspond au tableau de bord de visualisation automatique. Il génère automatiquement des graphiques et tableaux permettant l’exploration statistique en temps réel des données enregistrées dans la base de données. À ce jour, 90 rapports de patients atteints de myopathies congénitales sont enregistrés dans la base de données d’IMPatienT ce qui a permis la génération d’histogrammes de la répartition de ces patients par âge, gène muté et diagnostic. Une matrice de cooccurrence des termes du vocabulaire standard est aussi générée ainsi que des tableaux de fréquence d’apparition des termes pour chaque gène muté et chaque diagnostic. Enfin, des matrices de confusion permettent d’évaluer en temps réel les performances du système de suggestion de diagnostic.

L’ensemble de ces modules permettent à IMPatienT d’être une plateforme permettant l’annotation et l’exploration de données multimodales de patients atteints de myopathies congénitales. L’approche utilisée par IMPatienT est semi-automatique. Pour les rapports d’histologie, des méthodes de détection des termes dans le texte accélèrent le travail d’annotation, mais un important travail manuel reste nécessaire pour corriger les erreurs réalisées par ces méthodes de détection. Ce qui ne permet pas de traiter un grand volume de données dans un temps raisonnable. Il est alors nécessaire de développer de nouvelles méthodes de détection et d’annotation, basées sur IA pour faciliter l’exploration de ces données.

\subsubsection{Classification des rapports histologiques par IA Explicable (xAI)}

Le concept d’IA explicable (ou xAI pour eXplainable Artificial Intelligence) se réfère à la capacité de comprendre et d’expliquer le fonctionnement des systèmes d’IA (intelligence artificielle) de manière claire et compréhensible pour les êtres humains. C’est une caractéristique importante des modèles IA notamment dans le domaine du diagnostic, car il est nécessaire pour l’H omme d’être en mesure de comprendre sur quels critères une prédiction est réalisée. Les 90 rapports annotés via IMPatienT ont été utilisés pour entrainer un total de 11 algorithmes de ML afin de comparer leurs performances sur des données de la vie réelles. Pour cela, nous avons modifié et utilisé le pipeline Streamline qui entraine, optimise et compare un vaste panel d’algorithme dont les systèmes de classeurs (learning classifier systems, LCS),  considérés comme un système de référence en termes d’explicabilité. Nous avons aussi exploré de nouvelles façons de visualiser les connaissances contenues dans ces LCS pour faciliter l’extraction de connaissances de ces modèles.

\subsubsection{NLMyo : Traitement de rapport textuel par modèles linguistiques de grande taille}

Les avancées importantes de ces dernières années en termes d’IA pour le traitement de texte permettent de faciliter le travail d’analyse d’un grand volume de textes. Grâce aux développements et à la mise à disposition de modèles linguistiques de grandes tailles (LLMs ) comme GPT-3, ChatGPT ou LLAMA, nous avons pu développer NLMyo, un ensemble de quatre outils supplémentaires pour permettre d’exploiter de manière totalement automatique et rapide un grand nombre de rapports textuels.

Le premier outil est un outil d’anonymisation des données. Comme nous travaillons sur des données de santé, il est important de s’assurer que nous ne traitons que des données anonymisées. Ce travail d’anonymisation manuel est long et fastidieux. Ainsi, grâces aux LLMs, nous pouvons détecter automatiquement les nom, prénom ainsi que dates de naissance dans les documents et nous pouvons les censurer avant le traitement de ces données.
Le second outil est un outil d’extraction des métadonnées du rapport. Pour faciliter la numérisation des rapports, il est utile d’extraire de manière automatique des informations commune à chaque rapport. De fait, à partir du texte du rapport, nous sommes capables d’utiliser les LLMs pour extraire automatiquement le numéro de biopsie, l’âge du patient, le muscle prélevé et le diagnostic final. Grâce à une instruction spécifique donnée au LLM, nous extrayons ces informations dans un format informatique standard (JSON) qui permet son traitement de manière automatique pour préremplir les champs des formulaires informatiques utilisés pour numériser les données de patients, par exemple dans IMPatienT.

En troisième outil, nous avons exploré la possibilité de prédire un diagnostic de manière totalement automatique sans aucune annotation humaine à partir du texte brut du rapport et avons obtenu une justesse de classification de 65 % (vs versus 35 % pour le hasard). Cette classification est réalisée grâce à des techniques dites embedding. L’embedding correspond à la transformation par IA d’un texte en un unique vecteur numérique de grande taille (plusieurs centaines voire milliers de dimensions) capable de capturer le sens sémantique du texte. En appliquant cette méthode sur 149 rapports avec des diagnostics différents, nous avons pu entrainer un modèle d’IA capable de prédire le diagnostic associé à un rapport uniquement à partir de son embedding.

Enfin, en quatrième outil, nous avons finalement aussi  utilisé ces techniques d’embedding pour développer un véritable moteur de recherche intelligent de patient. Dans cet outil, l’utilisateur peut formuler une requête en texte libre pour rechercher par exemple des patients ayant un symptôme spécifique. L’embedding de cette requête sera comparé à l’embedding de l’ensemble des phrases contenues dans les rapports d’histologiques, et par calcul de similarité, les rapports avec les meilleurs scores seront présentés en premier comme correspondant à la requête. Cet outil permet de référencer et de rapidement trouver des patients ayant un profil symptomatique spécifique.
L’intégration future de ces méthodes dans IMPatienT pourrait permettre de faciliter la numérisation des patients et gagner un temps important lors de l’annotation de ces patients dans la base de données.

\subsubsection{Vers une génération de rapports automatique à partir d’imagerie avec MyoQuant.}

Grâce aux outils présentés précédemment, nous sommes en mesure d’exploiter les informations contenues dans les rapports histologiques de patients. Cependant, ces rapports sont rédigés à la main après observation de coupe de biopsie musculaire au microscope par un médecin. L’évaluation manuelle des images de biopsie musculaire est couteuse en temps et elle n’est que qualitative : par exemple pour la centralisation nucléaire, un marqueur pathologique typique des MC, il sera noté qu’il y en a peu ou beaucoup, mais sans valeur numérique, car le comptage des fibres individuelles serait trop couteux en temps. Il est donc intéressant d’avoir des outils capables de réaliser ce travail de comptage de façon automatique pour améliorer la précision de l’évaluation des biopsies musculaires et réduire le travail manuel nécessaire.

Nous avons développé MyoQuant, en collaboration avec l’équipe du Dr. Jocelyn Laporte de l’IGBMC à Strasbourg. Cet outil permet de réaliser la quantification automatique de marqueurs pathologiques détectés en routine dans les biopsies musculaires lors du diagnostic des myopathies congénitales. Actuellement MyoQuant est capable de quantifier automatiquement des marqueurs pathologiques dans trois des cinq techniques de coloration réalisée de en routine lors de la biopsie musculaire grâce à des systèmes d’intelligence artificielled’IA.

Pour la coloration hématoxyline-éosine (HE) qui met en évidence les noyaux cellulaires, nous avons développé un algorithme capable d’évaluer le niveau de centralisation de chaque noyau dans les fibres musculaires. Dans une fibre musculaire saine, les noyaux sont localisés en périphéries des fibres. Grâce aux outils existants CellPose et Stardist, qui permettent respectivement de segmenter les fibres et les noyaux cellulaires d’une coupe histologique, nous pouvons calculer pour chaque noyau un score d’excentricité, représentatif de son niveau de centralisation. En appliquant cette procédure à l’ensemble de la coupe, nous pouvons compter automatiquement le nombre de noyaux internalisés ou centralisés. Dans le même temps, l’outil est aussi en mesure d’évaluer la taille des fibres musculaires et de compter le nombre de fibres atrophiques.

Pour la coloration ATPase, qui colore de façon différentielle les fibres de type 1 et les fibres de type 2, nous avons développé une méthode de ML capable de définir automatiquement un ou plusieurs seuils d’intensité qui permet de différencier et compter les fibres de chaque catégorie.
Enfin, pour la coloration au succinate déshydrogénase (SDH) qui mets en évidence l’activité oxydative des fibres, nous avons développé un réseau de neurones de type Resnet50 capable de détecter les fibres ayant une répartition mitochondriale anormale. Entrainé sur un total de 17 000 fibres musculaires issues de 17 souris modèles de myopathies congénitales, notre réseau de neurones obtient une justesse de classification de 93 %.
L’ensemble de ces techniques permettent de quantifier rapidement et automatiquement plusieurs marqueurs pathologiques des myopathies congénitales. Nous souhaitons à présent étendre le champ de détection de MyoQuant en développant des méthodes pour détecter les agrégats protéiques des colorations au trichrome de Gomori (TG) ainsi que les cores  dans la coloration NADH. Ces méthodes pourraient permettre à terme d’être capable de générer automatiquement un rapport de biopsie musculaire plus précis. 

\subsubsection{Conclusions et Perspectives}
Dans le cadre de cette thèse, j’ai eu l’occasion de développer plusieurs outils permettant d’exploiter des données patientes multimodales de patients par approches IA. Avec IMPatienT, j’ai créé une plateforme d’annotation et d’exploration de rapports histologiques de patients qui a permis de numériser une centaine de rapports de patients. Ensuite, avec NLMyo, j’ai exploré comment les récentes avancées en traitement de langage naturel grâce aux LLMs pouvaient faciliter et accélérer l’annotation et la classification de ces rapports textuels. Enfin, avec MyoQuant, j’ai développé un outil qui permet d’accélérer le processus d’évaluation des images de biopsies musculaires avec comme but à terme, d’être capable de générer un rapport histologique de façon automatique.

Les avancées en IA pavent le chemin pour une amélioration du diagnostic des maladies rares. Les outils que j’ai développés dans le cadre de cette thèse sont un exemple de science translationnelle. Sur le plan diagnostic, ces outils peuvent permettre un grain de temps pour les praticiens via l’automatisation de l’extraction d’information lors de l’évaluation des résultats d’analyse. Sur le plan de la recherche, ils peuvent permettre la découverte de nouveaux critères de diagnostic potentiellement important. Il serait maintenant intéressant de compiler l’ensemble de ces outils dans une plateforme unique, cohérente et clé en main utilisable par la communauté pour démocratiser l’usage de ces outils.

\mainmatter
% Chapitre d'introduction (générale)
%%%%%%%%%%%%%%%%%%%%%%%%%%%%%%%%%%%%%%%%%%%%%%%%%%%%%%%%%%%%%%%%%%%%%%%%%%%%%%%

%
% Chapitre d'introduction (générale)
%%%%%%%%%%%%%%%%%%%%%%%%%%%%%%%%%%%%%%%%%%%%%%%%%%%%%%%%%%%%%%%%%%%%%%%%%%%%%%%
% \chapter*{Introduction}
% ...


%
% Chapitres ordinaires (avec parties éventuelles)
%%%%%%%%%%%%%%%%%%%%%%%%%%%%%%%%%%%%%%%%%%%%%%%%%%%%%%%%%%%%%%%%%%%%%%%%%%%%%%%
%
% Première partie éventuelle
% \part{...}
%
% Premier chapitre
% \include{corps/}
%
% Deuxième chapitre
% \include{corps/}
%
% Troisième chapitre
% \include{corps/}
%
%
% Deuxième partie éventuelle
% \part{...}
%
% Quatrième chapitre
% \include{corps/}
%
% Cinquième chapitre
% \include{corps/}
%
% Sixième chapitre
% \include{corps/}
%
% Chapitre  de conclusion (générale)
%%%%%%%%%%%%%%%%%%%%%%%%%%%%%%%%%%%%%%%%%%%%%%%%%%%%%%%%%%%%%%%%%%%%%%%%%%%%%%%
\chapter*{Conclusion}
% ...


%
% Liste des références bibliographiques
\printbibliography
%
%%%%%%%%%%%%%%%%%%%%%%%%%%%%%%%%%%%%%%%%%%%%%%%%%%%%%%%%%%%%%%%%%%%%%%%%%%%%%%%
% Début de la partie annexe éventuelle
%%%%%%%%%%%%%%%%%%%%%%%%%%%%%%%%%%%%%%%%%%%%%%%%%%%%%%%%%%%%%%%%%%%%%%%%%%%%%%%
\appendix
%
% Premier chapitre annexe (éventuel)
% \chapter{...}
% ...


%
% Deuxième chapitre annexe (éventuel)
% % \chapter{...}
% ...


%
%%%%%%%%%%%%%%%%%%%%%%%%%%%%%%%%%%%%%%%%%%%%%%%%%%%%%%%%%%%%%%%%%%%%%%%%%%%%%%%
% Début de la partie finale
%%%%%%%%%%%%%%%%%%%%%%%%%%%%%%%%%%%%%%%%%%%%%%%%%%%%%%%%%%%%%%%%%%%%%%%%%%%%%%%
% \backmatter
%
% (Facultatif) Glossaire (si souhaité distinct de la liste des acronymes) :
% \printglossary
%
% (Facultatif) Index :
% \printindex
%
% Table des matières
%\tableofcontents
%
% (Facultatif) Production de la 4e de couverture :
\makebackcover
%
\end{document}

