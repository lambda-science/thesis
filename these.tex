% Document de classe yathesis
\documentclass[colophon-location=nowhere, output=screen, secnumdepth=subsubsection]{yathesis}
\usepackage[T1]{fontenc}
\usepackage[utf8]{inputenc}
\usepackage{kpfonts}
\usepackage{booktabs}
\usepackage{siunitx}
\usepackage{pgfplots}
\pgfplotsset{compat=1.18} 
\usepackage{caption}
\usepackage{microtype}
\usepackage{varioref}
%\usepackage[xindy,quiet]{imakeidx}
\usepackage[autostyle]{csquotes}
\usepackage[style=apa,backend=biber,safeinputenc]{biblatex}
\usepackage{hyperref}
\usepackage[xindy,acronyms,symbols]{glossaries}
\usepackage{graphicx}

%
% (Facultatif) Génération de l'index (obligatoire si un package d'index, par
% exemple « imakeidx », est chargé)
% \makeindex
%
% (Facultatif) Spécification de la ou des ressources bibliographiques
% (obligatoire si le package « biblatex » est chargé)
\addbibresource{auxiliaires/references.bib}
\addbibresource{auxiliaires/}
%
% (Facultatif) Génération du glossaire (obligatoire si le package « glossaries »
% est chargé)
\makeglossaries
%
% (Facultatif) Configuration des styles du glossaire et de la liste d'acronymes
% (à n'utiliser que si le package « glossaries » est chargé)
\setglossarystyle{indexhypergroup}
\setacronymstyle{long-sc-short}
%
% (Facultatif) Spécification de la ou des ressources terminologiques
% \loadglsentries{auxiliaires/}
% \loadglsentries{auxiliaires/}
% \loadglsentries{auxiliaires/}
%
% Les réglages figurant habituellement dans le préambule, notamment concernant
% la bibliographie et l'éventuel index, peuvent être saisis dans le fichier
% « thesis.cfg » (situé dans le sous-dossier « configuration ») qui est
% automatiquement importé par la classe yathesis.
%
% Importation manuelle du fichier de macros personnelles
\DeclareUnicodeCharacter{2009}{\,}
\input{configuration/macros}
%
%%%%%%%%%%%%%%%%%%%%%%%%%%%%%%%%%%%%%%%%%%%%%%%%%%%%%%%%%%%%%%%%%%%%%%%%%%%%%%%
%%%%%%%%%%%%%%%%%%%%%%%%%%%%%%%%%%%%%%%%%%%%%%%%%%%%%%%%%%%%%%%%%%%%%%%%%%%%%%%
% Début du document
%%%%%%%%%%%%%%%%%%%%%%%%%%%%%%%%%%%%%%%%%%%%%%%%%%%%%%%%%%%%%%%%%%%%%%%%%%%%%%%
%%%%%%%%%%%%%%%%%%%%%%%%%%%%%%%%%%%%%%%%%%%%%%%%%%%%%%%%%%%%%%%%%%%%%%%%%%%%%%%
\begin{document}

%
%%%%%%%%%%%%%%%%%%%%%%%%%%%%%%%%%%%%%%%%%%%%%%%%%%%%%%%%%%%%%%%%%%%%%%%%%%%%%%%
% Caractéristiques du document
%%%%%%%%%%%%%%%%%%%%%%%%%%%%%%%%%%%%%%%%%%%%%%%%%%%%%%%%%%%%%%%%%%%%%%%%%%%%%%%
%
% Préparation des pages de couverture et de titre
%%%%%%%%%%%%%%%%%%%%%%%%%%%%%%%%%%%%%%%%%%%%%%%%%%%%%%%%%%%%%%%%%%%%%%%%%%%%%%%
% Les caractéristiques de la thèse sont saisies dans le fichier
% « characteristics.tex » (situé dans le dossier « configuration »).
%
% Production des pages de couverture et de titre
%%%%%%%%%%%%%%%%%%%%%%%%%%%%%%%%%%%%%%%%%%%%%%%%%%%%%%%%%%%%%%%%%%%%%%%%%%%%%%%
\maketitle

%
%%%%%%%%%%%%%%%%%%%%%%%%%%%%%%%%%%%%%%%%%%%%%%%%%%%%%%%%%%%%%%%%%%%%%%%%%%%%%%%
% Début de la partie liminaire de la thèse
%%%%%%%%%%%%%%%%%%%%%%%%%%%%%%%%%%%%%%%%%%%%%%%%%%%%%%%%%%%%%%%%%%%%%%%%%%%%%%%
%
% (Facultatif) Production de la page de clause de non-responsabilité
% \makedisclaimer
%
% (Facultatif) Production de la page de mots clés
% \makekeywords
%
% (Facultatif) Production de la page affichant les logo, nom et coordonnées du
% ou des laboratoires (ou unités de recherche) où la thèse a été préparée
% \makelaboratory
%
% (Facultatif) Dédicace(s)
% \input{liminaires/dedicaces}
%
% (Facultatif) Épigraphe(s)
% \input{liminaires/epigraphes}
%
% Résumés succincts
\begin{abstract}
La croissance exponentielle des données générées par le séquençage de nouvelle génération, l'imagerie médicale et les dossiers médicaux électroniques met en évidence le besoin d'outils spécialisés pour l'exploitation des données biomédicales multimodales. Nous avons développé des méthodes basées sur l'IA pour explorer les données de patients atteints de myopathies congénitales, une famille de maladies génétiques rares difficiles à diagnostiquer.

Tout d'abord, nous avons développé IMPatienT (Integrated digital Multimodal PATIENt daTa), une base de données de patients en ligne permettant d'annoter et d'explorer manuellement les rapports et les images histologiques des patients. Après avoir numérisés 89 rapports de biopsie nous avons utilisé cette base de données pour comparer les performances de 11 algorithmes de classification.

Ensuite, nous avons développé un outil appelé NLMyo (Natural Language Myopathies) qui s'appuie sur les développements récents dans le domaine du traitement du langage naturel, tels que ChatGPT et LLAMA, pour exploiter des rapports médicaux en texte libre. NLMyo est une boîte à outils qui permet d'anonymiser et d'extraire des informations, de faciliter le diagnostic et de créer automatiquement un moteur de recherche sur les symptômes des patients. Nous avons utilisé NLMyo pour analyser un corpus de 192 rapports de biopsie de patients atteints de myopathies congénitales.

Enfin, nous avons développé un outil appelé MyoQuant pour quantifier automatiquement les marqueurs pathologiques dans les biopsies de fibres musculaires. MyoQuant s'appuie sur des modèles d'IA récents dans le domaine de l'imagerie biomédicale, tels que Cellpose et Stardist. En utilisant des algorithmes et des modèles d'IA personnalisés, il peut quantifier automatiquement les marqueurs pathologiques dans trois colorations standard pour le diagnostic des myopathies congénitales (HE, ATPase et SDH), telles que les noyaux centralisés, le déséquilibre des types de fibres et les anomalies de répartition des mitochondries.

IMPatient, NLMyo et MyoQuant sont disponibles en tant qu'outils open source et également en version démo en ligne.

\end{abstract}
\begin{abstract}
The exponential growth of data generated by next-generation sequencing, medical imaging, and electronic health records highlights the need for specialized tools for the exploitation of multimodal biomedical data. We have developed AI-based methods to explore data from patients with congenital myopathies, rare genetic diseases that are difficult to diagnose.

First, we developed IMPatienT (Integrated digital Multimodal PATIENt daTa), an online patient database to perform manual annotation and exploration of patient histology reports and images. After digitizing 89 biopsy reports we used this database to compare the performance of 11 classification algorithms.

Then we developed a tool called NLMyo (Natural Language Myopathies) that is built on recent development in natural language processing such as ChatGPT and LLAMA to exploit free-text medical reports. NLMyo is a toolbox to anonymize and extract information, facilitate diagnosis and create a patient symptoms search engine automatically. We used NLMyo to analyze a corpus of 192 biopsy reports of patients with congenital myopathies.

Finally, we developed a tool called MyoQuant to automatically quantify pathological features in muscle fiber histology images. MyoQuant is built on recent AI models in biomedical imaging such as Cellpose and Stardist. Using custom algorithms and AI models, it can automatically quantify pathological features in three standard stainings for congenital myopathies diagnosis (HE, ATPase and SDH), such as centralized nuclei, fiber type imbalance and mitochondria repartition anomalies.

IMPatient, NLMyo and MyoQuant are available as open source tools and also as demo version online.


\end{abstract}
\makeabstract


%
% (Facultatif) Chapitre de remerciements
\chapter{Remerciements}
Place-holder pour les remerciements. (\fontencoding{T2A}\selectfont Спасибо\fontencoding{T1}\selectfont  Nastya !

%
% (Facultatif) Chapitre d'avertissement
% \include{liminaires/avertissement}

%
% (Facultatif) Liste des symboles
% \printsymbols
%
%
% Sommaire
\tableofcontents[depth=subsubsection,name=Sommaire]
% (Facultatif) Chapitre de résumé long
\chapter*{Resumé de Thèse}
% ...
\paragraph{\textbf{Intelligence artificielle et classification par machine-learning}}\mbox{}\\

L’intelligence artificielle (IA) représente un ensemble de techniques permettant de créer des programmes simulant l’intelligence humaine.  Une sous-branche de l’IA nommée Machine-Learning (ML) regroupe l’ensemble des algorithmes permettant à un programme informatique d’accomplir une tâche en apprenant d’un jeu de données. Il peut s’agir par exemple de tâche de régression (prédiction d’une valeur comme une durée ou une espérance de vie) ou bien d’une tâche de classification (prédiction d’une catégorie comme un diagnostic). Le but étant d’être capable de réaliser des prédictions sur de nouvelles données jamais rencontrées lors de l’apprentissage. 

Cet apprentissage est le plus souvent dit « supervisé ». C’est-à-dire que pour chaque point de donnée, le label (valeur ou classe à prédire) est connu et l’algorithme apprend simplement à le reproduire. Il est aussi possible de réaliser un apprentissage non supervisé, souvent nommé clustering, lors duquel les labels des données ne sont pas connus, l’algorithme de ML cherche alors à constituer des clusters, c’est-à-dire des groupes de point de données similaires. La prédiction effectuée sera alors l’appartenance à un cluster spécifique.

L’IA  possède un intérêt particulier dans le domaine médical, car elle pourrait  permettre sur le plan du diagnostic un gain de temps et de précision aux professionnels de santé pour poser un diagnostic ou trouver la cause génétique d’une maladie. Sur le plan de la recherche, l’IA peut aussi permettre de découvrir de nouveaux critères de diagnostic pour les maladies difficiles à diagnostiquer. En effet, les techniques d’IA peuvent identifier et utiliser des critères de classification différents de l’Homme, mettant alors en lumière de potentiels nouveaux facteurs discriminants pour le diagnostic de la maladie.

Cependant le développement de modèle de ML est couteux en temps de travail. Le jeu de données d’entrainement est le point central de cette approche, il doit être de suffisamment grande taille et de suffisamment bonne qualité. Les techniques de ML ne peuvent qu’apprendre que de données sous forme de tableaux où chaque ligne représente un point de donnée et chaque colonne un descripteur de ce point de donnée. La constitution d’un tel jeu de données demande un travail d’annotation et de curation conséquent tout en connaissant à l’avance les descripteurs pertinents qui pourront permettre la création d’un modèle performant. De plus cela ferme la porte à l’exploitation de données plus complexes, comme les images ou les textes libres qui peuvent difficilement être représentés sous la forme d’un tableau de données sans perdre d’information ou sans travail manuel et intensif d’annotation.

\paragraph{\textbf{Réseaux de neurones profonds et exploitation de données complexes}}\mbox{}\\

Cette dernière décennie, l’augmentation de la puissance des cartes graphiques (GPU) et l’optimisation des opérations des calculs ont permis la popularisation des réseaux de neurones profonds (deep neural networks, DNN). Les DNN sont une famille d’algorithme de ML reposant sur le concept bio-inspiré des neurones. Ces neurones sont la brique de base des DNN et sont organisés selon une architecture spécifique permettant de mettre en relation plusieurs millions voire milliards de neurones. Les DNN grâce à leur architecture de très grande taille permettent l’exploitation de données complexes sans avoir besoin de connaissance a priori sur les données, c’est-à-dire sans devoir définir des descripteurs pertinents manuellement.
Par exemple, grâce aux architectures de réseau neuronal convolutif (CNN), il est possible d’entrainer un modèle capable de faire la différence entre chien et chat à partir d’images brutes sans avoir à passer par des représentations intermédiaires. 

Grâce aux architectures à base de modules d’attention (Transformers), des réseaux de neurones ont été entrainés pour comprendre du langage naturel (texte libre) pour en extraire de l’information. Ces réseaux sont aussi capables de générer du texte similaire à ce qu’un humain pourrait produire.
Ces méthodes permettent d’explorer de façon rétrospective et multimodale (textes et images) l’ensemble des données acquises sur des patients qui n’étaient pas exploitables jusque là/jusqu’à maintenant’à lors et ceci sans avoir besoin de réaliser un travail manuel d’annotation trop important.

\paragraph{\textbf{L’exemple des myopathies congénitales et la difficulté du diagnostic}}\mbox{}\\

Les myopathies congénitales (MC) sont une famille de maladies rares et génétiques. Cette maladie peut être causée par une mutation sur un panel de 35 gènes différents et présente une prévalence d’environ 1,5 pour 100 000, soit environ 1000 patients en France. Actuellement, les myopathies congénitales sont différenciées en cinq sous-types : les myopathies à némaline (NM), avec cores (COM), centronucléaire (CNM), avec disproportion congénitale des fibres (CFTD) et sans précision.

L’examen principal permettant la différenciation de ces sous-types est la biopsie musculaire et l’observation de lames au microscope optique et électronique. Cet examen permet de poser un diagnostic et d’orienter le test génétique vers un groupe de gène candidat. 
Cependant encore aujourd’hui, ce diagnostic est compliqué en raison de l’hétérogénéité des manifestations au niveau du muscle entre patients atteints d’un même sous-type de myopathies congénitales. Mais aussi en raison d’un chevauchement important des manifestations phénotypiques entre des sous-types de myopathies congénitales différents. De fait, 50 \% des patients atteints de myopathies congénitales n’ont pas de diagnostic génétique à ce jour.  

\paragraph{\textbf{IMPatienT : un outil d’annotation et d’exploration de données multimodales de patients}}\mbox{}\\

Dans ce contexte, en collaboration avec l’institut de myologie de Paris et l’équipe du Dr Teresinha Evangelista, nous avons développé une plateforme en ligne nommée IMPatienT permettant de numériser et d’explorer les données de patients atteints de myopathies congénitales. Plusieurs centaines de rapports de biopsie musculaires ont été générées ces dernières décennies à l’institut de Myologie de Paris. Cette masse de documents contient un nombre important d’informations sur les critères de différenciation des sous-types de MC. Cependant, ces documents sont sous la forme de texte libre semi-structuré. Nous avons donc utilisé un système d’ontologies pour détecter et extraire les concepts clés dans ces rapports d’histologie et être capables d’en faire l’analyse statistique.

La plateforme IMPatienT a été conçue en quatre modules qui dépendant dépendent les uns des autres.
Le premier est le créateur de vocabulaire standard. Il permet aux utilisateurs de créer leur propre arborescence de termes ou de concepts qui peuvent être annotés ensuite comme présent ou absent dans chaque rapport de patients . Chaque terme possède une définition riche : un identifiant unique, des synonymes optionnels et une traduction en anglais. De plus, au fur et à mesure que la base de données intègre des patients, la définition des termes est enrichie avec les gènes et les diagnostics associés et les autres termes qui co-occurrent chez les patients.

Le deuxième module permet de numériser des rapports d’histologie textuels. Il est possible de déposer un rapport d’histologie au format PDF dont le contenu va être analysé par traitement de langage naturel (NLP)  et comparé au vocabulaire standard établi pour annoter automatiquement les concepts présents dans le rapport. Ce système d’analyse supporte aussi la détection de la négation des phrases pour annoter l’absence de concepts. Ces annotations peuvent être affinées à la main en sélectionnant manuellement l’absence et la présence des termes du vocabulaire standard au sein du rapport. De plus le formulaire est connecté aux ontologies déjà existantes pour pouvoir ajouter des métadonnées liées aux rapports comme des symptômes cliniques (ontologie HPO), un gène muté (ontologie HUGO), une mutation (nomenclature HGVS) et un diagnostic final (ontologie OrphaNet). Enfin, un système d’aide au diagnostic est disponible lors de l’ajout d’un patient dans la base de données qui permet de suggérer un diagnostic sur la base de la similarité du profil du patient avec les patients déjà enregistrés dans la base de données par méthodes bayésiennes.

En plus de la numérisation des rapports histologiques, le troisième module permet l’annotation et la segmentation automatique d’images histologiques. Il est possible d’associer des régions de l’image à des termes issus du vocabulaire standard défini. Ce qui donne ensuite lieu à une segmentation automatique de l’ensemble de l’image sur la base de l’intensité, du contraste et de la texture des régions annotées. Cette segmentation permet de préparer un jeu de données annoté pour le développement d’outils de segmentation automatique par IA ou de réaliser de la quantification des régions segmentées.

Enfin, le dernier module correspond au tableau de bord de visualisation automatique. Il génère automatiquement des graphiques et tableaux permettant l’exploration statistique en temps réel des données enregistrées dans la base de données. À ce jour, 90 rapports de patients atteints de myopathies congénitales sont enregistrés dans la base de données d’IMPatienT ce qui a permis la génération d’histogrammes de la répartition de ces patients par âge, gène muté et diagnostic. Une matrice de cooccurrence des termes du vocabulaire standard est aussi générée ainsi que des tableaux de fréquence d’apparition des termes pour chaque gène muté et chaque diagnostic. Enfin, des matrices de confusion permettent d’évaluer en temps réel les performances du système de suggestion de diagnostic.

L’ensemble de ces modules permettent à IMPatienT d’être une plateforme permettant l’annotation et l’exploration de données multimodales de patients atteints de myopathies congénitales. L’approche utilisée par IMPatienT est semi-automatique. Pour les rapports d’histologie, des méthodes de détection des termes dans le texte accélèrent le travail d’annotation, mais un important travail manuel reste nécessaire pour corriger les erreurs réalisées par ces méthodes de détection. Ce qui ne permet pas de traiter un grand volume de données dans un temps raisonnable. Il est alors nécessaire de développer de nouvelles méthodes de détection et d’annotation, basées sur IA pour faciliter l’exploration de ces données.

\paragraph{\textbf{Classification des rapports histologiques par IA Explicable (xAI)}}\mbox{}\\

Le concept d’IA explicable (ou xAI pour eXplainable Artificial Intelligence) se réfère à la capacité de comprendre et d’expliquer le fonctionnement des systèmes d’IA (intelligence artificielle) de manière claire et compréhensible pour les êtres humains. C’est une caractéristique importante des modèles IA notamment dans le domaine du diagnostic, car il est nécessaire pour l’H omme d’être en mesure de comprendre sur quels critères une prédiction est réalisée. Les 90 rapports annotés via IMPatienT ont été utilisés pour entrainer un total de 11 algorithmes de ML afin de comparer leurs performances sur des données de la vie réelles. Pour cela, nous avons modifié et utilisé le pipeline Streamline qui entraine, optimise et compare un vaste panel d’algorithme dont les systèmes de classeurs (learning classifier systems, LCS),  considérés comme un système de référence en termes d’explicabilité. Nous avons aussi exploré de nouvelles façons de visualiser les connaissances contenues dans ces LCS pour faciliter l’extraction de connaissances de ces modèles.

\paragraph{\textbf{NLMyo : Traitement de rapport textuel par modèles linguistiques de grande taille}}\mbox{}\\

Les avancées importantes de ces dernières années en termes d’IA pour le traitement de texte permettent de faciliter le travail d’analyse d’un grand volume de textes. Grâce aux développements et à la mise à disposition de modèles linguistiques de grandes tailles (LLMs ) comme GPT-3, ChatGPT ou LLAMA, nous avons pu développer NLMyo, un ensemble de quatre outils supplémentaires pour permettre d’exploiter de manière totalement automatique et rapide un grand nombre de rapports textuels.

Le premier outil est un outil d’anonymisation des données. Comme nous travaillons sur des données de santé, il est important de s’assurer que nous ne traitons que des données anonymisées. Ce travail d’anonymisation manuel est long et fastidieux. Ainsi, grâces aux LLMs, nous pouvons détecter automatiquement les nom, prénom ainsi que dates de naissance dans les documents et nous pouvons les censurer avant le traitement de ces données.
Le second outil est un outil d’extraction des métadonnées du rapport. Pour faciliter la numérisation des rapports, il est utile d’extraire de manière automatique des informations commune à chaque rapport. De fait, à partir du texte du rapport, nous sommes capables d’utiliser les LLMs pour extraire automatiquement le numéro de biopsie, l’âge du patient, le muscle prélevé et le diagnostic final. Grâce à une instruction spécifique donnée au LLM, nous extrayons ces informations dans un format informatique standard (JSON) qui permet son traitement de manière automatique pour préremplir les champs des formulaires informatiques utilisés pour numériser les données de patients, par exemple dans IMPatienT.

En troisième outil, nous avons exploré la possibilité de prédire un diagnostic de manière totalement automatique sans aucune annotation humaine à partir du texte brut du rapport et avons obtenu une justesse de classification de 65 \% (vs versus 35 \% pour le hasard). Cette classification est réalisée grâce à des techniques dites embedding. L’embedding correspond à la transformation par IA d’un texte en un unique vecteur numérique de grande taille (plusieurs centaines voire milliers de dimensions) capable de capturer le sens sémantique du texte. En appliquant cette méthode sur 149 rapports avec des diagnostics différents, nous avons pu entrainer un modèle d’IA capable de prédire le diagnostic associé à un rapport uniquement à partir de son embedding.

Enfin, en quatrième outil, nous avons finalement aussi  utilisé ces techniques d’embedding pour développer un véritable moteur de recherche intelligent de patient. Dans cet outil, l’utilisateur peut formuler une requête en texte libre pour rechercher par exemple des patients ayant un symptôme spécifique. L’embedding de cette requête sera comparé à l’embedding de l’ensemble des phrases contenues dans les rapports d’histologiques, et par calcul de similarité, les rapports avec les meilleurs scores seront présentés en premier comme correspondant à la requête. Cet outil permet de référencer et de rapidement trouver des patients ayant un profil symptomatique spécifique.
L’intégration future de ces méthodes dans IMPatienT pourrait permettre de faciliter la numérisation des patients et gagner un temps important lors de l’annotation de ces patients dans la base de données.

\paragraph{\textbf{Vers une génération de rapports automatique à partir d’imagerie avec MyoQuant}}\mbox{}\\

Grâce aux outils présentés précédemment, nous sommes en mesure d’exploiter les informations contenues dans les rapports histologiques de patients. Cependant, ces rapports sont rédigés à la main après observation de coupe de biopsie musculaire au microscope par un médecin. L’évaluation manuelle des images de biopsie musculaire est couteuse en temps et elle n’est que qualitative : par exemple pour la centralisation nucléaire, un marqueur pathologique typique des MC, il sera noté qu’il y en a peu ou beaucoup, mais sans valeur numérique, car le comptage des fibres individuelles serait trop couteux en temps. Il est donc intéressant d’avoir des outils capables de réaliser ce travail de comptage de façon automatique pour améliorer la précision de l’évaluation des biopsies musculaires et réduire le travail manuel nécessaire.

Nous avons développé MyoQuant, en collaboration avec l’équipe du Dr. Jocelyn Laporte de l’IGBMC à Strasbourg. Cet outil permet de réaliser la quantification automatique de marqueurs pathologiques détectés en routine dans les biopsies musculaires lors du diagnostic des myopathies congénitales. Actuellement MyoQuant est capable de quantifier automatiquement des marqueurs pathologiques dans trois des cinq techniques de coloration réalisée de en routine lors de la biopsie musculaire grâce à des systèmes d’intelligence artificielled’IA.

Pour la coloration hématoxyline-éosine (HE) qui met en évidence les noyaux cellulaires, nous avons développé un algorithme capable d’évaluer le niveau de centralisation de chaque noyau dans les fibres musculaires. Dans une fibre musculaire saine, les noyaux sont localisés en périphéries des fibres. Grâce aux outils existants CellPose et Stardist, qui permettent respectivement de segmenter les fibres et les noyaux cellulaires d’une coupe histologique, nous pouvons calculer pour chaque noyau un score d’excentricité, représentatif de son niveau de centralisation. En appliquant cette procédure à l’ensemble de la coupe, nous pouvons compter automatiquement le nombre de noyaux internalisés ou centralisés. Dans le même temps, l’outil est aussi en mesure d’évaluer la taille des fibres musculaires et de compter le nombre de fibres atrophiques.

Pour la coloration ATPase, qui colore de façon différentielle les fibres de type 1 et les fibres de type 2, nous avons développé une méthode de ML capable de définir automatiquement un ou plusieurs seuils d’intensité qui permet de différencier et compter les fibres de chaque catégorie.
Enfin, pour la coloration au succinate déshydrogénase (SDH) qui mets en évidence l’activité oxydative des fibres, nous avons développé un réseau de neurones de type Resnet50 capable de détecter les fibres ayant une répartition mitochondriale anormale. Entrainé sur un total de 17 000 fibres musculaires issues de 17 souris modèles de myopathies congénitales, notre réseau de neurones obtient une justesse de classification de 93 %.
L’ensemble de ces techniques permettent de quantifier rapidement et automatiquement plusieurs marqueurs pathologiques des myopathies congénitales. Nous souhaitons à présent étendre le champ de détection de MyoQuant en développant des méthodes pour détecter les agrégats protéiques des colorations au trichrome de Gomori (TG) ainsi que les cores  dans la coloration NADH. Ces méthodes pourraient permettre à terme d’être capable de générer automatiquement un rapport de biopsie musculaire plus précis. 

\paragraph{\textbf{Conclusions et Perspectives}}\mbox{}\\

Dans le cadre de cette thèse, j’ai eu l’occasion de développer plusieurs outils permettant d’exploiter des données patientes multimodales de patients par approches IA. Avec IMPatienT, j’ai créé une plateforme d’annotation et d’exploration de rapports histologiques de patients qui a permis de numériser une centaine de rapports de patients. Ensuite, avec NLMyo, j’ai exploré comment les récentes avancées en traitement de langage naturel grâce aux LLMs pouvaient faciliter et accélérer l’annotation et la classification de ces rapports textuels. Enfin, avec MyoQuant, j’ai développé un outil qui permet d’accélérer le processus d’évaluation des images de biopsies musculaires avec comme but à terme, d’être capable de générer un rapport histologique de façon automatique.

Les avancées en IA pavent le chemin pour une amélioration du diagnostic des maladies rares. Les outils que j’ai développés dans le cadre de cette thèse sont un exemple de science translationnelle. Sur le plan diagnostic, ces outils peuvent permettre un grain de temps pour les praticiens via l’automatisation de l’extraction d’information lors de l’évaluation des résultats d’analyse. Sur le plan de la recherche, ils peuvent permettre la découverte de nouveaux critères de diagnostic potentiellement important. Il serait maintenant intéressant de compiler l’ensemble de ces outils dans une plateforme unique, cohérente et clé en main utilisable par la communauté pour démocratiser l’usage de ces outils.

\paragraph{\textbf{Publications et communications}}\mbox{}\\
* la liste d’articles publiés, acceptés, soumis ou en cours de rédaction sous la forme classique : le titre, tous les auteurs, nom du journal, références,
* la liste des communications orales et par affiches : le titre, tous les auteurs, les coordonnées de la réunion scientifique (titre, date, lieu).
%
% (Facultatif) Table des figures
\listoffigures
%
% (Facultatif) Liste des tableaux
\listoftables
%
\newglossaryentry{mc}{
    name={MC},
    description={Myopathies congénitales},
    first={myopathies congénitales (MC)}
}
\newglossaryentry{dnn}{
    name={DNN},
    description={Réseaux de neurones profonds},
    first={réseaux de neurones profonds (\textit{Deep Neural Networks}, DNN)}
}
\newglossaryentry{nlp}{
    name={NLP},
    description={Traitement de langage naturel},
    first={traitement de langage naturel (\textit{Natural Language Processing}, NLP)}
}
\newglossaryentry{ml}{
    name={ML},
    description={Machine-Learning},
    first={machine-learning (ML)}
}
\newglossaryentry{ia}{
    name={IA},
    description={Intelligence artificielle},
    first={intelligence artificielle (IA)}
}
\newglossaryentry{xai}{
    name={xIA},
    description={Intelligence Artificielle Explicable},
    first={\textit{eXplainable Artificial Intelligence} (xAI)}
}
\newglossaryentry{cnn}{
    name={CNN},
    description={Réseau neuronal convolutif},
    first={réseau neuronal convolutif (\textit{Convolutional Neural Networks}, CNN)}
}
\newglossaryentry{llms}{
    name={LLMs},
    description={Modèles linguistiques de grande taille},
    first={modèles linguistiques de grande taille (\textit{Large Language Models}, LLMs)}
}
\newglossaryentry{nm}{
    name={NM},
    description={Myopathies à némaline},
    first={myopathies à Némaline (NM)}
}
\newglossaryentry{com}{
    name={COM},
    description={Myopathies à \textit{cores}},
    first={myopathies à \textit{cores} (COM)}
}
\newglossaryentry{cnm}{
    name={CNM},
    description={Myopathies centro-nucléaires},
    first={myopathies centro-nucléaires (CNM)}
}
\newglossaryentry{cftd}{
    name={CFTD},
    description={Myopathies à disproportion congénitale des fibres},
    first={myopathies à disproportion congénitale des fibres (CFTD)}
}
\newglossaryentry{hpo}{
    name={HPO},
    description={\textit{Human Phenotype Ontology}},
    first={\textit{Human Phenotype Ontology}, HPO}
}
\newglossaryentry{hgnc}{
    name={HGNC},
    description={\textit{HUGO Gene Nomenclature Committee}},
    first={\textit{HUGO Gene Nomenclature Committee, HGNC}}
}
\newglossaryentry{hgvs}{
    name={NLP},
    description={\textit{Human Genome Variation Society}},
    first={\textit{Human Genome Variation Society}, HGVS}
}
\newglossaryentry{ordo}{
    name={ORDO},
    description={\textit{Orphanet Rare Disease Ontology}},
    first={\textit{Orphanet Rare Disease Ontology}, ORDO}
}
\newglossaryentry{lcs}{
    name={LCS},
    description={Système de classeurs},
    first={système de classeurs (\textit{Learning Classifier Systems}, LCS)}
}
\newglossaryentry{impatient}{
    name={\textbf{IMPatienT}},
    description={\textit{Integrated digital Multimodal PATIENt daTa}},
    first={\textbf{IMPatienT} (\textit{Integrated digital Multimodal PATIENt daTa})}
}
\newglossaryentry{myoquant}{
    name={\textbf{MyoQuant}},
    description={outil de quantificaiton automatique de caractéristiques pathologiques dans les images histologiques des fibres musculaires},
    first={\textbf{MyoQuant}}
}
\newglossaryentry{nlmyo}{
    name={\textbf{NLMyo}},
    description={\textit{Natural Language Myopathies}},
    first={\textbf{NLMyo} (\textit{Natural Language Myopathies})}
}
\newglossaryentry{he}{
    name={HE},
    description={Hématoxyline-Eosine},
    first={Hématoxyline-Eosine (HE)}
}
\newglossaryentry{sdh}{
    name={SDH},
    description={Succinate déshydrogénase },
    first={Succinate Déshydrogénase (SDH)}
}
\newglossaryentry{tg}{
    name={TG},
    description={Trichrome de Gomori},
    first={Trichrome de Gomori (TG)}
}
% (Facultatif) Glossaire (si souhaité distinct de la liste des acronymes) :
% \printglossary[type=\acronymtype]
\printglossary
% \include{liminaires/abbreviations}
% (Facultatif) Liste des acronymes
% \printacronyms
%
% (Facultatif) Table des listings (nécessite que le package « listings » soit
% chargé)
% \lstlistoflistings
%

% (Facultatif) Chapitre d'avant-propos
\chapter{Avant-Propos}
L'objectif de cette thèse est de développer des méthodes et des outils d'intelligence artificielle adaptés à l'exploitation des données biomédicales multimodales, en particulier les comptes rendus médicaux et les données d'imagerie avec une application aux myopathies congénitales. Cette thèse a donné lieu aux développement de trois outils mettant à disposition les moyens nécessaires à l'exploitation de ces données. \\


L'\textbf{introduction} est séparée en trois chapitres (\textbf{chapitres 1-3}). Les deux premiers chapitre posent le contexte informatique dans lequel s'inscrivent ces travaux. 

Le \textbf{premier chapitre} s'intéresse à l'émergence du \textit{Big-Data} dans le cadre des données bio-médicale et décrit les principes de bases et les approches traditionnelle utilisée en intelligence artificielle pour analyser les données.

Le \textbf{second chapitre} d'introduction s'intéresse plus spécifiquement aux \gls{ia} de type réseaux de neurones et comment cette nouvelle technologie transforme la manière de traiter les données biomédicale multimodales avec des exemples de d'\gls{ia} déjà existantes en imagerie et en analyse de texte.

Enfin le \textbf{troisième chapitre} d'introduction pose le contexte biologique de la thèse avec une présentation de notre cas d'application: les myopathies congénitales. On y retrouve d'abord une présentation générale du muscle puis une description des myopathies congénitale, de leur diagnostic et des données générée suite à celui-ci. \\


Le \textbf{quatrième chapitre} \textbf{matériels et méthodes} décrit l'ensemble des ressources utilisées pour développer ces outils. On y retrouvera d'abord une description des ressources biologiques comme les ontologies et source de données biomédicales utilisée. Puis l'ensemble des ressources informatique nécessaire au développement des outils tels que les algorithmes utilisés, les bibliothèques de programmation, le matériel informatique, les modèles pré-existants ainsi que les méthodes d'évaluation des performances. Enfin dans une dernière section, l'accent est mis sur les outils permettant de rendre ce développement \textit{open-source} et reproductible. \\


La partie \textbf{contributions} est divisée en quatre chapitres (\textbf{chapitres 5-8}). 

Le \textbf{chapitre 5 }présente \gls{impatient}, le premier outil développé durant cette thèse, qui est une base de données d'annotations de comptes rendus et d'images de patients. Cet outil a fait l'objet d'une soumission dans un journal à comité de lecture dont le manuscrit est intégré au chapitre. 

Le \textbf{chapitre 6} s'intéresse à l'analyse détaillée de 89 comptes rendus de patients intégré dans cette base de données et leur classification par \gls{ia} traditionnelle et \gls{ia} explicable.

Le \textbf{chapitre 7} présente \gls{nlmyo}, une boite à outil basé sur les modèles linguistiques de grande tailles pour l'exploitation automatique des comptes rendus médicaux. Cette boite à outil permet d'anonymiser et d'extraire de l'information de comptes rendus textuels, ainsi que de les classer et de créer un moteur de recherche de symptomes automatiquement. 

Enfin le \textbf{chapitre 8 }présente la possibilité de générer de manière automatique des comptes rendus de biopsie grâce à des méthodes de quantification par intelligence artificielle. Pour cela nous avons développé \gls{myoquant}, un outils de quantification automatique de marqueurs pathologiques sur des biopsie musculaire. \\


Pour finir le \textbf{chapitre 9 discussions et ouvertures} traite des principaux challenges, limites et perspectives des outils développés. Tout d'abord des perspectives biologiques avec l'intégration des données génomiques et la mise en relation des différentes modalités. Mais aussi des perspectives techniques notamment en terme d'explicabilité de l'\gls{ia}, du déploiement de ce type d'approche à grande échelle en terme de ressources informatique et des questions de législation en terme de données de santé et de traitement automatique. Une ouverture supplémentaire est fait sur la possibilité de création d'une produit regroupant l'ensemble de ces outils dans le cadre du concours \textit{Mature Your PhD} organisé par la Satt Connectus.
%%%%%%%%%%%%%%%%%%%%%%%%%%%%%%%%%%%%%%%%%%%%%%%%%%%%%%%%%%%%%%%%%%%%%%%%%%%%%%%
% Début de la partie principale (du « corps ») de la thèse
%%%%%%%%%%%%%%%%%%%%%%%%%%%%%%%%%%%%%%%%%%%%%%%%%%%%%%%%%%%%%%%%%%%%%%%%%%%%%%%
\mainmatter
% Chapitre d'introduction (générale)
%%%%%%%%%%%%%%%%%%%%%%%%%%%%%%%%%%%%%%%%%%%%%%%%%%%%%%%%%%%%%%%%%%%%%%%%%%%%%%%

%
% Chapitre d'introduction (générale)
%%%%%%%%%%%%%%%%%%%%%%%%%%%%%%%%%%%%%%%%%%%%%%%%%%%%%%%%%%%%%%%%%%%%%%%%%%%%%%%
% \include{corps/introduction}
%
% Chapitres ordinaires (avec parties éventuelles)
%%%%%%%%%%%%%%%%%%%%%%%%%%%%%%%%%%%%%%%%%%%%%%%%%%%%%%%%%%%%%%%%%%%%%%%%%%%%%%%
%
% Première partie éventuelle
\part{INTRODUCTION}
\chapter{ \textit{Big Data}, données biomédicales et \textit{machine-learning}}

Hiding within those mounds of data is knowledge that could change the life of a patient, or change the world
- Atul Butte, 2012

L'informatisation du monde a permis la production et l'accumulation de données de façon exponentielle. En particulier dans le domaine de la santé, les avancées technologiques comme les technologies de séquençage, d'imagerie ou les dossiers médicaux électroniques ont permis au cours du temps la capture de données précieuses pouvant améliorer le parcours et la prise en charge des patients.

Dans le même temps, les technologies d'analyse de données massives (\textit{Big-Data}) se sont développée notamment grâce au \gls{ml}, une branche de l'\gls{ia} permettant à des algorithmes informatiques d'apprendre à partir des données. Cette massification des données biomédicales et le développement des technologies d'analyse de données permet d'entrevoir un monde où les soins de santé pourraient être personalisés, préventifs et prédictifs .

Dans ce premier chapitre d'introduction nous présenterons d'abord ce que sont les big-data, la variété des données-biomédicales et comment ces données sont utile pour une meilleure prise en charge des patients.. Puis dans une seconde partie, nous présenterons les concepts principaux du \gls{ml} pour le traitement de ces données.

\section{Les données biomédicale: des \textit{Big Data} au service des patients}

\subsection{Définiton du \textit{Big-Data}}

Le terme "Big Data" est utilisé pour faire référence à l'immense quantité de données complexes et hétérogènes produites par cette informatisation du monde et le développement des technologies haut débits (\cite{de_mauro_formal_2016}). La définition des big-data a été enrichie au fur et à mesure des années d'abord par 3 mots clés puis 5 et même jusqu'à 7 (\cite{garcia_what_2022}). La définition la plus commune aujourd'hui des Big-Data se compose des 5 "V": le volume, la variété, la vélocité, la véracité et la valeur (\cite{ishwarappa_brief_2015}). Ainsi pour être considérée comme Big-Data, les données doivent: (i) être volumineuse, du gigabytes à l'exabytes, (ii) être variée c'est à dire multimodales (iii) avoir une vélocité de création et de traitement importante, (iv) être vérace, c'est à dire valide et (v) avoir une forte valeur ajoutée, c'est-à-dire qu'elle doivent être utiles. (\cite{garcia_what_2022}).

Ainsi les données biomédicales sont en adéquation à cette définition (\cite{zheng_application_2021}). Grâce aux améliorations en techniques d'acquisitions (séquençage, imagerie) elles sont volumineuses et possèdent une forte vélocité. Ces données sont variées, contenant des informations sur le plan génétique, phénotypique et histologique. De plus elle ont un véracité assurée couplée à une forte valeur ajoutée. En effet ce sont de manière général des données générée par des experts (médecin ou biologiste) et liées à des patients (et donc fortement valorisées). Il est alors juste de parler de qualifier les données biomédicales de Big-Data (\cite{sonawane_network_2019}. Ainsi dans la prochaine section nous allons voir en détails les différents modalités de données biomédicales de patients, leur acquisitions et les challenges que présentent leur analyse.

\subsection{Variété des données biomédicales}
Les données biomédicales sont par nature multimodales (\cite{acosta_multimodal_2022}). Le diagnostic d'un patient peut se réaliser par l'intégration de différents niveau d'informations. Tout d'abord il y a les données phénotypiques, listant les symptômes et autre caractéristiques du patient après un examen médical. Ces données sont souvent sous la forme de texte libre, rédigé par le praticien de santé. Ensuite les données d'imagerie, issues le plus souvent d'examens complémentaire pour mieux caractériser l'atteinte du patient (échographie, IRM, histopathologie). Et enfin il y a les données génétiques et omiques, qui sont nécessaires dans le cadre de maladie génétiques pour cibler les dysfonctionnement d'origine génétiques (données de types séquences). Ce trio texte libre, imagerie et séquences implique des techniques d'acquisition, de traitement et des difficultés propres.

\subsubsection{Données textuelles et dossiers de santé électroniques}
Les données en texte libre sont très commune dans le cadre des données de santé. Un rendez-vous, un examen complémentaire ou un échange entre confrères, peuvent donner lieu à la rédaction de comptes rendus médicaux en texte libre contenant des informations expertisées et hautement valorisée concernant le diagnostic du patient. De plus, la rédaction de comptes-rendus ne nécessite pas de technologie d'acquisition particulière, ce qui a permis l'accumulation au cours du temps d'archive de comptes-rendus médicaux massive qui restent à explorer.

Dans le cadre de cette volonté d'explorer ces données, les \gls{dse} sont des outils pour numériser ces comptes-rendus textuels afin de les centraliser et les exploiter (\cite{graber_impact_2017}). Cependant le développement d'outils pour numériser et exploiter les comptes rendus en texte libre est difficile. La compréhension du texte libre par un programme informatique est une tâches ardue. C'est pourquoi la majorité des solutions de \gls{dse} demandent une phase d'annotation manuelle (remplissage de formulaires et de champs) par l'utilisateur pour numériser les données, ce qui est pratique est rarement réalisé faute de temps.

\subsubsection{Données d'imagerie: microscopie à haute résolution}
Le développement des techniques d'imagerie a permis une diversification des techniques d'imagerie (IRM, échographie, microscopie optique et électronique, imagerie 2D et 3D...) tout en améliorant leur résolutions et précision de capture  et en réduisant les coûts associés (\cite{abdallah_history_2017, prakash_super-resolution_2022, sheppard_structured_2021}). Ainsi l'imagerie médicale est devenue un examen de routine pour le diagnostic de divers pathologies. Cette production de données d'imagerie en routine et de grande résolution ont donné lieu à une massification des données d'imagerie. Ainsi il peut-être pour un clinicien d'évaluer manuellement ces données de manière exhaustive. Le développement d'outils capable d'analyser et de quantifier les éléments d'intérêt sur les données d'imagerie est donc un enjeu majeur (\cite{tchito_tchapga_biomedical_2021}) pour à la fois accélérer l'évaluation des données d'imagerie mais aussi pour améliorer la précision des cliniciens. Par exemple, dans le cadre de la microscopie photonique, il est maintenant courant d'utiliser des scanner de lame complète, générant ainsi des images à l'échelle du gigabytes par lame. Ces images sont extrêmement coûteuses en temps s'il on veut réaliser une évaluation manuelle exhaustive et un comptage des caractéristiques pathologiques en vue d'un diagnostic.

\subsubsection{Données génétiques et omiques}
Enfin, les progrès en termes de technologies de séquençage grâces notamment aux technologie de seconde génération à lecture courtes (technologie Illumina) et aux technologie de troisième générations à lecture longue (technologie PacBio et Nanopore) ont permis l'accès à l'ensemble des informations génétiques et omiques de l'Homme. De plus la baisse des coûts de séquençage, rend possible l'utilisation de séquençage de génome complet pour le diagnostic de maladies génétique chez les patients (\cite{rabbani_next-generation_2012}). Plus récemment, les techniques dites "omiques" (figure \ref{fig:intro-omics}, \cite{momeni_survey_2020}) sont utilisée pour mieux comprendre les pathologiques tels que les technologies de tanscriptiomiques (expression ARN des gènes), épigénétiques, protéomiques (expression protéiques des gènes) et métabolomiques (étude des métabolites). Ces technologies d'obtenir une vue globales de mécanismes biologiques qui opèrent au sein d'un tissus. 
\begin{figure}[!htbp]
 \centering
 \includegraphics[width=1\textwidth]{figures/intro_omics.png}
 \caption[Méthodes de séquençages "omiques"]{Schéma des différentes méthodes de séquençages "omiques" donnant accès à une vue globales des mécanismes biologiques dans les tissues biologiques. (Modifié de \cite{momeni_survey_2020})}
 \label{fig:intro-omics}
\end{figure}
Les données de séquençages sont massives, le génome humaine mesurant environ 3.1 miliards de paires de bases (bp), le séquençage d'un génome unique avec une profondeur de 50X (nécessaire pour la détection de mutation génétique) représente un minimum de 150 miliards de paires de bases lues et stockées, pour un individus. Ces données de séquence massives, requiert des outils spécifique et un matériel informatique adapté à leur traitement. De plus la détection de mutations pathogène est complexe. L'identification du gène responsable d'une maladie génétique reste un challenge lors du diagnostique, même avec les données de séquençage complètes.

La collecte et l'intégration de l'ensemble de ces données biomédicales générées grace aux nouvelles tehcnologies haut-débit en imagerie, séquençage et \gls{dse} ont le potentiel d'améliorer la compréhension des maladies et la prise en charge des patients.

\subsection{Collecte et utilisation des données biomédicales au service du patient}
Le Royaume-Uni est un pays pionnier dans la collecte et la mise à disposition de façon massive de données biomédicales à travers le \gls{nhs}. Cela s'illustre par exemple par le projet 100 000 génomes, lancé en 2012 qui a pour but de séquencer 100 000 génomes de patient anglais pour améliorer la recherche et le diagnostic de maladie rares, certains cancers et maladies infectieuses (\cite{nunn_public_2019}). En avril 2022 a été publié le rapport de 112 pages intitulé: "\textit{Better, broader, safer: using health data for research and analysis}" (\cite{ben_goldacre_better_2022}), écrit par le professeur Ben Goldacre missioné par le \gls{nhs}. Ce rapport mets en évidence le challenge et la stratégie à adopter pour une collecte et un usage à grand échelle de données biomédicales de patients. Le projet OpenSAFELY (\href{https://www.opensafely.org/}{https://www.opensafely.org/}), fondé par Ben Goldacre, est un exemple concret d'utilisation de données biomédicales au service de la recherche et de la prise en charge de patients. Ce projet, créé en juin 2020 pour lutter contre la pandémie de COVID-19, mets à disposition des chercheurs des outils et des données biomédicales massives de patients. A ce jour, ce projet a permis la publication des plus de 80 publications scientifique de recherches réalisés à partir de ces données. Des initiatives similaires  mettent en évidence l'utilité des big-data biomédicales comme catalyseur de découvertes scientifiques, tel que le projet "Big Data to Knowledge" fondé par le \textit{National Institutes of Health (NIH)} (\cite{toga_big_2015}).

Outre la phase de collecte, la difficulté dans l'exploitation des données biomédicale réside dans la disponibilité techniques d'analyse adaptées (\cite{wang_big_2019, ismail_requirements_2020}). Les données biomédicales étant volumineuses, complexes et multimodales, leur exploration manuelle ou via des techniques statistique de base ne sont pas suffisantes. Une des solutions à l'exploitation des données biomédicales réside dans l'utilisation de l'intelligence artificielle, et plus spécifiquement de la branche nommée \gls{ml} pour construire des systèmes capables d'exploiter ces données.

\section{Machine-Learning pour le traitement des données biomédicales}
Le \textit{machine-learning} est une branche de l'\gls{ia} qui regroupe un ensemble d'algorithmes capable d'accomplir une tâche en apprenant d'un jeu de données. Dans cette section nous allons définir les concepts des base du \gls{ml} tel que le format des données, les taches qui peuvent être accomplies, les méthodes d'apprentissages et les principaux algorithmes utilisés.

\subsection{Les formats et partitionnement des données}
Les données sont le point fondamental et critique des techniques de \gls{ml}. Les données représentent l'ensemble des informations brutes utilisée par notre algorithme de \gls{ml} pour réaliser son apprentissage et réaliser des prédictions. Pour être utilisables par les algorithmes de \gls{ml} les données doivent être structurées. Le tableau \ref{table:dataset_intro} présente un exemple de structure d'un jeu de donénes exploitable pour un algorihtme de \gls{ml}. Les données sont sous la forme de tableau où chaque ligne représente une observation (un point de données, par exemple un patient) et chaque colonne représente un descripteur (nommé \textit{feature} en anglais, par exemple le rythme cardiaque, la présence d'une toux chez le patient, la présence d'antécédant de diabète...). Enfin la dernière colonne représente le label, c'est en général ce que l'on souhaite prédire dans le cadre de l'entraînement de notre modèle.
\begin{table}[h!]
\centering
\begin{tabular}{|c|c|c|c|c|} 
 \hline
 ID Patient & Rythme Cardiaque (bpm) & Toux & Diabiète & Diagnostic \\
 \hline
 1 & 86 & non & non & Sain \\ 
 2 & 65 & non & non & Sain \\ 
 3 & 59 & non & non & Sain \\ 
 4 & 95 & oui & non & Malade \\ 
 5 & 101 & oui & oui & Malade\\ 
 \hline
\end{tabular}
\caption{Exemple de tableau de données fictives de patients}
\label{table:dataset_intro}
\end{table}
Cette contrainte sur la structure nécessaire du jeu de données pour les algorithmes de \gls{ml} mets en évidence les limites de leur utilisations pour l'analyse de données non-structurées telles que le texte-libre, les données d'imageries ou les données de séquence d'ADN. Il est nécessaire en amont de structurer ces données à travers des descripteurs pertinents pour les exploiter.

De plus, il est nécessaire de partitionner ce jeu de données sous forme de tableau en deux: les données d'apprentissage et les données de test. Les données d'apprentissage sont les données qui vont être utilisée par l'algorithme de \gls{ml} pour réaliser son entrainnement, c'est-à-dire pour apprendre à réaliser la tâche définie (prédiction du label par exemple). Le jeu de test quant à lui contient des données qui n'ont jamais été présentée au modèle au cours de l'apprentissage. Le modèle entrainé va alors prédire le label du jeu de test et les prédictions réalisée sont comparées aux labels réels. Cela permet d'évaluer les performances de notre entrainnement. Pour donner un ordre de grandeur, il est commun d'utiliser 80\% des données comme jeu d'entrainnement et 20\% des données restantes comme jeu de test.

Pour finir il existe un trosième partiionnement des données optionnel nommé jeu de validation. Le jeu de valiation est réalisée en général en prenant 10\% des données d'entrainnement, ce jeu de validation permet d'avaluer le modèle au cours de l'entrainneemnt et à ajuster ses paramtères. Ceci permet de s'assurer que l'entrainnement progresse correctement avant de tester les performances à la fin de l'entrainnement sur le jeu de test.

\subsection{Les différentes tâches que le machine-learning peut accomplir}
Les algorithmes de \gls{ml} peuvent accomplir de multiples taches dont quatres principales. Il y a: (i) les algorithmes de classlifications, (ii) de régression, (iii) de clustering et (iv) de réduction de dimentionalité.

Les tâches de classification sont les plus communes. Il s'agit ici d'apprendre à prédire une classe ou un label pour un point de donnée. Par exemple il peut s'agir dans le cadre de données biomédicale de la prédiction d'un diagnostic parmis une liste de maladies. Cette classification peut-être binaire (2 classes uniquement, par exemple sain vs malade) ou multiclasse (plus de 2 classes, par exemple faire la différence entre 10 diagnostic possibles). Enfin cette classification peut aussi être multilabel, c'est à dire que l'on peut prédire plusieurs classes pour un point de donnée. Par exemple, s'il on construit un algorithme capable de prédire le genre d'un film de cinéma, il est utile d'avoir un système de classification multilabel pour prédire plusieurs genre (comédie et horreur par exemple) pour un même film. Parmis les algorithmes de \gls{ml} capable de faire de la classification on retrouve de nombreux outils tels que les méthodes bayésiennes, les méthodes à bases d'arbres (arbre de décision, forêt aléatoire), les régressions logistiques où encore les systèmes de classeurs.

Les tâches de régressions ne cherchent pas à prédire une catégorie mais une valeur numérique. Par exemple, on peut construire un modèle \gls{ml} capable de prédire le prix d'une maison ou encore la pression sanguine d'un patient, dans ces cas là on cherche à prédire une valeur numérique continue. Les algorithmes à base d'arbres sont aussi capables de réaliser des tâches de régressions (arbre de décisions, forêt aléatoire) et on retrouve aussi d'autre algorithmes tels que la régression Lasso, Ridge et régression linéaire, qui est l'algorithme de base pour les taches de régression.

Les tâches de clustering cherchent à regrouper les points de données similaire en sous-groupes, c'est-à-dire en cluster. Les techniques de clustering sont utilisées dans le domaine biomédicale pour analyser les données d'expression génétique. A partir de l'expression des gènes d'une cohorte de patients, il est possible d'utiliser des algorithmes de clustering pour trouver des sous-groupes de patients ayant un profil génétique similaire par exemple dans le cadre d'une comparaison de l'expression des gènes chez des patients sains et des patients atteint de cancer. Les algorithmes classiques de clusteing sont l'algorithme K-means, DBSCAN et le custering hiérarchique.

Pour finir, les tâches de réduction de dimensionalité consistent à réduire le nombre de variable aléatoire d'un jeu de données en obtenant un ensemble variables principales. Typiquement, les données à haute dimensionalité comme les données transcriptomiques (expression de plusieurs dizaine de miliers d'ARN) sont complexes à analyser et présentent des problèmes spécifiques à cette haute dimensionalité, connus sous le nom de la malédiction de la dimension (\textit{curse of dimensionality}). Les techniques de réduction de dimensionalité tendent à atténuer ce problême. Les algorithmes de réduction de dimensionalités sont typiquement utilisés après une étapes de clustering pour observer graphiquement les clusters obtenu en un graphique 2D. Pour reprendre l'exemple précédant, après une analyse transcriptomiques une étape de réduction de dimensionalité peut-être appliquée pour visualiser quel est le principale axe de différenciations de nos échantillons. Les algorithmes de réductions de dimensionalité communément utilisés sont la PCA, le t-SNE et UMAP.

\subsection{Apprentissage supervisé, non-supervisé et par renforcement}
Les différentes tâches présentée peuvent se regrouper sous trois méthodes d'apprentissages différentes: l'apprentissage supervisé, non-supervisé et par renforcement.

Les tâches de classification et de régression sont possibles grâce à l'apprentissage supervisé. En apprentissage supervisé le modèle est entraîné sur des données labellisée, c'est àdire des données pour lesquels on connaît déjà le résultat attendu (diagnostic par exemple). Ainsi le modèle est entrainé à reproduire ce labels automatiquement.
Les taches clustering et de réduction de dimensionalités sont possibles grace à l'aprentissage non-supevisé. En apprentissage non-supervisé, les labels des données ne sont pas connus. L'obejctifs est donc de découvrir la structurée cahcées des données à partir des descripteurs. Ainsi le modèleessaie de déterminer des groupes ou des regroupement de dimensions qu'il détermine comme pertintent mais sans connaitre le résultat réel attendu.

Enfin l'apprentissage par renforcement est moins connus et représnete une méthode d'apprentissage où un agent (modèle) apprend à se comporter dans un environnement donné, recevant des pénalité et des récompenses en fonction de ses actions. Typiquement,  un modèle apprenant à jouer à un jeu d'échecs représente une tache d'apprentissage par renforcement.

La figure \ref{fig:ml-landscape} représente schématiquement la classification des tâches, des modes d'apprentissages et des différents cas d'applications et algorithmes associés.
\begin{figure}[!htbp]
 \centering
 \includegraphics[width=1\textwidth]{figures/ml_landscape.png}
 \caption[Schéma des méthodes de machine-learning]{Schéma de la classification des tâches, des modes d'apprentissage et des différents cas d'applications et algorithmes associés en \textit{machine-learning}}
 \label{fig:ml-landscape}
\end{figure}

\subsection{Différents algorithmes et explicabilité}
\subsubsection{Le concept d'explicabilité}
ici on introduit quelques méthodes mais pas exhaustif
\subsubsection{Méthodes à bayesiennes}
\subsubsection{Méthodes à base d'arbres}
citer MISTIC
\subsubsection{Systèmes de classeurs}

réseaux de neuronnes après

\subsection{Limites du ML aux données biomédicales}
traitement manuel des données 
Data Integration Challenges for Machine Learning in Precision Medicine
\chapter{Nouvelle génération d’IA pour le traitement du \textit{Big Data} grâce aux réseaux de neurones}
\section{Les réseaux de neuronnes}
\section{L'analyse d'imagerie par  réseau neuronnal convolutifs}
\subsection{La convolution pour l'analyse d'images}
\subsection{Modèle grand public pour l'histologie}
\subsubsection{CellPose}
\subsubsection{Stardist}
\subsubsection{En biologie mais hors imageries: Spliceator}
\section{La révolutions des modèles linguistiques de grande taille}
https://huyenchip.com/2023/04/11/llm-engineering.html application des LLMs
\subsection{Modèle linguistiques de grande taille: la ruée vers l'or et modèles d'attentions}
\subsubsection{Exemple de transformers pour le triatement de séquences: AlphaFold}
\subsubsection{Exemple de transformers pour le triatement de séquences: Expression Enchancer}k
\subsection{L'apprentissage auto-supervisé}
\subsection{Modèle génératifs, models embedding}
\subsection{Taille des modèles, hébergement, quantization}
\subsection{Exemple d'application en science}

Zero-shot, One-shot, Few-shot learning
\chapter{L’exemple des myopathies congénitales et la difficulté du diagnostic}

Placeholder

\part{MATÉRIELS ET MÉTHODES}
\section{Données biomédicales de myopathies congénitales}
\subsection{Rapport histologiques de patient de l'institut de myologie de Paris}
\subsection{Images de biopsie musculaire de souris}
\section{Ontologies biologiques}
\subsection{Ontologies des phénotypes: HPO}
\subsection{Ontologie de maladies: Orphanet}
\subsection{Nomenclature génétique: HUGO et HGVS}
\section{Reproductibilité des analyse par intelligence artificielle}
\subsection{Suivi d'expérience avec \textit{Weight and Biais}}
\subsection{Développement open-source et versionnage avec GitHub}
\subsection{Environnement de développement reproductible}
\subsubsection{Environnement virtuel Python}
\subsubsection{Docker}
\section{Développement de modèle de ML traditionnels et xAI}
\subsection{Scikit-Learn}
\subsection{Cross-validation et évaluation des performances}
\subsection{Rechercher d'hyper-paramètres}
\section{Techniques d'analyse d'images et réseaux de neurones}
\subsection{Méthode d'analyse d'image traditionelles avec scikit-image}
\subsection{Developement de réseau de neuronnes profond de type ResNet avec Tensorflow}
\section{Développement d'outils basés sur modèles linguistiques de grande taille}
\subsection{Interaction avec les modèles linguistique de grande taille avec LangChain}
\subsubsection{Modèle génératif Vicuna-7B}
\subsubsection{Modèle d'\textit{embedding} Instructor}
\subsection{Base de données de vecteurs avec ChromaDB}
\section{Developpment Web}
\subsection{Back-end avec Flask}
\subsection{Front-end avec Jquery et Boostrap}
\subsection{Base de données avec SQLite}
\subsection{Solution clé en main avec Streamlit}


\part{CONTRIBUTIONS}
\chapter{IMPatienT : un outil d’annotation et d’exploration de données multimodales de patients}

\section{Contexte}
\section{Manuscrit}
\section{Limitations}
\section{Perspectives futures de développement}
\chapter{IMPatienT : annotation et d’exploration de données multimodales de patients}\label{chap_imp}

Dans un premier temps, nous avons développé \gls{impatient} (fig. \ref{fig:impatient_logo}) une application web qui permet l'annotation semi-automatique de compte rendu et d'images de biopsies musculaires. Cette application web, développée en 2020-2021, c'est à dire avant la mise à disposition des \gls{llms}, permet de créer un jeu de données structuré (tableau de patients) à partir de données non structurées (texte libre et images). Ainsi dans ce contexte-là, pour exploiter des données sous la forme de texte libre, il est nécessaire de procéder à son annotation manuelle afin de les structurer. L'objectif d'\gls{impatient} est de mettre à disposition une interface qui permet la numérisation de ces données non structurée, de les préparer pour l'application des méthodes de \gls{ml} traditionnelles et de fournir des outils d'explorations automatiques.


Pour structurer ces données biomédicales, nous utilisons une approche basée sur le concept d'ontologie: un vocabulaire standardisé pour décrire de manière unifiée les observations réalisées. S'il existe déjà des ontologies pour nommer les maladies (\gls{ordo}), les observations cliniques (\gls{hpo}), et des nomemclatures pour les gènes et mutations (HUGO, \gls{hgvs}), il n'existe aucune ontologie à ce jour permettant de décrire les observations histopathologiques dans les biopsies musculaires. Pour cela, \gls{impatient} intègre un module permettant en premier lieu la création de vocabulaire standard, que nous avons utilisé pour créer un vocabulaire standard des observations histopathologiques. Ensuite à partir de ce vocabulaire standard nous avons développé un module pour numériser et détecter de manière semi-automatique les termes du vocabulaire standard dans les comptes rendus de biopsies. Pour ajouter à l'aspect multimodal, nous avons développé un module de segmentation d'images assisté par intelligence artificielle. Ce module permet de rapidement annoter des images de biopsies avec les termes du vocabulaire standard afin de créer un jeu données annoté qui permet l'entrainement d'IA de segmentation d'images automatique. Enfin, un dernier module de visualisation des données enregistrées permet d'explorer en temps réel les données de patient numérisées dans l'application web.

\begin{figure}[H]
  \centering
  \includegraphics[width=0.7\textwidth]{figures/impatient_banner.png}
  \caption[Logo IMPatienT]{\textbf{Logo d’IMPatienT}}
  \label{fig:impatient_logo}
\end{figure}


\section{Manuscrit} 
Le manuscrit d'\gls{impatient} a été soumis à la revue scientifique \textit{"Journal of Neuromuscular Diseases"} et est présenté ci-dessous.

\includepdf[pages={1-27},scale=1,dpi=300]{corps/IMPatienT_pdf.pdf}


\section{Données sensibles et déploiement de la plateforme}
Afin de traiter les données de patients atteints de \gls{mc} (qui doivent rester privées), mais aussi de pouvoir faire une démonstration technique de l'outil, nous avons mis en ligne deux instances de la plateforme. Une première instance publique, à l'adresse \url{https://impatient.lbgi.fr}, contient une quarantaine de rapports de comptes rendus fictifs générés aléatoirement. Cette instance est accessible à tout le monde et est remise à zéro chaque jour. Une seconde instance privée, à l'adresse \url{https://myoxia.lbgi.fr}, contient 89 rapports de comptes rendus de patients provenant de l'Institut de Myologie de Paris qui ont été numérisés. Cette instance n'est accessible que par mot de passe et son contenu est sauvegardé de manière régulière. Le code source d'\gls{impatient} est \textit{open-source} et disponible dans un répertoire GitHub à l'adresse: \url{https://github.com/lambda-science/IMPatienT}.


\section{Limitations et perspectives de développement}
\gls{impatient} est une application web permettant d'annoter et d'explorer les données biomédicales issues de la biopsie musculaire de patients atteints de \gls{mc}. Cette application web a été développée pour pallier au manque d'outils permettant d'annoter et d'extraire de l'information de comptes rendus médicaux en texte libre grâce à une approche basée sur les ontologies. En plus d'intégrer les ontologies médicales déjà existantes (\gls{ordo}, \gls{hpo}, HUGO, \gls{hgvs}), \gls{impatient} permet de créer facilement un vocabulaire standard similaire à une ontologie pour les domaines où il n'existe pas encore d'ontologie de référence. Dans notre cas, nous l'avons utilisé pour créer notre propre vocabulaire standard des observations histopathologiques réalisées dans les biopsies musculaires. Nous avons passé en revue l'ensemble des 89 comptes rendus de biopsies musculaires et nous avons extrait et hiérarchisé par colorations les termes uniques trouvés dans ces rapports. Ce travail réprésente un total de 175 termes extraits composant notre vocabulaire standard.


Cependant, l'approche développée conçue pour l'annotation est semi-automatique et requiert donc toujours un travail manuel et humain de correction et de validation des annotations. Cette limitation empêche donc le passage à l'échelle et le traitement d'une masse importante de comptes rendus textuels ou d'images. De plus, l'approche utilisée est dépendante de la définition d'un vocabulaire standard exhaustif. Si un terme nouveau est présent dans un compte rendu et est absent du vocabulaire au moment de l'annotation, il faut le rajouter au préalable. De même, si on opère des modifications importantes du vocabulaire standard, il est peut-être nécessaire de devoir passer en revue l'ensemble des comptes rendus déjà numérisés pour vérifier si de nouveaux termes sont à associer aux comptes rendus ou si d'anciennes annotations sont à supprimer.


En termes de développement futur, il est nécessaire d'intégrer à \gls{impatient} de nouveaux outils permettant l'automatisation de l'analyse des comptes rendus, par exemple un outil permettant de préremplir les formulaires avec les informations générales détectées. Ainsi, grâce aux récentes avancées dans le traitement de texte libre, notamment grâce aux\gls{llms}, nous avons développé de nouvelle méthode qui permettent d'automatiser ce processus d'annotation. Ces nouvelles méthodes sont décrites dans le chapitre 7: "NLMyo : Traitement de rapports textuels par LLMs". Il serait intéressant aussi de proposer des alternatives plus automatiques au système d'annotation avec le vocabulaire standard.


De plus au niveau de l'analyse d'images, il est intéressant de proposer un outil capable de quantifier grâce à l'\gls{ia} des marqueurs pathologiques dans les images de biopsie, car cette information est manquante dans les comptes rendus de biopsie, les observations ne sont que qualitatives et non quantitatives. Pour cela, nous avons développé un outil présenté dans le chapitre 8 "Vers une génération de rapports automatiques à partir d’imagerie avec MyoQuant".


Grâce à l'intégration de ces outils, l'application web\gls{impatient} deviendrait le socle de l'intégration de plusieurs outils d'\gls{ia} pour l'analyse de données multimodales. \gls{impatient} serait alors le point d'entrée mettant à disposition des outils pour créer une base de données multimodales de patients et fournir les outils adaptés à leur analyse et exploration.
\chapter*{Chapitre 6: NLMyo : Traitement de rapport textuel par modèles linguistiques de grande taille}
% ...
Placeholder
\chapter{NLMyo : Traitement de rapports textuels par LLMs}

\section{Contexte}
Les récentes avancées dans le \gls{nlp} ont permis de révolutionner la manière de traiter et d'exploiter les données sous forme de texte libre. \gls{impatient} utilise un système à base de règles et de correspondance exacte des mots à des ontologies existantes pour analyser les comptes-rendus de biopsie. Cette approche présente des limites importantes en terme de flexibilité et de précision et demande l'établissement d'un vocabulaire normalisé au préalable. 

Récemment, la mise à disposition de \gls{llms} performants et accessibles ouvre la porte à la création d'outils plus performants et flexibles pour le traitement de ces comptes rendus. Ces systèmes basés sur une approche sémantique et multi-langues éliminent la nécessité de définir un vocabulaire standard. Ainsi nous avons développé \gls{nlmyo} (fig \ref{fig:nlmyo_logo}), une boite à outils basée sur les \gls{llms} mettant à disposition quatre outils généralistes pour le traitement de comptes rendus médicaux: outil d'anonymisation, d'extraction d'information, de classification automatique et de création de moteur de recherche. L'ensemble de ces outils et des modèles utilisés est représenté dans la figure \ref{fig:nlmyo_struct}.
\begin{figure}[!ht]
 \centering
 \includegraphics[width=0.5\textwidth]{figures/nlmyo_banner.png}
 \caption[Logo NLMyo]{Logo de NLMyo}
 \label{fig:nlmyo_logo}
\end{figure}
\begin{figure}[!ht]
 \centering
 \includegraphics[width=1\textwidth]{figures/nlmyo_struct.png}
 \caption[Structure de NLMyo]{Structure de NLMyo}
 \label{fig:nlmyo_struct}
\end{figure}
\section{\textit{Anonymizer}: un outil d'anonymisation}
Le premier outil de \gls{nlmyo} est \textit{Anonymizer}, un outil permettant de supprimer automatiquement les informations identifiantes des comptes rendus médicaux. Dans les comptes rendus de biopsie de l'Institut de Myologie de Paris que nous traitons, deux données identifiantes et personnelles sont présentes et doivent être retirées: le nom du patient (et du personnel médical) ainsi que la date de naissance du patient. Non seulement ces informations ne sont pas utiles pour les analyses subséquentes mais de plus, par respect de la vie privée et des recommandations \gls{rgpd}, ces informations ne doivent être accessibles qu'aux professionnels médicaux en charge du patient. L'anonymisation des rapports est donc une étape essentielle avant le transfert des données, leur numérisation et leur analyse.

Afin de traiter un grand volume de rapports et d'éliminer le travail manuel nécessaire, nous avons tenter d'automatiser la tache d'anonymisation \textit{via} deux approches: une approche traditionnelle par\gls{regex} et une approche novatrice par\gls{llms}.

\subsection{Anonymisation par RegEx}
En première intention, nous avons développé une méthode basée sur les \gls{regex} pour leur simplicité de mise en place et leur rapidité d'exécution (coûts en puissance de calcul faible). Le tableau \ref{tab:regex} liste les \gls{regex} utilisées pour capturer les informations de noms et de dates. Les comptes rendus de biopsie sont semi-structurés. Pour la plupart, le nom du patient est facilement identifiable car il est précédé par le préfixe: "Nom: ". Ceci est facilement représenté par la première \gls{regex} listée dans le tableau. Ensuite pour les autres cas de figure, comme les noms de famille sont souvent en majuscule et les prénoms commencent souvent par une majuscule, nous avons développé deux autres \gls{regex} pour capturer les couples de mots dont un est en majuscule et le second commence par une majuscule (ou inversement, ligne 2 et 3 du tableau). 
Ensuite, une troisième \gls{regex} a été ajoutée pour détecter les dates au format JJ-MM-AAAA ou JJ.MM.AAAA (ligne 3 du tableau). Finalement une dernière \gls{regex} est utilisée pour essayer de trouver le numéro de biopsie dans le document afin de renommer le fichier avec un nom unique et anonyme (ligne 4 du tableau). 
\begin{table}[!ht]
\centering
\caption{Expressions régulières pour extraire les noms et les dates}
\label{tab:regex}
\begin{tabular}{|l|l|l|}
\hline
\textbf{Expression régulière} & \textbf{Syntaxe} & \textbf{Exemples d'utilisation} \\ \hline
Nom patient & Nom.*: *([A-Za-zÀ-ÿ- ]+) & Nom : John Doe \\ \hline
Nom patient 2& (([A-Z][a-zÀ-ÿ-]\{3,\} ?)+ ([A-Z-]\{3,\} ?)+) & DOE John \\ \hline
Nom patient 3 & (([A-Z-]\{3,\} ?)+ ([A-Z][a-zÀ-ÿ-]\{3,\} ?)+) & John Joe DOE \\ \hline
Date & ([(.]?[0-9]\{1,2\}[./][0-9]\{1,2\}[./][0-9]\{1,4\}[().]?) & 01/01/2023, 02.02.2023 \\ \hline
N° Biopsie & ([0-9]\{3,8\}[-/]?[0-9]\{0,3\}) & 9876-54, 1234/56 \\ \hline
\end{tabular}
\end{table}
\begin{figure}[!ht]
 \centering
 \includegraphics[width=1\textwidth]{figures/regex.png}
 \caption[Exemple anonymisation RegEx]{Exemple d'anonymisation d'un rapport de biopsie factice avec la méthode RegEx. En haut le rapport à l'état brut, en bas le rapport censuré.}
 \label{fig:regex}
\end{figure}

La figure \ref{fig:regex} présente les résultats de la technique d'anonymisation par \gls{regex} sur l'entête d'un rapport factice de patient mais avec une structure similaire aux rapports de l'Institut de Myologie de Paris. Les noms et les dates ont été censurés correctement et la méthode a produit de bon résultats. Cependant, cette méthode est insuffisante car elle est peu sensible et spécifique. Le tableau \ref{tab:regex_fail} liste trois exemples de cas où cette méthode ne fonctionne pas et produit des erreurs. Il est possible de corriger ces erreurs en augmentant la somme de \gls{regex} utilisées, mais augmenter le nombre de \gls{regex} augmente aussi potentiellement le nombre de faux positifs. De plus, il n'est pas toujours possible de construire une \gls{regex} adaptée pour extraire une information précise. Nous avons alors exploré la capacité des \gls{llms} pour la recherche et l'extraction de ces informations de manière plus robuste et flexible que la méthode \gls{regex}.
\begin{table}[!ht]
\centering
\caption{Exemples de faux positifs ou faux négatifs de la méthode RegEx}
\label{tab:regex_fail}
\begin{tabularx}{\textwidth}{|l|l|l|X|}
\hline
\textbf{Texte} & \textbf{RegEx déclenchée} & \textbf{Type} & \textbf{Commentaire} \\ \hline
\textit{PAS Staining} & Nom patient 3 & Faux Positif & Nom de coloration dont la notation est confondue avec le motif "NOM Prénom" \\ \hline
\textit{Louis C. Dupont} & N/A & Faux Négatif & La présence du "C." au centre ne permet pas aux RegEx de nom de se déclencher \\ \hline
\textit{12 mars 2001} & N/A & Faux Négatif & La notation de date avec un mois en lettres ne permet pas à la RegEx de date de se déclencher \\ \hline
\textit{1996-04} & N° Biopsie & Faux Positif & La notation de date AAAA-MM déclenche la \gls{regex} de numéro de biopsie. \\ \hline
\end{tabularx}
\end{table}

\subsection{Anonymisation par LLMs}
Les \gls{llms} sont basés sur la compréhension du sens sémantique du texte là où les \gls{regex} sont basées sur le principe de motif de caractères et donc sur la structure du document. L'idée de l'anonymisation par \gls{llms} est d'utiliser un modèle génératif auquel on fournit une instruction et le texte à anonymiser. L'instruction liste les informations à extraire du texte et spécifie le format de sortie. Pour intégrer ces modèles génératifs à notre programme, nous voulons récupérer les informations extraites dans un format exploitable informatiquement, par exemple au format JSON. Le format JSON représente un dictionnaire qui est utilisable ensuite par l'application pour censurer les PDF à partir des informations extraites.

\subsection{Instruction personnalisée et \textit{one-shot learning}}
Nous avons construit une instruction personnalisée en 3 parties qui intègre une méthode de \textit{one-shot learning}. Les trois parties sont: (i) la description de la tâche à effectuer, (ii) un exemple de réalisation de la tâche \textit{(one-shot learning)}, (iii) le texte d'intérêt à anonymiser.
Voici un exemple d'instruction que nous utilisons pour réaliser l'extraction des noms et des dates dans les rapports:
\begin{quote}
Tu es un assistant qui extrait des informations d'un texte libre. Le format de ta réponse doit être un format JSON valide qui respecte le nom des clés founies. Si une valeur est manquante, indique simplement N/A, n'essaie pas d'inventer. Voici la liste des informations à récupérer, les clés JSON sont indiquées entre parenthèses : nom complet (name), dates (date).

ENTRÉE :

Kendrick Lamar et Jane Clinton sont asymptomatiques. Date de naissance : 16 février 1991, numéro de biopsie : 666-77. Ce rapport a été expédié le 01.04.1991.

SORTIE :

\{"name" :["Kendrick Lamar", "Jane Clinton"], "date" : ["16 février 1991", "01.04.1991"]\}

ENTRÉE :

<texte à analyser>

SORTIE :
\end{quote}

La partie précédant le mot clé "ENTREE" correspond à l'instruction décrivant précisément la tâche que le modèle doit réaliser (liste des informations à extraire, format de sortie et comportement attendu). Le premier couple "ENTREE" et "SORTIE" correspond à un exemple de réalisation de la tâche, ce qui permet de spécifier le schéma JSON attendu au modèle (\textit{one-shot learning}). Puis le second couple "ENTREE", "SORTIE" correspond à l'endroit où l'on injecte notre texte d'intérêt à analyser et spécifie au modèle que l'on attend maintenant une sortie textuel au format JSON pour l'entrée précédente.

\subsection{Exemple et comparaison à la méthode RegEx}
\begin{figure}[!ht]
 \centering
 \includegraphics[width=0.8\textwidth]{figures/llms_anonym.png}
 \caption[Exemple anonymisation LLMs]{Exemple d'anonymisation d'un courrier médical factice avec la méthode par LLMs. (A) Texte brut, (B) Anonymisation par RegEx, (C) JSON brut généré par le \gls{llms} GPT-3.5-turbo (OpenAI), (D) Anonymisation à partir du JSON généré par \gls{llms}.}
 \label{fig:llms_anonym}
\end{figure}

Dans cet exemple (figure \ref{fig:llms_anonym}), nous avons construit un début de courrier médical factice faisant figurer des informations personnelles qui se sont pas détectables par la méthode \gls{regex}. Les prénoms "Jérémy" et "Clara", ainsi que les dates "2 mai 2023" et "08 12 2003" n'ont pas été détectés par la méthode \gls{regex} et n'ont pas été censurés. On observe par contre que la méthode par \gls{llms} (C et D) a été capable d'identifier ces informations et de les extraire du texte. Cet exemple montre que les \gls{llms} peuvent être la base d'un système d'anonymisation plus flexible et moins dépendant de la structure du document.

\section{\textit{MyoExtract}: un outil d'extraction d'information}
A partir des résultats encourageants obtenu avec la technique d'anonymisation par LLMs pour l'extraction d'information, nous avons voulu étendre le champ des informations extraites de manière automatique à partir de texte libre. Nous avons utilisé la même stratégie d'extraction d'information, c'est à dire l'utilisation de \gls{llms} génératifs mais avec une instruction légèrement différente. Cette fois-ci nous avons ajouté une liste plus importante d'informations à extraire non pas dans le but d'anonymiser le document, mais d'en extraire les méta-données. Par exemple nous avons chercher à extraire: les noms, date de naissance, date d'envoi de la biopsie, numéro de biopsie, muscle prélevé, diagnostic final. De plus, nous avons cherché à savoir s'il était possible d'extraire les mentions d'anomalie pour certaines colorations telles que la coloration PAS, Soudan, COX, ATP et Phosphorilase. Cette extraction d'information pourrait permettre d'annoter automatiquement les rapports avec une liste d'anomalies détectées pour chaque coloration. De même que précédemment, nous avons construit une instruction personnalisée en 3 parties (description, exemple, texte à analyser).

Voici un exemple d'instruction que nous utilisons pour réaliser l'extraction des méta-données et d'anomalies générales des colorations à partir de rapports:
\begin{quote}
Tu es un assistant qui extrait des informations d'un texte libre. Le format de ta réponse doit être un format JSON valide qui respecte le nom des clés founies. Si une valeur est manquante, indique simplement N/A, n'essaie pas d'inventer. Formate les dates sous la forme DD-MM-YYYY et convertis les âges en années (0 si inférieur à 1 an). Voici la liste des informations à récupérer, les clés JSON sont indiquées entre parenthèses : nom complet (name), âge (age), date de naissance (birth), date de la biopsie (biodate), date d'envoi de la biopsie (sending), muscle (muscle), numéro de la biopsie (bionumber), diagnostic (diag), présence d'une anomalie dans la coloration du PAS (PAS), présence d'une anomalie dans la coloration Soudan (Soudan), présence d'une anomalie dans la coloration COX (COX), présence d'une anomalie dans la coloration ATP (ATP), présence d'une anomalie dans la coloration Phosphorylase (phospho)

ENTRÉE:

Kendrick Lamar et Jane Clinton ne sont pas asymptomatiques. Date de naissance: 16 février 1991, numéro de biopsie: 666-77. Anomalie forte à la coloration PAS mais pas d'anomalie à la coloration lipide soudan. Le tableau est révélateur d'une myopathie à némaline

SORTIE:

\{"name":["Kendrick Lamar", "Jane Clinton"], "age":"N/A", "birth": "16-02-1991", "biodate": "N/A"", "sending": "N/A"", "muscle": "N/A"", "bionumber": "666-77", "diag": "myopathie à némaline", "PAS": "yes", "Soudan": "no", "COX": "N/A", "ATP": "N/A", "phospho": "N/A"\}

ENTRÉE :

<texte à analyser>

SORTIE :
\end{quote}

\subsection{Exemple d'extraction d'information}
Pour cet exemple d'utilisation, nous avons généré un rapport factice de patient avec une structure similaire aux rapports de l'Institut de Myologie de Paris qui reprend des observations typiques trouvées dans les rapports réels de biopsie. Ce rapport est disponible en figure \ref{fig:factice_report}. 
\begin{figure}[!ht]
 \centering
 \includegraphics[width=1\textwidth]{figures/pdf_biopsie.png}
 \caption[Rapport de biopsie factice]{Exemple rapport de biopsie factice}
 \label{fig:factice_report}
\end{figure}

Les résultats de l'extraction d'information présentés dans le tableau \ref{tab:json_data} montrent que le modèle d'\textit{OpenAI} GPT-3.5-turbo est capable d'extraire l'ensemble des informations demandées de manière satisfaisante tout en étant capable de détecter l'absence de certaines informations. Concernant le modèle \textit{Vicuna-7B}, qui a l'avantage d'être auto-hébergé et donc d'être utilisable pour des données sensibles, les performances sont moindres. En effet, six données sur treize (diagnostic et anomalies) demandées sont tout simplement manquantes dans le JSON de sortie. Cependant, pour les sept données (nom, âge, dates, muscle et numéro de biopsie) présentes, les résultats sont satisfaisants, le modèle a extrait les bonnes informations.

Il est important de noter qu'en terme de ressources de calcul et de temps d'inférence, \textit{GPT-3.5} possède un avantage non-négligeable car ce modèle n'est accessible que par une \gls{api}. Les coûts de calcul pour l'application sont donc nuls et la requête ne prend que quelques secondes à être réalisée. Pour \textit{Vicuna}, le modèle étant auto-hébergé, chaque requête requiert une quantité importante de ressources de calcul et monopolise ces ressources pour un temps important (environ 1min30 par document). Ces coûts en ressources et en temps de calcul couplés à une précision moindre, rendent difficile l'exploitation de modèle \gls{llms} génératif auto-hébergé pour la tâche d'extraction d'information à travers une interface en ligne.
\begin{table}[!ht]
\centering
\caption{Résultats de \textit{MyoExtract} pour \textit{GPT-3.5-turbo} and \textit{Vicuna7B}}
\label{tab:json_data}
\begin{tabularx}{\textwidth}{|l|X|X|}
\hline
\textbf{Information} & \textbf{GPT-3.5-turbo} & \textbf{Vicuna7B} \\ \hline
Nom & DOE John & DOE John \\ \hline
Âge & 7 & 7 years old, born on 20/11/2015 \\ \hline
Date de naissance & 20-11-2015 & 20-11-2015 \\ \hline
Date de biopsie & 20-11-2021 & 20-11-2021 \\ \hline
Date d'envoi & 21-11-2021 & 21-11-2021 \\ \hline
Muscle & Quadriceps & quadriceps \\ \hline
N° Biopsie & 777-07 & 777-07 \\ \hline
Diagnostic & congenital myopathy with a nemaline subtype with strong fiber type disproportion and smaller fiber type & \textit{missing} \\ \hline
Anomalie PAS & N/A & \textit{missing} \\ \hline
Anomalie Soudan & N/A & \textit{missing} \\ \hline
Anomalie COX & N/A & \textit{missing} \\ \hline
Anomalie ATP & abnormal fiber differentiation and no fiber bundling & \textit{missing} \\ \hline
Anomalie Phospho. & N/A & \textit{missing} \\ \hline
\end{tabularx}
\end{table}

\section{\textit{MyoClassify}: un outil d'aide au diagnostic}
L'outil \textit{MyoClassify} a pour objectif de suggérer un diagnostic parmi les 3 types majoritaires de \gls{mc} (\gls{nm}, \gls{com}, \gls{cnm}) de manière automatique sur la base du rapport de biopsie brut. Pour cela, nous avons utilisé comme jeu de données un corpus élargi de 192 rapports de biopsies fournis par l'institut de myologie de Paris labellisés selon 5 classes (tableau \ref{tab:number_patients}: \gls{nm}, \gls{com}, \gls{cnm}, diagnostic différent des 3 sous-types majoritaires (\textit{non-CM}) et pas de diagnostic final établi (\textit{UNCLEAR}).
\begin{figure}[!ht]
 \centering
 \includegraphics[width=1\textwidth]{figures/myoclassify_flow.png}
 \caption[Entraînement modèle \textit{MyoClassify}]{Étapes de préparation et d'entraînement des modèles de \textit{MyoClassify}}
 \label{fig:myoclassify_flow}
\end{figure}
\subsection{Méthodologie}
La figure \ref{fig:myoclassify_flow} représente l'ensemble des étapes réalisées pour préparer les données et entraîner un modèle de classification. Pour l'ensemble de ces rapports, nous avons réalisé une étape de détection de texte par \gls{ocr} avec \textit{Tesseract}, puis nous avons retiré les conclusions des rapports (indiquant la décision de diagnostic final, c'est-à-dire le label). A partir de ces conclusions nous avons labellisé à la main chaque rapport avec un diagnostic parmi les 5 catégories listée ci-dessus.

La contenu de chaque rapport (texte brut sans la partie conclusion) a été traduit en anglais grâce à l'\gls{api} DeepL afin de comparer les performances sur les textes anglais (traduit) et français (orignaux). Puis ces textes ont été encodés numériquement grâce à deux modèles\gls{llms} d'\textit{embedding}: le modèle d'OpenAI (disponible uniquement via \gls{api}) et le modèle \textit{Instructor-Large} (auto-hébergé). Les modèles d'\textit{embedding} sont des modèles qui prennent en entrée un texte (un mot, une phrase, un paragraphe ou un document) et qui produisent en sortie un vecteur numérique de grande taille capturant le sens sémantique du document d'entrée. Par exemple pour le modèle \textit{Instructor-Large}, le vecteur numérique de sortie pour un document possède une dimension de (1, 768) et (1, 1536) pour le modèle d'OpenAI. Ces modèles sont des boites noires, c'est à dire que la signification des centaines (voire des milliers dans le cas d'OpenAI) de valeurs numériques décrivant le document ne sont pas connues, cependant elles représentent le sens sémantique du texte.
A partir de ces 4 jeux de données (192 rapports dans 4 conditions: français/anglais et \textit{embedding} par OpenAI/Instructor), nous avons entraîné et comparé les performances pour la prédiction de diagnostic de deux algorithmes: les \textit{random forest} et les perceptrons (réseaux de neurones simples). Nous avons retiré du jeu de données les 54 rapports sans diagnostic, car ils ne peuvent pas être utilisés pour l'entraînement des modèles à apprentissage supervisé ce qui aboutit à 138 rapports utilisés sur 4 labels différents pour l'entraînement des modèles (\gls{nm}, \gls{com}, \gls{cnm}, non-CM).
\begin{table}[!ht]
\centering
\caption{Nombre de rapports par diagnostic}
\label{tab:number_patients}
\begin{tabular}{|l|c|}
\hline
\textbf{Diagnostic} & \textbf{Nombre de rapports} \\
\hline
Myopathie à Némaline (NM) & 44 \\
\hline
Myopathie à Cores (COM) & 48 \\
\hline
Myopathie centronucléaire (CNM) & 16 \\
\hline
Diagnostic non établi (UNCLEAR) & 54 \\
\hline
Autre (non-CM) & 30 \\
\hline
\end{tabular}
\end{table}

\subsection{Résultats des entraînements et performances des systèmes d'\textit{embedding}}
Au total 8 conditions expérimentales pour la prédiction de diagnostics ont été évaluées: \textit{Embedding} OpenAI vs Instructor, rapports en Français vs traduits Anglais, \textit{Random Forest} vs Perceptrons. Pour chacune des conditions expérimentales, les hyper-paramètres des modèles ont été optimisés par grille et les performances ont été évaluées grâce à 10 cross-validations. Ceci a été fait pour (i) obtenir des modèles avec les meilleures performances possibles (optimisation par grille) et (ii) avoir une estimation robuste des performances (moyenne sur 10 essais par cross-validation). L'ensemble des résultats de ces entraînements (métriques de performances et modèles) sont disponibles en ligne à l'adresse: \href{https://wandb.ai/lambda-science/myo-text-classify/reports/MyoClassify-all-conditions-results--Vmlldzo0NDMyMTcw}{https://wandb.ai/lambda-science/myo-text-classify/reports/MyoClassify-all-conditions-results--Vmlldzo0NDMyMTcw}. 
\begin{table}[!ht]
\centering
\caption{Récapitulatif des performances des modèles \textit{MyoClassify}}
\label{tab:myoclassify_metrics}
\begin{tabularx}{\textwidth}{|X|c|c|c|c|c|}
\hline
\textbf{Nom} & \textbf{Exactitude} & \textbf{Exact. Pond.} & \textbf{F1 pond.} & \textbf{F1-Macro} & \textbf{F1 CNM} \\\hline
Instructor FR RF & 0.6522 & 0.5556 & 0.6231 & 0.5408 & 0.1053 \\ \hline
Instructor EN RF & 0.6667 & 0.5436 & 0.6215 & 0.5172 & 0 \\ \hline
Instructor FR MLPC & 0.6739 & 0.5703 & 0.6597 & 0.5608 & 0.07692 \\ \hline
Instructor EN MLPC & 0.6449 & 0.5761 & 0.6359 & 0.5792 & 0.32 \\ \hline
Openai FR RF & 0.6087 & 0.5591 & 0.6022 & 0.5777 & 0.4545 \\ \hline
Openai EN RF & 0.6522 & 0.5792 & 0.639 & 0.5996 & 0.4 \\ \hline
Openai FR MLPC & 0.6377 & 0.6 & 0.6345 & 0.6128 & 0.5385 \\ \hline
\textbf{Openai EN MLPC} & \textbf{0.6957} &\textbf{0.6666}& \textbf{0.6931} &\textbf{ 0.68} & \textbf{0.6429} \\ \hline
\end{tabularx}
\end{table}
\begin{figure}[!ht]
 \centering
 \includegraphics[width=1\textwidth]{figures/histo_myoclassify.png}
 \caption[Histogrammes des performances des modèle \textit{MyoClassify}]{Histogrammes des performances des modèles \textit{MyoClassify} pour le score F1-macro (haut) et le score F1 pour la classe minoritaire (CNM) uniquement (bas)}
 \label{fig:myoclassify_histo}
\end{figure}
\begin{figure}[!ht]
 \centering
 \includegraphics[width=1\textwidth]{figures/matrix_conf_myoclassify.png}
 \caption[Matrice de confusion \textit{MyoClassify}]{Matrice de confusion des modèles \textit{MyoClassify} pour le moins bon (\textit{instructor\_en\_rf} en haut) et le meilleur modèle (\textit{openai\_en\_mlpc} en bas) en terme de score F1}
 \label{fig:myoclassify_conf}
\end{figure}

Sur le tableau \ref{tab:myoclassify_metrics} et les figures \ref{fig:myoclassify_histo} et \ref{fig:myoclassify_conf} on observe que, au global, les performances à travers les conditions en terme de score F1 sont située dans un intervalle entre 0.51 et 0.68. Les modèles entraînés sur la base du modèle d'\textit{embedding} auto-hébergé \textit{Instructor} ont de moins bonnes performances à travers toutes les conditions. Dans le cadre du modèle d'\textit{embedding} d'OpenAI, la traduction des rapports en anglais a permis d'obtenir de meilleures performances dans toutes les conditions. De même, l'utilisation d'un perceptron multi-couche a permis d'obtenir de meilleures performances en terme de score F1 dans toutes les conditions par rapport à la \textit{random forest}.

De plus, il est à noter que pour la classe minoritaire (les \gls{cnm} avec 16 rapports), les performances des modèles sont très faibles pour l'\textit{embedding} du modèle \textit{Instructor}. Par exemple, pour notre modèle \textit{Instructor\_FR\_RF}, aucune \gls{cnm} n'a été prédite (matrice de confusion \ref{fig:myoclassify_conf}). Le modèle \textit{OpenAI\_EN\_MLPC} quant à lui obtient de meilleurs performances et a été capable de prédire 9 des 16 \gls{cnm}. 

En terme de performances brutes, il semble recommandable de: (i) traduire les comptes-rendus en anglais, (ii) d'utiliser le modèle d'OpenAI pour l'\textit{embedding} et (iii) d'entrainer un perceptron multi-couche pour apprendre à différencier les diagnostics en fonction des \textit{embeddings} des rapports. 

\section{\textit{MyoSearch}: un moteur de recherche de patients}
L'\textit{embedding} permet de représenter un texte sous forme numérique en capturant son sens sémantique. Il est alors possible de calculer un score de similarité entre une requête en texte libre et une masse de documents pour trouver le document le plus proche. Avec \textit{MyoSearch} nous avons créé un outil qui permet de faire des requêtes en texte libre parmi l'ensemble des rapports de biopsies de patients disponibles. Il est alors possible de chercher rapidement chez quels patients un symptôme ou diagnostic particulier est présent. Cette création de moteur de recherche est totalement automatique et ne nécessite aucun travail d'annotation. Elle se déroule en deux phases: (i) l'ingestion des données pour constituer la base de données puis (ii) la phase de requêtage de la base de données en fonction de l'entrée de l'utilisateur.

\subsection{Ingestion des rapports: création de la base de données de vecteurs}
\begin{figure}[!ht]
 \centering
 \includegraphics[width=1\textwidth]{figures/myosearch_ingest.png}
 \caption[Ingestion des données dans \textit{MyoSearch}]{Schéma de l'ingestion des données pour le moteur de recherche \textit{MyoSearch}}
 \label{fig:myosearch_ingest}
\end{figure}

A la différence de \textit{MyoClassify}, cette fois ci nous ne voulons pas générer 1 \textit{embedding} par document mais plutôt séparer les documents en fragments et avoir un \textit{embedding} par fragment de document. Nous avons choisis de découper les documents en fragments de le taille d'une phrase et ainsi d'obtenir un \textit{embedding} pour chaque phrase du document. Ceci permet d'obtenir de meilleur résultat lors du requêtage de la base de données car le sens sémantique de chaque phrase va pouvoir être comparé à la requête, plutôt que la moyenne de l'ensemble du document.

La figure \ref{fig:myosearch_ingest} présente la phase d'ingestion des données. Nous avons d'abord détecté le texte des rapports PDF par \gls{ocr}. Comme cette détection est hétérogène et bruitée, il est difficile de trouver les bornes exact des phrases. Dès lors, nous avons fragmenter le contenu en fragments de taille maximale de 100 \textit{tokens} (environ 30 à 50 mots) avec un recouvrement de 50. Pour 192 rapports, cela représente 8220 fragments de texte. Pour ces 8220 fragments nous avons calculé leur \textit{embedding} grâce au modèle \textit{Instructor} et les avons intégrés dans une base de données \textit{ChromaDB}, spécifique au stockage et requêtage de vecteurs. Pour chaque fragment, il est possible d'ajouter des méta-données qui peuvent servir de filtre pour les requêtes comme le diagnostic final ou le gène responsable de la maladie.

\subsection{Requêtage des données}
\begin{figure}[!ht]
 \centering
 \includegraphics[width=1\textwidth]{figures/myosearch_query.png}
 \caption[Requêtage des données dans \textit{MyoSearch}]{Schéma du requêtage des données dans \textit{MyoSearch}}
 \label{fig:myosearch_query}
\end{figure}
Quand l'ensemble des documents a été découpé et ingéré dans la base de données, il est possible de réaliser des requêtes. La figure \ref{fig:myosearch_query} présente la phase de requêtage des données. L'utilisateur peut entrer \textit{via} l'interface web un symptôme d'intérêt en texte libre tel que "surcharge lipidique". Cette requête va ensuite être transformée en vecteur numérique par le modèle d'\textit{embedding }\textit{Instructor}. Ce vecteur va être comparé à la base de données de vecteurs pour rechercher les top 5 plus proches voisins grâce à un algorithme nommé \textit{Hierarchical Navigable Small Worlds, HNSW}. Les cinq fragments avec les scores de similarité les plus élevés sont ensuite affichés sur l'interface web. Le tableau \ref{tab:myosearch_results} présente les résultats obtenus pour une requête dans MyoSearch. Par exemple pour la requête "surcharge lipidique" les trois rapports les plus proches font mention d'une surcharge en lipides chez des patients dont (i) le diagnostic n'est pas connu, (ii) le diagnostic n'est pas une myopathie congénitale et (iii) chez un patient avec une \gls{nm}. Cette recherche est aussi multilingue, le modèle d'\textit{embedding} autorise des recherches entre une base de données française avec requête en anglais, ou inversement.
\begin{table}[!ht]
\centering
\caption{Exemple d'une requête et des résultats de \textit{MyoSearch}}
\label{tab:myosearch_results}
\begin{tabularx}{\textwidth}{|l|X|p{1cm}|p{2cm}|}
\hline
\textbf{Requête} & \textbf{Fragment le plus similaire} & \textbf{Rang} & \textbf{Rapport et diagnostic} \\\hline
"surcharge lipidique" & "d’inclusions. Il est à noter que l’on observe également une surcharge importante en lipides dans" \newline & 1 & 13405-105.txt UNCLEAR \\
 & "une myopathie congénitale. D'autre part, une surcharge importante en lipides qui nécessite"\newline & 2 & 11391-79.txt non-CM \\
 & "de surcharge en lipides - Technique de Koëlle : CONCLUSIONS : Anomalies caractéristiques d’une" & 3 & 5060-35.txt NM \\ \hline
\end{tabularx}
\end{table}

\section{Déploiement de l'outil}
Développé de façon \textit{open-source}, le code source de \gls{nlmyo} est disponible sur GitHub à l'adresse: \href{https://github.com/lambda-science/NLMyo}{https://github.com/lambda-science/NLMyo} soit une licence AGPL-3 assurant le statut open-source de l'outil. Une version de démonstration en ligne est déployée grâce à \textit{Streamlit} à l'adresse \href{https://lbgi.fr/NLMyo/}{https://lbgi.fr/NLMyo/}. Comme \gls{nlmyo} propose l'utilisation de \gls{llms} auto-hébergé, l'outil est hébergé sur un serveur avec un processeur 64 coeurs pour accélérer l'inférence du notre \gls{llms}. Si l'outil n'utilise que l'API OpenAI comme modèle génératif et d'\textit{embedding} alors il est possible d'héberger l'application sur un serveur avec très peu de ressource de calcul.

\section{Discussions et perspectives de développement}
\gls{nlmyo} met à disposition des outils permettant le traitement de façon massive de rapports de comptes-rendus médicaux et notamment des rapports de biopsie. Cependant, les défis pour rendre l'outil plus robuste sont multiples. 

Le premier défi concerne MyoSearch, le moteur de recherche de données de patients. Bien que la méthode soit fonctionnelle et novatrice, il n'est actuellement possible que de chercher un symptôme à la fois. De plus, les résultats obtenus ne sont pas tout le temps pertinents. Un travail d'amélioration de la méthode de fragmentation et de requêtage est nécessaire. Par exemple, il faudrait créer un système permettant de croiser les résultats des requêtes pour plusieurs symptômes, permettant ainsi de chercher un profil de symptômes complet. De plus, l'ajout de méta-données supplémentaires aux fragments (telles que les informations complètes sur les patients) permettrait de réaliser des requêtes plus fines pour ne sélectionner, par exemple, que les patients liés à un gène en particulier.

Le second défi majeur concerne la protection de la confidentialité des données de santé. En effet, certains outils, pour obtenir les meilleures performances, reposent sur l'utilisation de \gls{llms} externes \textit{via} des l'\gls{api} OpenAI, ce qui est problématique dans le cadre de données sensibles, même anonymisées. Pour cela, nous avons aussi proposé une alternative avec un modèle auto-hébergé mais pour l'instant celui-ci sous-performe. Par exemple dans \textit{MyoExtract}, les informations extraites sont incomplètes et dans \textit{MyoClassify} les scores d'exactitude et F1 sont globalement plus faibles voire très faibles pour la classe minoritaire (\gls{cnm}). Cependant, la recherche en terme de \gls{llms} est un domaine très dynamique et il est très probable qu'une solution auto-hébergée et performante soit disponible sous peu.

\part{DISCUSSIONS}
\chapter*{Chapitre 8: Discussions et perspectives}
% ...
Placeholder
%
% Chapitre  de conclusion (générale)
%%%%%%%%%%%%%%%%%%%%%%%%%%%%%%%%%%%%%%%%%%%%%%%%%%%%%%%%%%%%%%%%%%%%%%%%%%%%%%%
%\include{corps/conclusion}
%
% Liste des références bibliographiques
\printbibliography
%
%%%%%%%%%%%%%%%%%%%%%%%%%%%%%%%%%%%%%%%%%%%%%%%%%%%%%%%%%%%%%%%%%%%%%%%%%%%%%%%
% Début de la partie annexe éventuelle
%%%%%%%%%%%%%%%%%%%%%%%%%%%%%%%%%%%%%%%%%%%%%%%%%%%%%%%%%%%%%%%%%%%%%%%%%%%%%%%
\appendix
%
% Premier chapitre annexe (éventuel)
\include{annexes/annexeI}
%
% Deuxième chapitre annexe (éventuel)
% \include{annexes/annexeII}
%
%%%%%%%%%%%%%%%%%%%%%%%%%%%%%%%%%%%%%%%%%%%%%%%%%%%%%%%%%%%%%%%%%%%%%%%%%%%%%%%
% Début de la partie finale
%%%%%%%%%%%%%%%%%%%%%%%%%%%%%%%%%%%%%%%%%%%%%%%%%%%%%%%%%%%%%%%%%%%%%%%%%%%%%%%
% \backmatter
%
%
% (Facultatif) Index :
% \printindex
%
% Table des matières
%\tableofcontents
%
% (Facultatif) Production de la 4e de couverture :
\makebackcover
%
\end{document}

