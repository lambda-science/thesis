\begin{abstract}
La croissance exponentielle des données générées par le séquençage de nouvelle génération, l'imagerie médicale et les dossiers médicaux électroniques met en évidence le besoin d'outils spécialisés pour l'exploitation des données biomédicales multimodales. Nous avons développé des méthodes basées sur l'IA pour explorer les données de patients atteints de myopathies congénitales, une famille de maladies génétiques rares difficiles à diagnostiquer.

Tout d'abord, nous avons développé IMPatienT (Integrated digital Multimodal PATIENt daTa), une base de données de patients en ligne permettant d'annoter et d'explorer manuellement les rapports et les images histologiques des patients. Après avoir numérisés 89 rapports de biopsie nous avons utilisé cette base de données pour comparer les performances de 11 algorithmes de classification.

Ensuite, nous avons développé un outil appelé NLMyo (Natural Language Myopathies) qui s'appuie sur les développements récents dans le domaine du traitement du langage naturel, tels que ChatGPT et LLAMA, pour exploiter des rapports médicaux en texte libre. NLMyo est une boîte à outils qui permet d'anonymiser et d'extraire des informations, de faciliter le diagnostic et de créer automatiquement un moteur de recherche sur les symptômes des patients. Nous avons utilisé NLMyo pour analyser un corpus de 192 rapports de biopsie de patients atteints de myopathies congénitales.

Enfin, nous avons développé un outil appelé MyoQuant pour quantifier automatiquement les marqueurs pathologiques dans les biopsies de fibres musculaires. MyoQuant s'appuie sur des modèles d'IA récents dans le domaine de l'imagerie biomédicale, tels que Cellpose et Stardist. En utilisant des algorithmes et des modèles d'IA personnalisés, il peut quantifier automatiquement les marqueurs pathologiques dans trois colorations standard pour le diagnostic des myopathies congénitales (HE, ATPase et SDH), telles que les noyaux centralisés, le déséquilibre des types de fibres et les anomalies de répartition des mitochondries.

IMPatient, NLMyo et MyoQuant sont disponibles en tant qu'outils open source et également en version démo en ligne.

\end{abstract}
\begin{abstract}
The exponential growth of data generated by next-generation sequencing, medical imaging, and electronic health records highlights the need for specialized tools for the exploitation of multimodal biomedical data. We have developed AI-based methods to explore data from patients with congenital myopathies, rare genetic diseases that are difficult to diagnose.

First, we developed IMPatienT (Integrated digital Multimodal PATIENt daTa), an online patient database to perform manual annotation and exploration of patient histology reports and images. After digitizing 89 biopsy reports we used this database to compare the performance of 11 classification algorithms.

Then we developed a tool called NLMyo (Natural Language Myopathies) that is built on recent development in natural language processing such as ChatGPT and LLAMA to exploit free-text medical reports. NLMyo is a toolbox to anonymize and extract information, facilitate diagnosis and create a patient symptoms search engine automatically. We used NLMyo to analyze a corpus of 192 biopsy reports of patients with congenital myopathies.

Finally, we developed a tool called MyoQuant to automatically quantify pathological features in muscle fiber histology images. MyoQuant is built on recent AI models in biomedical imaging such as Cellpose and Stardist. Using custom algorithms and AI models, it can automatically quantify pathological features in three standard stainings for congenital myopathies diagnosis (HE, ATPase and SDH), such as centralized nuclei, fiber type imbalance and mitochondria repartition anomalies.

IMPatient, NLMyo and MyoQuant are available as open source tools and also as demo version online.


\end{abstract}
\makeabstract
