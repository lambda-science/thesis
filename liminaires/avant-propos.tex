\chapter{Avant-Propos}
L'objectif de cette thèse est de développer des méthodes et des outils d'intelligence artificielle adaptés à l'exploitation des données biomédicales multimodales, en particulier les comptes rendus médicaux et les données d'imagerie avec une application aux myopathies congénitales. Cette thèse a donné lieu aux développement de trois outils mettant à disposition les moyens nécessaires à l'exploitation de ces données. \\


L'\textbf{introduction} est séparée en trois chapitres (\textbf{chapitres 1-3}). Les deux premiers chapitre posent le contexte informatique dans lequel s'inscrivent ces travaux. 

Le \textbf{premier chapitre} s'intéresse à l'émergence du \textit{Big-Data} dans le cadre des données bio-médicale et décrit les principes de bases et les approches traditionnelle utilisée en intelligence artificielle pour analyser les données.

Le \textbf{second chapitre} d'introduction s'intéresse plus spécifiquement aux \gls{ia} de type réseaux de neurones et comment cette nouvelle technologie transforme la manière de traiter les données biomédicale multimodales avec des exemples de d'\gls{ia} déjà existantes en imagerie et en analyse de texte.

Enfin le \textbf{troisième chapitre} d'introduction pose le contexte biologique de la thèse avec une présentation de notre cas d'application: les myopathies congénitales. On y retrouve d'abord une présentation générale du muscle puis une description des myopathies congénitale, de leur diagnostic et des données générée suite à celui-ci. \\


Le \textbf{quatrième chapitre} \textbf{matériels et méthodes} décrit l'ensemble des ressources utilisées pour développer ces outils. On y retrouvera d'abord une description des ressources biologiques comme les ontologies et source de données biomédicales utilisée. Puis l'ensemble des ressources informatique nécessaire au développement des outils tels que les algorithmes utilisés, les bibliothèques de programmation, le matériel informatique, les modèles pré-existants ainsi que les méthodes d'évaluation des performances. Enfin dans une dernière section, l'accent est mis sur les outils permettant de rendre ces travaux de recherches \textit{open-source} et reproductible. \\


La partie \textbf{contributions} est divisée en quatre chapitres (\textbf{chapitres 5-8}). 

Le \textbf{chapitre 5 }présente \gls{impatient}, le premier outil développé durant cette thèse, qui est une base de données d'annotations de comptes rendus et d'images de patients. Cet outil a fait l'objet d'une soumission dans un journal à comité de lecture dont le manuscrit est intégré au chapitre. 

Le \textbf{chapitre 6} s'intéresse à l'analyse détaillée de 89 comptes rendus de patients intégré dans cette base de données et leur classification par \gls{ia} traditionnelle et \gls{ia} explicable.

Le \textbf{chapitre 7} présente \gls{nlmyo}, une boite à outil basé sur les modèles linguistiques de grande tailles pour l'exploitation automatique des comptes rendus médicaux. Cette boite à outil permet d'anonymiser et d'extraire de l'information de comptes rendus textuels, ainsi que de les classer et de créer un moteur de recherche de symptomes automatiquement. 

Enfin le \textbf{chapitre 8 }présente la possibilité de générer de manière automatique des comptes rendus de biopsie grâce à des méthodes de quantification par intelligence artificielle. Pour cela nous avons développé \gls{myoquant}, un outils de quantification automatique de marqueurs pathologiques sur des biopsie musculaire. \\


Pour finir le \textbf{chapitre 9 discussions et ouvertures} traite des principaux challenges, limites et perspectives des outils développés. Tout d'abord des perspectives biologiques avec l'intégration des données génomiques et la mise en relation des différentes modalités. Mais aussi des perspectives techniques notamment en terme d'explicabilité de l'\gls{ia}, du déploiement de ce type d'approche à grande échelle en terme de ressources informatique et des questions de législation en terme de données de santé et de traitement automatique. Une ouverture supplémentaire est fait sur la possibilité de création d'une produit regroupant l'ensemble de ces outils dans le cadre du concours \textit{Mature Your PhD} organisé par la Satt Connectus.