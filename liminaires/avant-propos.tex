\chapter{Avant-Propos}
L'objectif de cette thèse est de développer des méthodes et des outils d'intelligence artificielle adaptés à l'exploitation des données biomédicales multimodales, en particulier les comptes rendus médicaux et les données d'imagerie avec une application aux myopathies congénitales. Cette thèse a donné lieu au développement de trois outils mettant à disposition les moyens nécessaires à l'exploitation de ces données. \\


L'\textbf{introduction} est séparée en trois chapitres (\textbf{chapitres 1-3}). Les deux premiers chapitres posent le contexte informatique et biologique dans lequel s'inscrivent ces travaux. 

Le \textbf{chapitre 1} s'intéresse à l'émergence du \textit{Big-Data} dans le cadre des données biomédicales, une partie de la diversité des données biomédicales et décrit les principes de bases et les approches traditionnelles utilisées en intelligence artificielle pour analyser les données.

Le \textbf{chapitre 2} s'intéresse plus spécifiquement aux \gls{ia} de type réseaux de neurones et comment cette nouvelle technologie transforme la manière de traiter les données biomédicales multimodales non structurées avec des exemples d'\gls{ia} déjà existantes en imagerie et en analyse de texte.

Enfin, le \textbf{chapitre 3} pose le contexte biologique de la thèse avec une présentation de notre cas d'application: les myopathies congénitales. On y retrouve d'abord une présentation générale du muscle puis une description des myopathies congénitales, de leur diagnostic et des données générées suite à celui-ci. \\


Le \textbf{chapitre 4} \textbf{matériels et méthodes} décrit l'ensemble des ressources utilisées pour développer ces outils. On y retrouvera d'abord une description des ressources biologiques comme les ontologies et la source de données biomédicales utilisées. Puis l'ensemble des ressources informatiques nécessaires au développement des outils est détaillé, tel que les algorithmes utilisés, les bibliothèques de programmation, le matériel informatique, les modèles préexistants ainsi que les méthodes d'évaluation des performances. Enfin, dans une dernière section, l'accent est mis sur les outils permettant de rendre ces travaux de recherche \textit{open-source} et reproductibles. \\


La partie \textbf{contributions} est divisée en quatre chapitres (\textbf{chapitres 5-8}). 

Le \textbf{chapitre 5 }présente \gls{impatient}, le premier outil développé durant cette thèse, qui est une application web d'annotation de comptes rendus et d'images de patients. Cet outil a fait l'objet d'un article soumis dans un journal à comité de lecture dont le manuscrit est intégré au chapitre. 

Le \textbf{chapitre 6} s'intéresse à l'analyse détaillée de 89 comptes rendus de patients intégrés dans \gls{impatient} et leur classification par modèles de\gls{ml} classiques et \gls{ml} explicable.

Le \textbf{chapitre 7} présente \gls{nlmyo}, une boite à outil basé sur les modèles linguistiques de grandes tailles pour l'exploitation automatique des comptes rendus médicaux. Cette boite à outils permet d'anonymiser et d'extraire de l'information de comptes rendus textuels, ainsi que de les classer et de créer un moteur de recherche de symptômes automatiquement. 

Enfin, le \textbf{chapitre 8 }présente la possibilité de générer de manière automatique des comptes rendus de biopsie grâce à des méthodes de quantification par \gls{ia}. Pour cela, nous avons développé \gls{myoquant}, un outil de quantification automatique de marqueurs pathologiques sur des biopsies musculaires. \\


Pour finir le \textbf{chapitre, 9 discussions et ouvertures} traitent des principaux challenges, limites et perspectives des outils développés. Tout d'abord, des perspectives biologiques sont traitées avec l'intégration des données génomiques et la mise en relation des différentes modalités. Des perspectives techniques sont également considérées, notamment concernant l'explicabilité de l'\gls{ia}, du déploiement de ce type d'approche à grande échelle en termes de ressources informatiques et des questions de législation en termes de données de santé et de traitement automatique. Une ouverture supplémentaire est faite sur la possibilité de création d'un produit regroupant l'ensemble de ces outils dans le cadre du concours \textit{Mature Your PhD} organisé par la Satt Connectus.