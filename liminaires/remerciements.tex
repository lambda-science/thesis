\chapter{Remerciements}
Tout d'abord, je souhaite remercier les membres de mon jury de thèse qui me font l'honneur d'avoir accepté d'évaluer mes travaux de thèse. Merci au Dr Malika Smail-Tabbone, au Dr Antonio Rausell, au Dr Gisèle Bonne et au Pr Cédric Wemmert. \\

J'aimerais remercier l'ensemble des personnes qui ont rendus cette thèse possible, à commencer par l'équipe du CSTB, une équipe que je pourrais qualifier d'atypique, par sa diversité tant sur le plan scientifique que humain. Une équipe qui m'a permis de découvrir le monde de la recherche et de découvrir ce que je souhaitais vraiment faire. Alors merci aux permanents de l'équipe dans leur ensemble. Je souhaiterais remercier en particulier un premier bureau, celui de Laetitia, Arnaud et Nicolas. Pour toi Arnaud la lutte continue, merci pour ces discussions techniques, mais aussi politiques, culturelles et ces \textit{memes} très obscurs. Nicolas je te souhaite soit d'être parmi les prochains, soit de réussir à ouvrir une salle d'escalade dans un autre pays. 
J'aimerais avoir un mot particulier pour le duo Julie et Odile. Merci, Julie, de m'avoir donné l'opportunité de commencer en stage avec toi, puis de continuer en thèse dans cette équipe et enfin de m'avoir aidé dans cette dernière ligne droite. J'ai toujours admiré ta modestie et ta capacité à aller droit au but. Merci à toi Odile de m'avoir donné l'opportunité d'enseigner à l'ESBS c'était une expérience formidable qui m'a beaucoup aidé et je me suis même découvert une petite passion pour l'enseignement. En dehors du travail, vous êtes deux personnes que j'admire beaucoup sur le plan humain, avec vous, la bio-informatique à Strasbourg est sereine. \\

Je tiens à remercier aussi ceux que je nommerais les précaires, ces stagiaires et doctorants que j'ai pu rencontrer dans l' équipe et avec qui j'ai pu tisser du lien. Merci à Christelle, Hiba et Amani du "bureau d'à côté", vous êtes toutes les trois les prochaines à soutenir et je suis persuadé que cela va bien se passer. Enfin, merci aux personnes qui ont animé et habité le même bureau que moi, par le passé ou par le présent, merci à Célia, Lucille, Dorine et Alix, c'était un plaisir de partager ces moments avec vous qui ont donné lieu à des tranches de vie pleines de rebondissements, mais qui n'en resteront pas moins de bons souvenirs. \\

Merci à Jocelyn Laporte et son équipe ainsi qu'à l'équipe de Norma Romero pour ces collaborations qui ont porté leurs fruits à travers ces travaux de thèse. Vos connaissances extensives des myopathies ont été très stimulantes et nos échanges m'ont toujours permis de repartir l'esprit plein de nouvelles idées à mettre en place. \\

Enfin, je souhaite dédier un remerciement particulier pour mes directeurs de thèses et mes encadrants. Merci à Anne et Pierre pour leurs conseils et expertises en IA (ainsi qu'en philosophie !). Merci à ce duo de choc (c'est le cas de le dire), Olivier et Kirsley, dont l'énergie et l'animation contrebalancent et n'ont d'égales que le calme et l'organisation d'Odile et Julie. Merci à Kirsley, toujours au four et au moulin entre les rédactions de \textit{grant}, le développement et l'encadrement de stagiaires, pour ses connaissances techniques et sa capacité à toujours aller plus loin dans les questionnements scientifiques. Merci à Olivier d'avoir été infatigablement derrière cette thèse du début à la fin, même dans les moments de flottement. Ainsi que pour cette capacité de raconter des histoires en continu dès que l'occasion se présente. \\

De manière générale, à tous ceux qui ont commencé comme collègues de travail, mais qui sont devenus bien plus que ça: merci, on sera amené à se revoir. 

Avant d'être une aventure professionnelle, la thèse est avant tout une aventure personnelle. J'aimerais dédier cette partie à l'ensemble de mes proches qui n'ont pas été impliqués directement dans cette thèse, mais sans qui elle n'aurait jamais vu le jour. Je me dois donc de dédier cette thèse à un certain nombre de personnes. \\

À mes frères du groupe \textit{Persepolis} et ADN: Adil, Arthur, Ernest, Keziah, Lucie, Malo et les autres. Merci. Du fond du cœur. Vous côtoyer tous les jours est un réel plaisir et j'espère que ça va perdurer le plus longtemps possible, on se sait. On ira au Japon à un moment, je vous le jure.

À ceux qu'on côtoie moins à cause des trajets de vie, mais avec qui chaque retrouvaille est comme un retour à hier: merci, Bilal, Vincent et Baptiste. Vous avez vu, j'ai fini, je vais peut-être pouvoir sortir de la salle du temps.

Au Discord des doctorants français "\textit{PhD Students}" et à sa communauté. C'est étrange de remercier une entité, mais j'aurais clairement arrêté la thèse sans ce refuge et sans avoir échangé (et râler en cœur) avec tant de gens géniaux. Merci notamment à Anaïs et Floriane. Et merci à la délégation strasbourgeoise avec qui on a pu échanger tant de bons moments (et qui j'espère vont continuer). Merci au duo Alix et Alix, Emilien, Erin, Maï et Nato. Pour certains d'entre vous aussi, c'est bientôt votre tour. Un mot en particulier pour Alix n°2, camarade de conférences, de \textit{gossip} et de discussions désastreuses, courage pour la suite, tu es bien entourée.

À tous ceux qui m'ont permis de sortir la tête de la thèse, souvent par l'escalade, mais pas que, merci. Merci à Louise, Magalie, Morgane, Pierre. \fontencoding{T2A}\selectfont Спасибо, Настя, надеюсь, что в Бретани нам удастся поесть морепродуктов !\fontencoding{T1}\selectfont \\

Enfin, il me reste à remercier trois personnes spéciales, qui constituent ma famille. Je ne suis pas très doué pour le montrer, mais vous savez que je vous aime. Merci, Maman, merci, Olivier, merci, Coraline. Sans votre soutien inconditionnel toutes ces années durant, je ne serais évidemment pas ici aujourd'hui. Je sais que pour vous, le travail de thèse est assez obscur, j'avais peur de ne pas réussir à aller au bout, mais ne vous inquiétez pas après ça, c'est fini. \\

À tous mes proches que je remercie sincèrement à travers ces mots et qui savent que je ne suis pas forcément le plus habile pour ce genre d'exercice, si je devais résumer le fond de ma pensée en trois mots: \\
\begin{center}
\Large \textbf{Vous êtes lumineux.}
\end{center}
