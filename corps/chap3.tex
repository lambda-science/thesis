\chapter{L’exemple des myopathies congénitales et la difficulté du diagnostic}
Dans cette thèse, nous cherchons à développer des méthodes \gls{ia} pour exploiter les données biomédicales. En guise de cas d'application pour développer et évaluer ces méthodes, nous  nous intéressons aux myopathies congénitales, une famille de maladies des muscles rare et génétique dont le diagnostic est complexe. Dans ce chapitre nous allons présenter le contexte biologique de cette maladie avec une présentation de la structure du muscle, de la classification des myopathies congénitales et de leur diagnostic.

\section{Le muscle, un organe particulier assurant des fonctions diverses}
Le muscle est un organe particulier tant par sa structure que son abondance dans le corps humain. Pour un adulte, les muscles représentent près de 40\% de la masse corporelle totale (\cite{janssen_skeletal_2000}). Les muscles assurent une variété de fonction dont le mouvement et la posture, mais aussi la thermogénèse et l'équilibre métabolique.

Dans le muscle, une variété de processus sont nécessaires pour permettre d'assurer ses fonctions. Le muscle intègre des processus neurologiques pour le transfert de signal induisant la contraction. Cette contraction est possible grâce à une structure particulière du muscle en fibres que nous présenterons en détail dans une section suivant. De plus, cette contraction requiert de l'énergie qui est assurée par des processus métaboliques. Enfin le muscle, en raison des contraintes physiques subit lors des mouvements, est un organique dynamique, ainsi des processus assurant la régénération musculaire sont à l'oeuvre pour assurer sa plasticité.

\subsection{Type de muscles}
Il existe dans le corps humains trois types de muscles différents avec des structures différentes en fonction de la fonction qu'ils doivent assurer (figure \ref{fig:muscle-type}, \cite{gomez_oca_physiological_2021}): le muscle lisse, le muscle cardiaque et le muscle strié squelettique.
\begin{figure}[!ht]
 \centering
 \includegraphics[width=0.8\textwidth]{figures/muscle_type.png}
 \caption[Schéma des trois types de muscles]{Schéma des trois types de muscles (traduit de \cite{gomez_oca_physiological_2021})}
 \label{fig:muscle-type}
\end{figure}
\subsubsection{Muscle lisse}
Le muscle lisse, aussi nommé non-strié, tire son nom de l'absence de striations lors de l'observation au microscope. Les fibres ne possède qu'un seul noyaux central. Ce type de muscles est présent dans les vaisseaux sanguins, l'appareil digestif, l'appareil respiratoire, urinaire et la parois des viscère. La particularité des muscles lisses est qu'ils ne contractent que de manière involontaires, sans contrôle conscient. La contraction de ces muscles est régie par le système nerveux neuro-végétatif (système autonome)

\subsubsection{Muscle strié cardiaque}
Le muscle strié cardiaque est un muscle qui se retrouve exclusivement dans le coeur. Il partage des caractéristiques commune à la foi au muscle lisse et au muscle strié squelettique. Ce muscle se contracte de manière involontaire, et les fibres ne possède qu'un seul noyaux central similairement au muscle lisse. Mais le muscle cardiaque présente des striations similaires au muscle strié squelettique. La caractéristique unique du muscle strié cardiaque est qu'il fonctionne en continu et de se contracter en rythme de façon coordonnée. Ainsi ce muscle est très dépendant du métabolisme oxydatif.

\subsubsection{Muscle strié squelettique}
Enfin les muscles striés squelettiques (nommés communément muscles volontaires) sont les muscles mobilisés lors des mouvements conscient et volontaire, lorsque l'on porte un objet ou que l'on fait du sport. Au miscoscope ces muscles se démarquent par la présence de stries transversales et longitudinales. Ces cellules composants les fibres musculaires du muscle strié squelettique ont la particularité d'avoir fusionner ensembles formant ainsi un "syncitium vrai". Cette fusion donne donc aux fibres musculaires les caractéristiques des cellules géantes (entre 1 et 5 cm de long et 10 à 100µm de diamètre). Ainsi ces cellules osnt multinuclées avec des noyaux périphériques.

Comme leur nom l'indique, ces muscles sont reliés aux os par l'intermédiaire du tendon, leur contraction permet donc le mouvement des os et donc du corps. Cette contraction est engagée sous le contrôle du système nerveux somatique. En fonction de l'intensité et de la durée de la contraction, les muscles striés squelettiques peuvent fonctionner de manière aérobie grâce à leur vascularisation importante ou de façon anaérobie.

Dans le cadre des myopathies congénitales, les muscles striées squelettiques sont affectés et ne sont plus capables d'assurer leur fonction normale en raison d'altérations structurelles, entraînant un déficit de tonus musculaire et de force.

\subsection{Structure du muscle strié squelettique}
Pour comprendre comment les altérations de la structure du muscle qui ont leur lors des myopathies congénitales peuvent mener à son dysfonctionnement, il est important de comprendre comment le muscle est structurer et comment il peut se contracter.

L'organisation du muscle peut être décrite comme celle d'une corde d'escalade. Une corde d'escalade semble être composée d'un seul élément fort et résistant. Mais si l'on regarde de plus près, une corde d'escalade est constituée d'une gaine qui entoure l'âme de la corde. Cette âme est composée de plusieurs gros filaments, eux-même composée de plusieurs filaments de plus en plus fins, torsadés ensemble. 

Ainsi par une métaphore, le muscle entier est comme la corde d'escalade, sa structure est décrite dans la figure \ref{fig:muscle_struct} (\cite{burr_basic_2019}). Le muscle entier entouré par sa gaine nommée \textit{epimysium} est constitué de plusieurs faisceaux musculaire qui le composent. Chacun de ces faisceaux ou fascicules, est composée de filaments encore plus fins nommés fibre musculaire. Ces fasicules et fibres musculaire sont encore observables au microscope optique. Enfin, chacun de ces fibres musculaire est en fait un regroupement de plus petits filaments nommées Myofribrilles qui  sont des filaments composés d'une multitude de myofilaments capables de se contracter. Les myofibrilles et myofilaments ne sont observables qu'en microscopie électronique.

\begin{figure}[!ht]
 \centering
 \includegraphics[width=0.8\textwidth]{figures/muscle.png}
 \caption[Schéma de la structure du muscle strié squelettique (modifié de \cite{burr_basic_2019}]{Schéma de la structure du muscle strié squelettique (modifié de \cite{burr_basic_2019}.}
 \label{fig:muscle_struct}
\end{figure}
A une échelle encore plus petite, les myofibrilles sont organisée en plusieurs sarcomère, présentés en figure \ref{fig:sarcomere} (\cite{burr_basic_2019}). Les sarcomères sont l'unité contractile de base des myofibrilles. Il est constitué des trois systèmes de filaments protéïques: (i) un filaments épais de myosine, (ii) un filament mince d'actine inséré sur le disque Z et (iii) un filament élastique enchâssé sur le filament de myosine composé de la titine inséré sur le disque Z aussi.

Lors d'une contraction musculaire les filaments de myosine et d'actine dans la bande A glissent les uns sur les autres, réduisant ainsi la zone H, le muscle est contracté, le sarcomère est raccourcis. On comprend alors qu'un dysfonctionnement dans l'une de trois protéines essentielle à ce mouvement de contraction altère la structure du sarcomère et donc la capacité contractile du muscle. Les gènes TTN, ACTA1 et MYH2/MYH7 (codant respectives pour la titine, actine et myosine) sont des gènes typiquement responsables de myopathies lors qu'ils sont mutés. Cependant nous verrons l'ensemble des gènes responsables des myopathies congénitales dans un section suivante.
\begin{figure}[!ht]
 \centering
 \includegraphics[width=0.8\textwidth]{figures/sarcomere.png}
 \caption[Schéma de la structure du sarcomère.]{Schéma de la structure du sarcomère (modifié de \cite{burr_basic_2019}).}
 \label{fig:sarcomere}
\end{figure}

\subsection{Types de fibres musculaires}
Les fibres musculaires peuvent être classée en trois sous-type de fibres: les fibres de type 1, les fibres de type 2A et 2B qui ont des caractéristiques stucturelles et fonctionnelles différentes. Cette classification repose sur le profil d'expression de la chaine lourde de la myosine. Il existe plusieurs isoforme de la myosine, permettant chacun une contraction plus ou moins rapide et résitante à la fatigue. Le tableau \ref{table:fiber-compare} répertorie les différence principales entre les fibre de type 1, 2B et 2A.

\begin{table}[!ht]
\centering
\begin{tabular}{|l|c|c|c|} 
 \hline
 \textbf{Caractéristique} & \textbf{Fibre Type 1} & \textbf{Fibre Type 2B}  & \textbf{Fibre Type 2A} \\
 \hline
Couleur & Rouge & Pâle & Pâle \\
Vitesse de Contraction & Lente & Rapide & Rapide \\
Voie métabolique & Aérobie & Anaérobie & Aérobie et anaérobie \\
Réserve en oxygène & Importante & Faible & Moyenne \\
Réserve en glycogène & Faible & Importante & Moyenne \\
Fatiguabilité & Faible & Importante & Moyenne \\
 \hline
\end{tabular}
\caption{Tableau de comparaison des fibres de type 1, 2B et 2A.}
\label{table:fiber-compare}
\end{table}

Les fibres de type 1, nommées fibres rouges, sont des fibres à contraction lente. Elles ont en général un plus petit diamètre et sont plus vascularisée. Ce sont des fibres spécialisée dans l'aérobie et très résistantes à la fatigue. Elles sont efficaces dans l'utilisation de l'oxygène pour générer de l'\gls{atp} qui est la source d'énergie permettant la constraction musculaire. Leur couleur rouge provient de la présence de myoglobine, protéine sotckant l'oxygène dans le muscle. Ces fibres sont utilisée pour le maintient de posture et sont mobilisée dans les activité d'endurance à faible intensitée.


Les fibres de type 2B sont des fibres à contraction rapide. Ces fibres sont aussi nomée fibres blanches, elles sont spécialisée dans les mouvements rapides et explosifs. Ces fibres utilisent des voies anaérobie pour générer de l'\gls{atp} et se fatiguent beaucoup plus rapidement. La voie anaérobie ne repose pas sur l'oxygène mais l'utilisation du glycogène. Ces fibres possèdent donc des réserves de glycogène plus importantes que les fibres de type 1. Ces fibres sont mobilisée dans le cadre d'activité intenses tel que des sprints.


Enfin les fibres de type 2A représentent un intermédiaire entre les fibres de type 1 et 2B.  Ces fibres peuvent génèrer de l'énergie (\gls{atp}) à la fois par la voie aérobie et anaérobie. Similairement aux fibres 2B, elles peuvent génrer des contractions puissantes mais se fatiguent rapdiement. Ces fibres sont mobilisée lors des exercices à intensité moyenne demandant de l'endurance tels que la natation.


La balance entre fibre type 1 et 2A/2B dans un muscle est un marqueur important des myopathies congénitales. Souvent dans le cadre de myopathies, une prédominance des fibres de type 1 va se manifester dans le muscle. En microscopie, la visualisation de ces fibres se réalise par des méthodes histochimiques tel que la coloration ATPase qui colore différentiellement les fibres de type 1 et fibre de type 2. 

\subsection{Classification des atteintes neuromusculaires}
Les variétés des processus impliqué dans le fonctionnement normal du muscle strié squelettique ouvrent la portes à tout autant de dysfonctionnements potentiels pouvant amener à une mal fonction du muscle. Ces mal fonction peuvent provoquer des maladies aux manifestations variées que l'on nomme les \gls{nmd}. Les \gls{nmd} ont une prévalence d'environ 3.7 à 4.9 pour 10 000 (\cite{lace_overview_2022}), ce qui les classe parmi les maladies rares (prévalence inférieur à 5 pour 10 000).  Les \gls{nmd} peuvent être classifiée parmis quatres grandes catégories présenté dans la table \ref{table:nmd} (\cite{lornage_identification_2019}). Les dystrophies musculaires sont caractérisée principalement un une perte progressive de la force et de la masse musculaire. Les pâtients atteients de myopathies métaboltiques présentent une forte intolérence à l'exercice et des épisode de fatigues. Les myopathies mitochondriales sont aussi caractérisées par uen faiblesse muscsulaire et une intolérence à l'exercice mais en plus par des problêmes cardiaques, audititifs et des crises d'épilepsies.

Enfin les myopathies congénitales, qui sont le sujet principal de notre cas d'application, sont des maladies avec un départ précoce de la faiblesse musculaire (souvent dès la naissance) et dont la prograssion est lente. De plus les patients atteint de myopathies congénitales présentent souvent des caractéristiques faciles dysmorphiques (visage, bouche). Dans la prochaine section, nous allons voir en détails la classification et la prévalence des myopathies congénitales ainsi que leur diagnostic.
\begin{table}[!ht]
\centering
\begin{tabularx}{\textwidth}{|X|X|}
 \hline
\multicolumn{1}{|c|}{\textbf{Dystrophies musculaires}} & \multicolumn{1}{|c|}{\textbf{Myopathies métaboliques}} \\
\hline
\begin{itemize}
\item Faiblesse musculaire progressive
\item Perte de masse musculaire
\end{itemize} &
\begin{itemize}
\item Intolérance à l'exercice
\item Épisodes de fatigue
\item Myalgie
\end{itemize} \\
\hline

\multicolumn{1}{|c|}{\textbf{Myopathies mitochondriales}} & \multicolumn{1}{|c|}{\textbf{Myopathies congénitales}} \\
\hline
\begin{itemize}
\item Faiblesse musculaire
\item Intolérance à l'exercice
\item Implication cardiaque
\item Perte auditive
\item Crises d'épilepsie
\end{itemize} &
\begin{itemize}
\item Faiblesse musculaire à départ précoce
\item Progression lente de la maladie
\item Caractéristiques faciales dysmorphiques (visage allongé et palais voûté)
\end{itemize} \\
\hline
\end{tabularx}

\caption{Tableau des différentes atteintes neuromusculaires et leurs caractéristiques principales (\cite{lornage_identification_2019})}
\label{table:nmd}
\end{table}

\section{Les myopathies congénitales}
\subsubsection{Description générale}
Dans le cadre de cette thèse, nous nous intéressons en particulier aux myopathies congénitales. Les myopathies congénitales sont une famille de maladies génétiques rares qui affectent les muscles en général dès la naissance, mais les premiers signes symptômes peuvent n'apparaître qu'à l'adscolenccence ou l'âge adulte. Les \gls{mc} se différence des autres \gls{nmd} par la présence d'anomalie histolpathologiques caractéristiques dans la biopsie musculaire. Elles se caractérisent par une anomalie de la sturcture musculaire et de ses capacités contractiles.

Dans la population générales les myopathies congénitales ont une prévalence d'environ 1,5 pour 100 000, et 2,73 dans la population pédiatrique (\cite{huang_systematic_2021}), ce sont donc des maladies rares. Ces maladies sont dues à des mutations génétiques. \textit{Muscle Gene Table} (\href{https://www.musclegenetable.fr/}{https://www.musclegenetable.fr/}, \cite{benarroch_2023_2023}) référencie l'ensemble des gènes et des mutations connues à ce jour responsables de \gls{nmd}. Sur les 658 gènes référencés, il y a 47 gènes identifiés comme pouvant être responsables de \gls{mc} et cette liste évolue chaque année avec l'identification de nouveaux gènes, ce qui rend le diagnostic génétique complexe. 

\subsubsection{Approche curatives des myopathies congénitales}
Actuellement il n'existe aucun traitement curatif autorisé sur le marché contre les \gls{mc}. Des approches curatives sont en cours de developpement soit au stade pré-clinique soit au stade d'étude clinique (\cite{gineste_therapeutic_2023, guan_gene_2016}). Parmi les approches explorées ont retrouve des approches de thérapies génique, qui consiste à essayer de rétablir la fonction du gène défaillaint en introduisant une copie fonctionnelle du gène dans l'organisme. Cette approche est en cours d'étude pour les \gls{mc} liées à une défaillance du gène ACTA1, MTM1 et DNM2, BIN1, RYR1 et STIM1. Cependant les thérapies génique sont très couteuses et complexe à développer et mettre en place. Une seconde approche utilisée est l'utilisation de molécules pharmacologiques pour pallier à la fonction défaillante. Ces molécules sont en études pré-cliniques dans de nombreuses \gls{mc}, telles que celle causée par les gènes cités précédemment, ainsi que TPM2, TMP3, NEB et SEPN1. 

Cependant que ce soit pour la thérapie génique ou l'utilisation de molécules pharmacologique, les developpements restent difficile, notamment dans le cadre du passage de l'exprimentation animale à l'Homme. La start-up strasbourgeoise Dynacure, a developpé en 2019 une petite molécule, un oligonucléotide antisens, nommée DYN101 pour le trainement des myopathies congénitales centro-nucléaires liée aux gènes DNM2 et MTM1. Le developpement de ce traitement a du être arrêté suite à une toxicité hépatique trop importante chez l'Homme lors de la seconde phrase de l'essaie clinique. Cette toxicité n'a pas été observe dans le modèle animal de souris utilisé.

\subsection{Classification et prévalence}
Les myopathies sont un groupe de maladies hétérogènes dont le principale critère de classification est la biopsie musculaire. Les myopathies congénitales peuvent être classées en trois sous-types principaux: (i) \gls{com}, (ii) \gls{nm}, (iii) \gls{cnm} et deux sous-types supplémentaires:  (iv) \gls{cftd} et (v) myopathie de stockage de myosine (\textit{myosin storage myopathies}, MSM). (\cite{cassandrini_congenital_2017, claeys_congenital_2020, north_approach_2014}). Dans cette sous-section nous allons présenter les caractéristiques histologiques et tcliniques principales de chaque sous-types ainsi que les gènes impliqués. 

\subsubsection{Myopathies à cores (COM)}
Les myopathies à cores se peuvent se diviser en deux sous-groupes supplémentaires: les myoapthies à core unique et central (CCD) et les myopathies à multi minicores (MmD). Les myopathies à cores (COM) sont largment associées à différentes mutations dans le gène RYR1 avec des formes dominantes provoquant principalement le sous-tpye CCD et récessives provoquant principalement le sous-type MmD. Les myopathies à core de type MmD possèdent un terrain génétique plus variés, des mutataions dans SEPN1 et MYH7 pouvant aussi être reponsable de la maladie (\cite{cassandrini_congenital_2017} ). Ce sous-type de myopathies possède une prévalence d'environ 0,37 pour 100 000 (\cite{huang_systematic_2021}) en faisant la \gls{mc} la plus commune. 


Au niveau histopathologique, les \gls{com} se caractérisent principalement par la présence de cores (présentés en figure \ref{fig:cores}), des zones avec une activité oxydative très réduites. Ces zones sont soit unique, centrales et de grande taille, soit de petites tailles et multiple dans une même fibre musculaire. De plus on peut retrouve moins fréquement la présence de noyaux centralisés, de disproportion dans le ratio de fibre type 1 et type 2 et la présence d'alterration du tissus conjonctif (\cite{jungbluth_congenital_2018}). Au niveau clinique, en fonction du gène impliqué on va retrouvement très fréquement des atteintes des muscles extra-occulaires et des atteintes bulbaires (RYR1 recessif), ainsi que des atteintes respiratoires et cardiaques (SEPN1). Les  symptômes des \gls{com} peuvent apparaître a la naissance, durant l'enfance où à l'age adulte en fonction du gène impliqué.

\begin{figure}[!ht]
 \centering
 \includegraphics[width=0.5\textwidth]{figures/core.jpg}
 \caption[Biopsie de muscle de myopathies à cores]{Biopsie de muscle présentant des cores (zone à faible activité oxydative) caractéristique des myopathies à core à la coloration NADH (\cite{alan_pestronk_neuromuscular_2022})}
 \label{fig:cores}
\end{figure}

\subsubsection{Myopathies à némaline (NM)}
Les myopathies à némaline (NM) sont associées à plus d'une dizaine de gènes, donc le plus communément responsable est NEB (\cite{jungbluth_congenital_2018} ). Ce sous-type de myopathie présente une prévalence de 0,20 pour 100 000 (\cite{huang_systematic_2021}). Au niveau histopathologique, cette myopathie se caractérise principalement par la présence d'inclusions ressemblant à des batonnet (illustré en figure \ref{fig:rods}). Il peut aussi y avoir la présence des cores similaires aux \gls{com}. Au niveau clinique, les patients atteints de \gls{nm} présentent des problèmes respiratoires, des contractures et des atteinte bulbaires (\cite{jungbluth_congenital_2018} ) mais pas d'atteinte cardiaque. Les symtpomes apparaissent en général dès la naissance et des fois à l'enfance, mais pas à l'âge adulte (\cite{jungbluth_congenital_2018} ).


Les myopathies à némaline possède un niveau de sous-typage supplémentaires avec deux classes: (i) les myopathies à "cap", une forme très rare de myopathies à némaline avec 20 patients décrit entre 1981 et 2017 et (ii) les myopathies nommées "\textit{zebra body}" où le muscle possède une apparence zébrée, qui est une forme de \gls{mc} bégnine avec moins de 10 patients décrits (\cite{cassandrini_congenital_2017}).
\begin{figure}[!ht]
 \centering
 \includegraphics[width=0.5\textwidth]{figures/rods.jpg}
 \caption[Biopsie de muscle de myopathies à némaline]{Biopsie de muscle présentant des bâtonnets (inclusions sombres) caractéristique des myopathies à némaline à la coloration trichrome de Gomori (\cite{alan_pestronk_neuromuscular_2022})}
 \label{fig:rods}
\end{figure}



\subsubsection{Myopathies centro-nucléaires (CNM)}
Les myopathies centro-nucléaires  (CNM) aussi nommée myopathies myotubulaires, sont des myopathies congénitales plus rare avec unes prévalence d'environ 0.08 pour 100 000 (\cite{huang_systematic_2021}). Ce groupe des myopathies peut-être découpés en deux sous-groupes: les \gls{cnm} liées au chromosome X (XLMTM) due au gène MTM1 et non-lié à l'X (DNM2, RYR1, BIN1, TTN..., \cite{north_approach_2014}). Ce groupe de \gls{mc} se caractéristise au niveau histopathologique par la présence de noyaux plus gros que la normale, d'apparence vésiculaire en position centrale des fibres musculaire (présenté en figure \ref{fig:nuc}). Dans une fibre musculaire saine, les noyaux sont positionnés en périphérie des fibres. De plus on observe très fréquemment une augmentation de tissus adipeux et conjonctifs dans les fibres de muscles atteints de \gls{cnm} (\cite{jungbluth_congenital_2018}). Concernant les formes de \gls{cnm} liées à l'X, l'atteinte est présente dès la naissance, tandis que pour les formes non liées à l'X l'atteinte se déclare fréquemment à l'âge adulte, notamment dans le cas de DNM2 (\cite{jungbluth_congenital_2018}). Au niveau clinique on retrouve fréquement les atteintes des muscles extra-oculaires présent aussi dans les \gls{com}, les atteintes bulbaires, respiratoires, cardiaque et des contractures. 

\begin{figure}[!ht]
 \centering
 \includegraphics[width=0.5\textwidth]{figures/nuc.png}
 \caption[Biopsie de muscle de myopathie centronucléaire.]{Biopsie de muscle présentant des noyaux centralisés caractéristiques des myopathies centronucléaire à la coloration hématoxiline-éosine (\cite{alan_pestronk_neuromuscular_2022})}
 \label{fig:nuc}
\end{figure}

\subsubsection{Myopathies à disproportion congénitale des fibres (CFTD)}
Les myopathies à disproportion congénitale des fibres (CFTD) sont un sous-type moins bien défini et spécifique que les trois sous-typé présentés précédemment (\gls{nm}, \gls{cnm}, \gls{com}) qui ont une prévalence d'environ 0.23 pour 100 000. Ce sous-type est principalement défini par la seule présence d'une prédominance des fibres de type 1 et de leur atrophie d'environ 40\% par rapport aux fibres de type 2 (présenté en figure \ref{fig:cftd}, \cite{claeys_congenital_2020} ). Aucune autre anomalie de structure du muscle n'est présent (tel que les batonnets, les ocores ou les noyaux centralisés) dans ce sous-type de \gls{mc} (\cite{claeys_congenital_2020}). Au niveau clinique, les enfants atteints présentent une hypotonie et une faiblesse musculaire généralisée dès la naissance ou pendant les premières années. De plus ils présentent une atteinte importante au niveau des muscles du visage et des épaules ainsi que des problèmes respiratoires (\cite{claeys_congenital_2020} ). Ce sous-type de myopathie est lié à un dizaine de gènes tel que ACTA1, MYH7 et RYR1.
\begin{figure}[!ht]
 \centering
 \includegraphics[width=0.5\textwidth]{figures/cftd.jpg}
 \caption[Biopsie de muscle de myopathie à disproportion congénitale des fibres]{Biopsie de muscle présentant des une atrophie et une prédominance des fibres de type 1 caractéristiques des myopathies à disproportion congénitale des fibres à la coloration ATPase pH 4.3 (\cite{alan_pestronk_neuromuscular_2022})}
 \label{fig:cftd}
\end{figure}
\subsubsection{Myopathies de stockage de myosine (MSM)}
Enfin les myopathies de stockage de myosine (MSM), anciennent nomées "\textit{hyaline body myopathy}" se caractérisent principalement  par la présence de \textit{"hyaline body"} , des régions d'apparence granulaires et basophiles à la coloration hématoxyline-éosine (voir figure \ref{fig:hyaline}, \cite{claeys_congenital_2020, victor_dubowitz_muscle_2020}).  Sur le plan clinique, les patients présentent des problèmes cardiaque, une perte de force distale (main, poignets), des pieds tombants et une pseudo-hypertrophies des mollets (\cite{cassandrini_congenital_2017}). Seul un seul gène est identifié pour l'instant comme pouvant causer une MSM, il s'agit de MYH7, pouvant aussi causer des \gls{cftd} et des\gls{com}.
\begin{figure}[!ht]
 \centering
 \includegraphics[width=0.5\textwidth]{figures/hyalin.jpg}
 \caption[Biopsie de muscle des \textit{hyaline bodies}]{Biopsie de muscle présentant des \textit{hyaline bodies} indiqué par les flèches noires caractéristiques des myopathies à stockage de myosine (\cite{victor_dubowitz_muscle_2020})}
 \label{fig:hyaline}
\end{figure}

La diversité des sous-types de myopathies congénitales tant par leur nombre que par leur hétérogénéité intra sous-type et les recouvrement sur le plan clinique histologique et génétique entre les sous-type rendent leur diagnostic difficile. Dans la prochiane section nous verrons quels sont les stratégies de diagnostic utilisées actuellement et qu'elles sont les données générées dans le cadre de ce diagnostic.

\section{Le défi du diagnostic des myopathies congénitales}
Les myopathies congénitales présentent un triple diversité importante sur le plan clinique histhopathologique et génétique. Un sous-type de myopathies peut être causé par une mutation dans différents gène et une mutation dans un gène spécifique peut causer plusieurs sous-type de myopathies, rendant le diagnostic des patients complexe. De plus une même mutation peut mener à des manifestations differents d'un point de vue pathologique (\cite{north_approach_2014}). 

Deux approches la classification et le diagnostic des \gls{mc} existent (\cite{north_approach_2014}). La première nommée \textit{genotype-up}, consiste à partir de chaque gène responsable de \gls{mc} et de lister l'ensemble des phénotype (cliniques ou histopathologiques) suggestif d'une mutation dans ces gènes. Le seconde approchée nommée \textit{"phenotype down"} consiste, à partir des observations cliniques et histopathologiques, à lister les sous-type de myopathies associés et les potentiels gène candidats pouvant causer ces phénotypes.

Le processus de diagnostic des \gls{mc} se base sur trois niveaux d'informations (clinique, histopathologiques et génétiques), générant ainsi trois type des données (imagerie, textuelles et de séquences) que nous allons présenter ici pour clore ce chapitre.

\subsection{Séquençage NGS et données de séquences génétiques}
Historiquement les techniques de séquençage Sanger ont été utilisée pendant des décénies pour identifier les mutations responsables de \gls{mc}. Cependant, les avancées en termes de séquençages grace aux \gls{ngs}, permettent aujourd'hui de séquencer beaucoup plus rapidement et à bas cout, l'ensemble d'un panel de gène d'intérêt pour trouver les mutations. présentes dans les génome du patient. Cependant cette approches permettant d'évaluer un grand nombre de gènes en même temps ne résout pas tout les challenges qui résident dans l'identification des mutations causant les myopathies congénitales. En effet, même avec les techniques de \gls{ngs}, 50\% des patients atteints de myopathies congénitales n'ont pas de diagnostic génétique à ce jour montrant qu'il reste un travaille à faire sur l'identifications de gènes et de mutations inconnues responsables de \gls{mc}. Par exemple, le projet MYOCAPTURE porté par le groupe de recherche de Jocelyn Laporte, est issue d'un consortium de sept équipes de recherche avec pour objectifs le séquences de 1000 exomes de familles atteintes de \gls{nmd} afin d'identifier de nouveaux gènes et mutations inconnues.

De plus, comme évoqué précédemment et présenté dans le tableau \ref{tab:gene_myo} (\cite{cassandrini_congenital_2017}), un gène peut être reponsable de différent sous-type de \gls{mc}. Les données génétiques ne sont donc pas suffisantes seules pour poser un diagnostic fiable. Le traitements de ces données de types séquence génétique requierent le developpement d'outils qui permettent de traiter les résultats de séquençage pour en extraire les mutations, de qualitifer la pathogénicité des mutations et de filtrer les mutations pertinentes dans le cadre du diagnostic des \gls{mc}. 

\begin{table}[!ht]
\centering
\includegraphics[width=1\textwidth]{figures/gene_tab.png}
\caption{Tableau des principaux gènes responsables de myopathies congénitales et des sous-types associées (\cite{cassandrini_congenital_2017})}
\label{tab:gene_myo}
\end{table}
\subsection{Histopathologie et données d'imagerie }
L'histopathologie est le premier critère de classification des sous-types de \gls{mc}. Comme vu précédemment dans la description de sous-types de \gls{mc}, plusieurs marqueurs pathologiques sont à évaluer pour pouvoir orienter le diagnostic. Pour observer ces marqueurs, l'examen classique est une biopsie musculaire (pouvant porter sur divers muscle du patient comme le quadriceps, le deltoïde, la vaste externe ou autre) puis une observation de la biopsie au microscope sous différentes colorations, permettant chacune d'apporter une information sur la structure du muscle. En routine cinq colorations pricipales sont réalisée dans le cadre du diagnostic à partir de biopsie musculaire: (i) la coloration \gls{he} qui révèle la taille des fibres et la positions des noyaux, (ii) la coloration \gls{tg} qui révèle la charpente protéique des fibres et qui permet d'observer des aggrégats proétiques, (iii) la coloration ATPase qui permet de différencier les fibres de type 1 et type 2. et (iv) et (v) les colorations \gls{sdh} et NADH qui révèlent l'acitivité oxydative des fibres et la position des mitochondries. De plus des colorations supplémentaire peuvent être réalisée telles que de l'immuno-histochimie et les colorations phosphorylases, PAS (glycogène) et Soudan (lipides). Pour finir, dans les cas complexes il peut être nécessaire d'avoir recours à la microscopie électronique pour observer avec une forte résolutions la structure précise des fibres musculaires.

Ainsi l'examen de la biopsie musculaire génère plusieurs images (1 par coloration) de grande tailles (de l'ordre du milier de fibre par biopsie) qui sont évaluée manuellement pour identifier les marqueurs pathologiques. Cependant comme pour les données génétiques, il est difficile de poser une diagnostic sur le seule base des observations phénotypiques. Comme présenté dans le tableau \ref{tab:histopath} (\cite{jungbluth_congenital_2018}), les caractéristiques typique d'un sous-type de myopathies, comme les cores dans les myopathies à cores, peuvent être présent dans dans d'autre sous-type tel que les myopathies à némaline par exemple. Des travaux de recherches sont encore nécessaire pour trouver des critères réellement discriminants entre les sous-type de myopathies sur le plan histopathologique, tel que le projet de l'Atlas du Muscle porté par Bruno Cadot et Norma Romero (\cite{cadot_atlas_2022}). Cet atlas est une banque de données en ligne d'image d'histopathologie de biopsie de muscle dans différentes colorations pour un large panel de \gls{nmd}. A ce jour ce sont plus de 5000 photos de biopsies musculaires qui sont disponibles avec le gène et le diagnostic de \gls{nmd} associé. 

\begin{table}[!ht]
\begin{tabularx}{\textwidth}{|p{1.8cm}|X|X|X|X|X|X|X|X|X|}
 \hline
\textbf{Observation} & \textbf{RYR1 AD} & \textbf{RYR1 AR} & \textbf{SEPN1} & \textbf{TTN} & \textbf{MTM1} & \textbf{DNM2} & \textbf{NEB} & \textbf{ACTA1} & \textbf{KLHL 40} \\
\hline
Cores & +++ & +++ & +++ & ++ & - & + & + & + & - \\
\hline
Noyau central & ++ & ++ & - & +++ & +++ & +++ & - & - & - \\
\hline
Bâtonnets  & + & + & - & + & - & - & +++ & +++ & +++ \\
\hline
CFTD & + & +++ & + & + & + & - & - & + & - \\
\hline
Infiltration tissus conjonctif & ++ & ++ & ++ & +++ & - & + & - & - & - \\
\hline
\end{tabularx}
\caption{Tableau des fréquences des principaux marqueurs pathologiques observables sur la biopsie musculaire en fonction du gène impliqué (\cite{jungbluth_congenital_2018}). }
\label{tab:histopath}
\end{table}
L'évaluation des données d'imageries histopathologique est un processus long en raison du nombre de fibres musculaire et du nombre de colorations. En général il est réalisé de façon qualtitative, sans comptage de fibres et des éléments pathologiques. C'est pourquoi il serait intéressant de développer des outils capables de réaliser cette évaluation automatiquement de façon quantitative, grace à des méthodes IA. En règle générale, cette évaluation donne lieu à la rédaction d'un rapport textuel rendant compte des observations réalisée dans la biopsie musculaire.

\subsection{Données cliniques et histopathologiques textuelles }
Les observations cliniques de patients sont aussi un niveau d'information utile et necessaire au diagnostic des myopathies congénitales. Même si certains phénotypes sont très commun et peu informatifs (hypotonie, faiblesse musculaire générales) d'autre plus spécifique peuvent permettre d'éliminer certains gènes. Le tableau \ref{tab:clinic} (\cite{jungbluth_congenital_2018}) présente la fréquence des principales observation cliniques en fonction du gène impliqué. On oberve par exemple que la présence d'une atteinte cardiaque est très fréquente lors d'une mutation du gène TTN, peu fréquente dans les  gènes ACT1, SEPN1 et RYR1 récessif, et totalement absente pour RYR1 dominant, MTM1, DNM2, NEB et KLHL40. Ainsi il est possible d'orienter les tests génétiques et diagnostic grace à cette information. Cependant on observe aussi que ces phénotypes peuvent apparaitre pour de multiples gènes, rendant impossible le diagnostic sur la seule base de critères cliniques.

L'observation des phénotypes cliniques, tout comme l'observations des marqueurs histopathologique, mène à la rédaction d'un compte-rendus clinique textuel qui liste les observations réalisées. Afin de pouvoir traiter ces données pour en extraire de nouveaux critères de classifications, il est necessaire de developper des outils adaptés à la compréhension de texte libre en langage naturel, notamment grace aux avancées en \gls{ia}. 

\begin{table}[!ht]
\begin{tabularx}{\textwidth}{|p{1.8cm}|X|X|X|X|X|X|X|X|X|}
 \hline
\textbf{Observation} & \textbf{RYR1 AD} & \textbf{RYR1 AR} & \textbf{SEPN1} & \textbf{TTN} & \textbf{MTM1} & \textbf{DNM2} & \textbf{NEB} & \textbf{ACTA1} & \textbf{KLHL 40} \\
\hline
Muscle extra-occulaire & + & +++ & - & - & +++ & +++ & - & - & ++ \\
\hline
Atteinte bulbaire & + & +++ & ++ & ++ & +++ & ++ & ++ & ++ & +++ \\
\hline
Atteinte distale  & - & + & - & ++ & + & +++ & ++ & + & + \\
\hline
Atteinte respiratoire & + & ++ & +++ & ++ & +++ & + & ++ & ++ & +++ \\
\hline
Atteinte cardiaque & - & + & + & +++ & - & - & - & + & - \\
\hline
Contractures & + & + & + & +++ & +++ & ++ & ++ & ++ & +++ \\
\hline
\end{tabularx}
\caption{Tableau des fréquences des principales observation cliniques en fonction du gène impliqué (\cite{jungbluth_congenital_2018}). }
\label{tab:clinic}
\end{table}

\section{Conclusion}
Cette description des myopathies congénitales dresse un portrait complexe du processus de diagnostic, qui requiert l'intégration de trois niveaux d'informations de modalité différentes (textes, images, séquences). L'ensemble de ces données et leur complexité mettent en avant le besoin d'outils adapté pour aider à leur exploration automatique. Le but de cette thèse est de developper des méthodes permettant d'exploiter les données d'imagerie (biopsie musculaires) et les données textuelles (cliniques et histhopathologiques) pour permettre de mieux caractériser les \gls{mc} et d'aider à leur classification. 