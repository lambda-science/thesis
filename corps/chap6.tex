\chapter{NLMyo : Traitement de rapport textuel par LLMs}

\section{Contexte}
Les récentes avancée dans le \gls{nlp} ont permis de révolutionner la manière de traiter et d'exploitée les données sous forme de texte libre. \gls{impatient} utilise un système à base de règles et de correspondance exacte des mots pour analyser les compte-rendus de biopsie. Cette approche présente des limites importantes en terme de flexibilité et de précision et demande l'établissement d'un vocabulaire normalisé au pré-alable. 
Récemment, la mise à disposition de \gls{llms} performants et accessible ouvre la porte à la création d'outil plus performant et flexibles pour le traitement de ces comptes-rendus. Ces système basé sur une approche sémantique et multi-langues éliminent la nécessitée de définir un vocabulaire standard. Ansi nous avons développé \gls{nlmyo} (fig \ref{fig:nlmyo_logo}), une boite à outils basée sur les \gls{llms} mettant à disposition quatre outils généralistes pour le traitement de compte-rendus médicaux: outil d'anonymisation, d'extractions de d'information, de classification automatique et de création moteur de recherche. L'ensemble de ces outils et des modèles utilisés est représenté dans la figure \ref{fig:nlmyo_struct}.
\begin{figure}[htbp]
  \centering
  \includegraphics[width=0.5\textwidth]{figures/nlmyo_banner.png}
  \caption[Logo NLMyo]{Logo de NLMyo}
  \label{fig:nlmyo_logo}
\end{figure}
\begin{figure}[htbp]
  \centering
  \includegraphics[width=1\textwidth]{figures/nlmyo_struct.png}
  \caption[Structure de NLMyo]{Structure de NLMyo}
  \label{fig:nlmyo_struct}
\end{figure}
\section{Anonymizer: un outil d'anonymisation}
Le premier outil de \gls{nlmyo} est \textit{Anonymizer}, un outil permettant de supprimer automatiquement les informations identifiantes des compte-rendus médicaux. Dans les compte-rendus de biopsie de l'institut de myologie que nous traitons deux données identifiantes et personnelles sont présentes et doivent être retirée: le nom du patient (et du personnel médical) ainsi que la date de naissance du patient. Non seulement ces informations ne sont pas utiles pour les analyses subséquentes mais de plus, par respect pour le \gls{rgpd} et la vie privée, ces informations ne doivent être accessibles qu'aux professionnels médicaux en charge du patient. L'anonymisation des rapports est donc une étape essentielle avant le transfert de ces données, leur numérisation et analyse.

Afin de traiter un grand volume de rapports et d'éliminer le travail manuel nécessaire, nous avons tenter d'automatiser la tache d'anonymisation via deux approches: un approche traditionnelle par\gls{regex} et une approche novatrice par\gls{llms}.
\subsection{Anonymisation par RegEx}
En première intention, nous avec développé une méthode basée sur les \gls{regex} pour leur simplicité à mettre en place et leur rapidité d'exécution (coûts en puissance de calcul faible). Le tableau \ref{tab:regex} liste les \gls{regex} utilisée pour capturer les informations de noms et de dates. Les comptes-rendus de biopsie sont semi-structurés, pour la pluspart le nom du patient est facilement identifiant car il est précédé par le préfixe: "Nom: ". Ceci est facilement capturable par le première \gls{regex} listée dans le tableau. Ensuite, comme les noms de famille sont souvent en majuscule et les prénom commencent souvent pas une majuscule, nous avons developpé deux autres \gls{regex} pour capturer les couples de mots dont un est en majuscule et le second commence par une majuscule (ou inversement) (ligne 2 et 3 du tableau). 
Ensuite, une troisième \gls{regex} a été ajouté pour détecter les dates au format JJ-MM-AAAA ou JJ.MM.AAAA (ligne 3 du tableau). Finalement une dernière \gls{regex} est utilisée pour essayer de trouver le numéro de biopsie dans le document afin de renommer le fichier avec un nom unique et anonyme (ligne 4 du tableau). 
\begin{table}[ht]
\centering
\caption{Expressions régulières pour extraire les noms et les dates}
\label{tab:regex}
\begin{tabular}{|l|l|l|}
\hline
\textbf{Expression régulière} & \textbf{Syntaxe} & \textbf{Exemples d'utilisation} \\ \hline
Nom patient & Nom.*: *([A-Za-zÀ-ÿ- ]+) & Nom : John Doe \\ \hline
Nom patient 2& (([A-Z][a-zÀ-ÿ-]\{3,\} ?)+ ([A-Z-]\{3,\} ?)+) & DOE John \\ \hline
Nom patient 3 & (([A-Z-]\{3,\} ?)+ ([A-Z][a-zÀ-ÿ-]\{3,\} ?)+) & John Joe DOE \\ \hline
Date & ([(.]?[0-9]\{1,2\}[./][0-9]\{1,2\}[./][0-9]\{1,4\}[().]?) & 01/01/2023, 02.02.2023 \\ \hline
N° Biopsie & ([0-9]\{3,8\}[-/]?[0-9]\{0,3\}) & 9876-54, 1234/56 \\ \hline
\end{tabular}
\end{table}
\begin{figure}[htbp]
  \centering
  \includegraphics[width=1\textwidth]{figures/regex.png}
  \caption[Exemple anonymisation RegEx]{Exemple d'anonymisation d'un rapport de biopsie factice avec la méthode RegEx. En haut le rapport à l'état brut, en bas le rapport censuré.}
  \label{fig:regex}
\end{figure}
La figure \ref{fig:regex}  présente les résultats de la technique d'anonymisation par \gls{regex} sur l'entête d'un rapport factice de patient mais avec une structure similaire aux rapport de l'institut de myologie de Paris. Les noms et les dates ont été censurée correctement et la méthode a produit de bon résultats. Cependant cette méthode est insuffisante car elle est peu sensible et spécifique. Le tableau \ref{tab:regex_fail} liste trois exemples de cas où cette méthode ne fonctionne pas et produits des erreurs. Il est possible de corriger ces erreurs en augmentant la somme de \gls{regex} utilisée, mais augmenter le nom de \gls{regex} augmente aussi potentiellement le nombre de faux positifs. De plus il n'est pas toujours possible de construire une \gls{regex} adaptée pour extraire une information précise. Nous avons alors exploré la capacité des \gls{llms} pour la recherche et l'extraction de ces informations de manière plus robuste et flexible que la méthode \gls{regex}.
\begin{table}[ht]
\centering
\caption{Exemples de faux positif ou faut négatif de la méthode RegEx}
\label{tab:regex_fail}
\begin{tabularx}{\textwidth}{|X|X|X|X|}
\hline
\textbf{Texte} & \textbf{RegEx déclenchée} & \textbf{Type} & \textbf{Commentaire} \\ \hline
\textit{PAS Staining} & Nom patient 3 & Faux Positif & Nom de coloration dont la notation est confondue avec le motif "NOM Prénom" \\ \hline
\textit{Louis C. Dupont} & N/A & Faux Négatif & La présence du "C." au centre ne permet pas aux RegEx de nom de se déclencher \\ \hline
\textit{12 mars 2001} & N/A & Faux Négatif & La notation de date avec un mois en lettres ne permet par à la RegEx de date de se déclencher \\ \hline
\textit{1996-04} & N° Biopsie & Faux Positif & La notation de date AAAA-MM déclenche la \gls{regex} de numéro de biopsie. \\ \hline
\end{tabularx}
\end{table}

\subsection{Anonymisation par LLMs}
Les \gls{llms} sont basé sur la compréhension du sens sémantique du texte là où les \gls{regex} sont basée sur le principe de motif de caractères et donc sur la structure du document. L'idée de l'anonymisation par \gls{llms} est d'utiliser un modèle génératif auquel on fournis une instruction et le texte à anonymiser. L'instruction fournie liste les informations à extraire du texte et spécifie le format de sortie. Pour intégrer ces modèles génératif à notre programme, nous voulons récupérer les informations extraire dans un format exploitable informatiquement, c'est à dire au format JSON Le format JSON réprésente un dictionnaire qui est utilisable ensuite par l'application pour censurer les PDF à partir des informations extraites.
\subsection{Instruction personnalisée et \textit{one-shot learning}}
Nous avons construit une instruction personnalisée en 3 parties qui intègre une méthode de \textit{one-shot learning}. Les trois parties sont: (i) la description de la tâches à effectuer, (ii) un exemple de réalisation de la tâche \textit{(one-shot learning)}, (iii) le texte d'intérêt à anonymiser.
Voici un exemple d'instruction que nous utilisons pour réaliser l'extraction des noms et des dates dans les rapports:
\begin{quote}
Tu es un assistant qui extrait des informations d'un texte libre. Le format de ta réponse doit être un format JSON valide qui respecte le nom des clés founie. Si une valeur est manquante, indique simplement N/A, n'essaie pas d'inventer.Voici la liste des informations à récupérer, les clés JSON sont indiquées entre parenthèses : nom complet (name), dates (date).

ENTRÉE :

Kendrick Lamar et Jane Clinton sont asymptomatiques. Date de naissance : 16 février 1991, numéro de biopsie : 666-77. Ce rapport a été expédié le 01.04.1991.

SORTIE :

\{"name" :["Kendrick Lamar", "Jane Clinton"], "date" : ["16 février 1991", "01.04.1991"]\}

ENTRÉE :

<texte à analyser>

SORTIE :
\end{quote}
La partie précédant le mot clé "ENTREE" correspond à l'instruction décrivant précisément la tâche à réaliser au modèle (liste des informations à extraire, format de sortie et comportement attendu). Le premier couple "ENTREE" et "SORTIE" correspond à un exemple de réalisation de la tâche, ce qui permet de spécifier le schéma JSON attendu au modèle. Puis le second couple "ENTREE", "SORTIE" correspond à l'endroit où l'on injecte notre texte d'intéret à analyser et spécifie au modèle que l'on attend maintenant une sortie textuel au format JSON pour l'entrée précédante.
\subsection{Exemple et comparaison à la méthode RegEx}
\begin{figure}[htbp]
  \centering
  \includegraphics[width=0.8\textwidth]{figures/llms_anonym.png}
  \caption[Exemple anonymisation LLMs]{Exemple d'anonymisation d'un courrier médical factice avec la méthode par LLMs. (A) Texte brute, (B) Anonymisation par RegEx, (C) JSON Brut générée par le \gls{llms} GPT-3.5, (D) Anonymisation à partir du JSON généré par le LLM}
  \label{fig:llms_anonym}
\end{figure}
Dans cet exemple (figure \ref{fig:llms_anonym}), nous avons construit un début de courrier médical factice faisant figurer des informations personnelles qui se sont pas détectable par notre méthode \gls{regex}. Les prénoms "Jérémy" et "Clara", ainsi que les dates "2 mai 2023" et "08 12 2003" n'ont pas été détectée par la méthodes \gls{regex} et censurés. On observe par contre que la méthode par \gls{llms} (C et D) a été capable d'identifier ces informations et de les extraire du texte. Cet exemple montre que les \gls{llms} peuvent être la base d'un système d'anonymisation plus flexible et moins dépendant de la structure du document.

\section{MyoExtract: un outil d'extraction d'information}
A partir des résultats encourageant obtenu la technique d'anonymisation par LLMs pour l'extraction d'information, nous avons voulu étendre le champs des informations extraire de manière automatique à partir de texte libre. Nous avons utilisé la même stratégie d'extraction d'information, c'est à dire l'utilisation de \gls{llms} génératifs mais avec une instruction légèrement différente. Cette fois ci nous avons ajouter une liste plus importante d'informations à extraire non pas dans le but d'anonymiser le document, mais d'en extraire les méta-données. Par exemple nous avons chercher à extraire: les noms, date de naissance, date d'envoi de la biopsie, numéro de biopsie, muscle prélevé, diagnostic final. De plus nous avons chercher à savoir s'il était possible d'extraire si il est mention d'anomalie pour certaines colorations tel que la coloration PAS, Soudan, COX, ATP et Phosphorilase. Cette extraction d'information pourrait permettre d'annoter automatiquement les rapports avec une liste d'anomalie détectée pour chaque coloration. De même que précédemment, nous avons construit une instruction personnalisée en 3 parties (description, exemple, texte à analyser.)
Voici un exemple d'instruction que nous utilisons pour réaliser l'extraction des méta-données et d'anomalie génération des colorations à partir de rapports:
\begin{quote}
Tu es un assistant qui extrait des informations d'un texte libre. Le format de ta réponse doit être un format JSON valide qui respecte le nom des clés founie. Si une valeur est manquante, indique simplement N/A, n'essaie pas d'inventer. Formate les dates sous la forme DD-MM-YYYY et convertis les âges en années (0 si inférieur à 1 an). Voici la liste des informations à récupérer, les clés JSON sont indiquées entre parenthèses : nom complet (name), âge (age), date de naissance (birth), date de la biopsie (biodate), date d'envoi de la biopsie (sending), muscle (muscle), numéro de la biopsie (bionumber), diagnostic (diag), présence d'une anomalie dans la coloration du PAS (PAS), présence d'une anomalie dans la coloration Soudan (Soudan), présence d'une anomalie dans la coloration COX (COX), présence d'une anomalie dans la coloration ATP (ATP), présence d'une anomalie dans la coloration Phosphorylase (phospho)

ENTRÉE:

Kendrick Lamar et Jane Clinton sont asymptomatique. Date de naissance: 16 février 1991, numéro de biopsie: 666-77. Anomalie forte à la coloration PAS mais pas d'anomalie à la coloration lipide soudan. Le tableau est révélateur d'une myopathie à némaline

SORTIE:

\{"name":["Kendrick Lamar", "Jane Clinton"], "age":"N/A", "birth": "16-02-1991", "biodate": "N/A"", "sending": "N/A"", "muscle": "N/A"", "bionumber": "666-77", "diag": "myopathie à némaline", "PAS": "yes", "Soudan": "no", "COX": "N/A", "ATP": "N/A", "phospho": "N/A"\}

ENTRÉE :

<texte à analyser>

SORTIE :
\end{quote}
\subsection{Exemple d'extraction d'information}
Pour cet exemple d'utilisation nous avons généré un rapport factice de patient avec une structure similaire aux rapport de l'institut de myologie de Paris qui reprend des observations typique trouvée des les rapports réel de biopsie. Ce rapport est disponible en figure \ref{fig:factice_report}. 
\begin{figure}[htbp]
  \centering
  \includegraphics[width=1\textwidth]{figures/pdf_biopsie.png}
  \caption[Rapport de biopsie factice]{Exemple rapport de biopsie factice}
  \label{fig:factice_report}
\end{figure}

Les  résultats de l'extraction d'information présentés dans le tableau \ref{tab:json_data} montrent que le modèle d'\textit{OpenAI} GPT-3.5 est capable d'extraire l'ensemble des informations demandée de manière satisfaisantes tout en étant capable de détecter l'absence de certaines informations. Concernant le modèle \textit{Vicuna7B}, qui a l'avantage d'être auto-hébergé et donc d'être utilisable pour des données sensibles, les performances sont moindres. En effet, six données sur treize (diagnostic et anomalies) demandées sont tout simplement manquante dans le JSON de sortie. Cependant pour les sept données (nom, age, dates, muscle et numéro de biopsie) présente, les résultats sont satisfaisant, le modèle a extrait les bonne informations.
Il est important de noter qu'en terme de ressources de calcul et de temps de d'inférence, \textit{GPT-3.5} possède un avantage non-négligeable car ce modèle n'est accessible que par une \gls{api}. Les coûts de calcul pour l'application sont donc nul et la requête ne prend que quelques secondes à être réalisée. Pour \textit{Vicuna}, le modèle auto-hébergé, chaque requête requiert une quantité importante de ressources de calcul et monopolise ces ressources pour un temps important (environ 1min30 par document). Ces coûts en ressources et temps de calcul couplé à une précision plus faible, rendent difficile l'exploitation de modèle \gls{llms} génératif auto-hébergé pour la tâche d'extraction d'informations à travers une interface en ligne.
\begin{table}[ht]
\centering
\caption{Résultat de MyoExtract pour GPT-3.5 and Vicuna7B}
\label{tab:json_data}
\begin{tabularx}{\textwidth}{|X|X|X|}
\hline
\textbf{Information} & \textbf{GPT-3.5} & \textbf{Vicuna7B} \\ \hline
Nom & DOE John & DOE John \\ \hline
Âge & 7 & 7 years old, born on 20/11/2015 \\ \hline
Date de naissance & 20-11-2015 & 20-11-2015 \\ \hline
Date de biopsie & 20-11-2021 & 20-11-2021 \\ \hline
Date d'envoie & 21-11-2021 & 21-11-2021 \\ \hline
Muscle & Quadriceps & quadriceps \\ \hline
N° Biopsie & 777-07 & 777-07 \\ \hline
Diagnostic & congenital myopathy with a nemaline subtype with strong fiber type disproportion and smaller fiber type & \textit{missing} \\ \hline
Anomalie PAS & N/A & \textit{missing} \\ \hline
Anomalie Soudan & N/A & \textit{missing} \\ \hline
Anomalie COX & N/A & \textit{missing} \\ \hline
Anomalie ATP & abnormal fiber differentiation and no fiber bundling & \textit{missing} \\ \hline
Anomalie Phospho. & N/A & \textit{missing} \\ \hline
\end{tabularx}
\end{table}

\section{MyoClassify: un outil d'aide au diagnostic}
\subsection{Embedding et classificaiton}
\subsection{Résultats et performances}
\section{MyoSearch: un moteur de recherche de patients}
\subsection{Ingestion des données}
\subsection{Retrieval des données}
\section{Déploiement de la plateforme}
Developpé de façon \textit{open-source}, le code source de \gls{nlmyo} est disponible sur GitHub à l'adresse: \href{https://github.com/lambda-science/NLMyo}{https://github.com/lambda-science/NLMyo} et une version de démonstration en ligne est déployée grace à \textit{Streamlit} à l'adresse \href{https://lbgi.fr/NLMyo/}{https://lbgi.fr/NLMyo/}
\section{Perspectives futures de développement}