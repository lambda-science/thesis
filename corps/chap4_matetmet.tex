\chapter{Outils informatiques et données utilisées}
\section{Données biomédicales de myopathies congénitales}
\subsection{Rapport histologiques de patient de l'Institut de Myologie de Paris}
\subsection{Images de biopsie musculaire de souris}
\section{Ontologies biologiques}
\subsection{Ontologie des phénotypes: HPO}
\subsection{Ontologie de maladies: ORDO par Orphanet}
\subsection{Nomenclature génétique: HUGO et HGVS}
\section{Développement de modèle de ML traditionnels et xAI}
\subsection{Scikit-Learn}
\subsection{Cross-validation et évaluation des performances}
\subsection{Rechercher d'hyper-paramètres}
\subsection{Système de classeurs (LCS)}
\subsection{Pipeline Streamline}
\subsection{Métriques de performances}
\section{Techniques d'analyse d'images et réseaux de neurones}
\subsection{Méthode d'analyse d'image traditionnelles avec scikit-image}
\subsection{Modèle pré-entrainé Cellpose et Stardist}
\subsection{Développement de réseau de neurones profond de type ResNet avec Keras Tensorflow}
entrainement et inférence RTX 2070Ti 12gb
\section{Développement d'outils basés sur modèles linguistiques de grande taille}
\subsection{Reconnaissance de text avec Tesseract}
\subsection{Paramètre de LLMs: context window et température}
\subsection{Interaction avec les modèles linguistique de grande taille avec LangChain}
\subsubsection{Langchain}
\subsubsection{Modèle génératif: OpenAI et Vicuna-7B}
ChatGPT = gpt-3.5-turbo
vicuna car contexte windows et 7B car petit
\subsubsection{Modèle d'\textit{embedding}: OpenAI et Instructor}
text-embedding-ada-002
\subsubsection{\textit{Zero-shot, One-shot, Few-shot learning}}
\subsection{Base de données de vecteurs avec ChromaDB}
OCR tesseract
\section{Developpment Web}
\subsection{Back-end avec Flask}
\subsection{Front-end avec Jquery et Boostrap}
\subsection{Base de données avec SQLite}
\subsection{Solution clé en main avec Streamlit}
\section{Reproductibilité des analyse par intelligence artificielle}
\subsection{Suivi d'expérience avec \textit{Weight and Biais}}
\subsection{Développement open-source et versionnage avec GitHub}
\subsection{Développement IA open-source et versionnage avec HuggingFace}
\subsection{Environnement de développement reproductible}
\subsubsection{Environnement virtuel Python}
\subsubsection{Docker}

Zenodo mega-repo