\chapter{Resumé de Thèse}
\paragraph{\textbf{L’avènement des \textit{Big Data} dans le domaine biomédical}}\mbox{}\\

Les données biomédicales regroupent l’ensemble des données relatives à la santé humaine telles que les données génétiques, les dossiers cliniques, les images médicales ou les rapports d’analyses. Ces données sont utilisées pour améliorer la compréhension des maladies et pour poser des diagnostics pour les patients. Les améliorations des techniques d’analyse ont permis de faciliter l’acquisition de données biomédicales en grande quantité. Par exemple, les technologies de séquençage de nouvelle génération ont permis de séquencer rapidement et à moindre coût des génomes entiers et les avancées en imagerie médicale ont permis d'acquérir des quantités importantes d’images de grande résolution des tissus. De plus, l’archivage des comptes rendus médicaux représente une masse de connaissance importante encore non exploitée de nos jours.

Le terme "\textit{Big Data}" fait référence à des ensembles de données extrêmement vastes, complexes et hétérogènes qui dépassent la capacité des outils de traitement de données traditionnels. Les big data sont caractérisées par les "5 V" : le volume, la variété, la vélocité, la véracité et la valeur. Le volume fait référence à la quantité massive de données, la variété à la diversité des sources et des types de données, et la vélocité à la rapidité avec laquelle les données sont générées et doivent être traitées. La véracité et la valeur font référence à la qualité et fiabilité des données ainsi que leur utilité pour aider à la prise de décision.

De fait, les données biomédicales sont donc un exemple concret de \textit{Big-Data} : elles sont multimodales (ex. : texte, images, génomes …) donc variées, elles sont massives (ex. : taille des données de séquençage, imagerie à l’échelle du giga pixel, milliers de comptes rendus en texte libre…) et ont une forte vélocité grâce à l’amélioration des techniques d’acquisition d’images, d’analyse et de séquençage. De plus ce sont des données présentant une véracité et une valeur importante, car elles sont expertisées et possèdent une valeur importante dans la cadre de l’aide au diagnostic.

L’augmentation des volumes, modalités et complexité des données biomédicales rend impossible leur traitement manuel et requiert donc l’utilisation d’outils adaptés pour traiter et extraire des informations pertinentes dans le cadre de la recherche médicale, comme les méthodes basées sur l’\gls{ia}


\paragraph{\textbf{Nouvelle génération d’\gls{ia} pour le traitement du Big Data grâce aux réseaux de neurones}}\mbox{}\\

L’\gls{ia} représente un ensemble de techniques permettant de créer des programmes simulant l’intelligence humaine. Une sous-branche de l’\gls{ia} nommée \gls{ml} regroupe l’ensemble des algorithmes permettant à un programme informatique d’accomplir une tâche en apprenant d’un jeu de données. Le plus souvent il s’agit de tache de classification à apprentissage supervisée, c’est-à-dire la prédiction d’une catégorie (par exemple un diagnostic) en apprenant d’un jeu de données labellisé.

Le \gls{ml} possède un intérêt particulier dans le domaine biomédical, car il peut permettre sur le plan du diagnostic d’aider à la décision et donc de permettre un gain de temps et de précision et sur le plan de la recherche, d’aiguiller vers la découverte de nouveaux critères de diagnostic pour les maladies difficiles à diagnostiquer. Cependant, les techniques de \gls{ml} traditionnelles ne peuvent qu’apprendre que de données sous forme de tableaux, où chaque ligne représente un point de donnée et chaque colonne un descripteur de ce point de donnée. Cela ferme la porte à l’exploitation de données plus complexes, comme les images ou les textes libres et demande de réaliser un travail d’annotation et de curation important.

Cette dernière décennie, la popularisation des \gls{dnn}, une famille d’algorithmes de ML reposant sur le concept bio-inspiré de neurones, a permis l’exploitation de données complexes sans avoir besoin de connaissance à priori sur les données, c’est-à-dire sans devoir définir des descripteurs pertinents manuellement comme dans le \gls{ml} traditionnel. Ouvrant alors la porte à l’exploitation de données complexes comme les images et les corpus biomédicaux.

Par exemple, grâce aux architectures de \gls{cnn}, des modèles d’analyse d’images biomédicales ont été développés et mis à disposition de la communauté. À l’instar par exemple de Cellpose, un modèle développé en 2021 capable de segmenter automatiquement les cellules à partir de coupes histologiques ou StarDist (2018) capable de localiser les noyaux de cellules de manière automatique.

Plus récemment encore, grâce aux architectures à base de modules d’attention (\textit{Transformers}), des réseaux de neurones ont été entrainés pour comprendre du langage naturel (texte libre) pour en extraire de l’information. L’année 2023 représente une année clé dans l’histoire du \gls{nlp} avec le développement et la mise à disposition de \gls{llms} généraux performants et accessibles tels que GPT-3, ChatGPT ou LLAMA. Ces modèles représentent une véritable révolution dans le traitement des données non structurées et ouvrent la voie au développement d’outils innovant pour l’analyse des données biomédicales.

Ces méthodes permettent d’explorer de façon rétrospective et multimodale (textes et images) l’ensemble des données biomédicales acquises sur des patients qui n’étaient pas exploitables jusqu’à maintenant sans avoir besoin de réaliser un travail manuel d’annotation trop important.


\paragraph{\textbf{L’exemple des myopathies congénitales et la difficulté du diagnostic}}\mbox{}\\

Les \gls{mc} sont une famille de maladies rares et génétiques. Cette maladie peut être causée par une mutation sur un panel de 35 gènes différents et présente une prévalence d’environ 1,5 pour 100 000, soit environ 1000 patients en France. Actuellement, les \gls{mc} sont différenciées en cinq sous-types : \gls{nm}, \gls{com}, \gls{cnm}, \gls{cftd} et sans précision.

L’examen principal permettant la différenciation de ces sous-types de \gls{mc} est l’histopathologie du muscle. Cet examen donne lieu à la production d’un rapport d’analyse et permet de poser un diagnostic pour orienter le test génétique vers un groupe de gène candidat.

Cependant encore aujourd’hui, ce diagnostic est compliqué en raison de l’hétérogénéité des manifestations au niveau du muscle entre patients atteints d’un même sous-type de \gls{mc}. Mais aussi en raison d’un chevauchement important des manifestations phénotypiques entre des sous-types de \gls{mc} différents. La triple hétérogénéité des myopathies sur le plan clinique, histologique et génétique rendent difficile le diagnostic et l’orientation du test génétique. De nos jours, en raison de cette hétérogénéité et de la rareté de la maladie, 50 \% des patients atteints de \gls{mc} n’ont pas de diagnostic génétique à ce jour. 


\paragraph{\textbf{IMPatienT : un outil d’annotation et d’exploration de données multimodales de patients}}\mbox{}\\

Dans ce contexte, en collaboration avec l’Institut de Myologie de Paris et l’équipe du Dr Teresinha Evangelista, nous avons développé une plateforme en ligne nommée \gls{impatient} permettant de numériser et d’explorer les données de patients atteints de \gls{mc}. Plusieurs centaines de rapports de biopsies musculaires ont été générés ces dernières décennies à l’Institut de Myologie de Paris. Cette masse de documents contient un nombre important d’informations expertisées sur les critères de différenciation des sous-types de \gls{mc}. Cependant, ces documents sont sous la forme de texte libre semi-structuré. Nous avons donc utilisé un système de vocabulaire standardisé pour détecter et extraire les concepts clés dans ces rapports d’histologie et être capable d’en faire l’analyse statistique.

La plateforme \gls{impatient} a été conçue en quatre modules reliés à une base de données unique. 

Le premier module (M1) est le créateur de vocabulaire standard. Il permet aux utilisateurs de créer leur propre arborescence de termes ou de concepts qui peuvent ensuite être utilisés pour annoter les rapports de patients, en fonction de leurs présences ou absences. Chaque terme possède une définition riche : un identifiant unique, des synonymes (optionnel) et une traduction dans d’autres langues. De plus, au fur et à mesure que la base de données intègre des données patients, la définition des termes est enrichie avec les gènes et les diagnostics associés et les autres termes qui co-occurrent chez les patients.

Le deuxième module (M2) permet de numériser des rapports d’histologie. Il est possible de déposer un rapport d’histologie au format PDF dont le contenu va être analysé par comparaison (grep-search) au vocabulaire standard établi (c.f. M1) pour annoter automatiquement les concepts présents dans le rapport. Ces annotations peuvent être affinées à la main en sélectionnant manuellement l’absence et la granularité de la présence des termes du vocabulaire standard au sein du rapport. De plus le formulaire est connecté aux ontologies déjà existantes pour pouvoir ajouter des métadonnées liées aux rapports comme des symptômes cliniques (ontologie \gls{hpo}), un gène muté (nomenclature \gls{hgnc}), une variation (nomenclature \gls{hgvs}) et un diagnostic final (\gls{ordo}). Enfin, un système d’aide à la décision est disponible afin de suggérer un diagnostic sur la base de la similarité (méthodes bayésiennes) du profil du patient avec les patients déjà enregistrés dans la base de données d’\gls{impatient}.

Le troisième module (M3) permet l’annotation et la segmentation automatique d’images histologiques. Il est possible d’associer des régions de l’image à des termes issus du vocabulaire standard défini (c.f. M1). Ce qui donne ensuite lieu à une segmentation automatique de l’ensemble de l’image sur la base de l’intensité, du contraste et de la texture des régions annotées. Cette segmentation permet de préparer un jeu de données annoté pour le développement d’outils de segmentation automatique par \gls{ia} ou de réaliser de la quantification des régions segmentées.

Enfin, le dernier module (M4) correspond au tableau de bord de visualisation automatique. Il génère automatiquement des graphiques et tableaux permettant l’exploration statistique en temps réel des données enregistrées dans la base de données. À ce jour, 90 rapports de patients atteints de \gls{mc} sont enregistrés dans la base de données d’\gls{impatient} ce qui a permis la génération d’histogrammes de la répartition de ces patients par âge, gène muté et diagnostic. Une matrice de co-occurrence des termes du vocabulaire standard est aussi générée ainsi que des tableaux de fréquence d’apparition des termes pour chaque gène muté et chaque diagnostic. Enfin, des matrices de confusion permettent d’évaluer en temps réel les performances du système de suggestion de diagnostic.

Développé avant la révolution des \gls{llms}, \gls{impatient} est une plateforme permettant l’annotation semi-automatique et l’exploration de données multimodales de patients atteints de \gls{mc}. Les méthodes de détection des termes dans les rapports accélèrent le travail d’annotation, mais requièrent encore un travail manuel pour affiner les annotations réalisées. Il est alors intéressant d’explorer comment les \gls{llms} peuvent automatiser ce travail d’annotation pour faciliter l’exploration de ces données.



\paragraph{\textbf{Analyse de la base de données d’IMPatienT : classification des rapports par IA Explicable (xAI)}}\mbox{}\\

Le concept d’\gls{xai} se réfère à la capacité de comprendre et d’expliquer le fonctionnement des systèmes d’\gls{ia} de manière claire et compréhensible pour les êtres humains. C’est une caractéristique importante des modèles \gls{ia} notamment dans le domaine du diagnostic, car il est préférable pour l’Homme d’être en mesure de comprendre sur quels critères une prédiction est réalisée. Les 90 rapports annotés via \gls{impatient} ont été utilisés pour entrainer un total de 11 algorithmes de \gls{ml} afin de comparer leurs performances sur des données réelles. Pour cela, nous avons modifié et utilisé le pipeline Streamline qui entraine, optimise et compare un vaste panel d’algorithme dont les \gls{lcs}, considérés comme un système de référence en termes d’explicabilité. Nos résultats préliminaires ont montré que sur 90 rapports, il est possible de construire des modèles d’aide au diagnostic capable de faire la différence entre 3 sous-types de \gls{mc} avec une exactitude de 83\%. Nous avons aussi exploré de nouvelles façons de visualiser les connaissances contenues dans ces \gls{lcs} pour faciliter l’extraction de connaissances de ces modèles.

\paragraph{\textbf{NLMyo : Traitement de rapport textuel par modèles linguistiques de grande taille}}\mbox{}\\

La plateforme \gls{impatient} se base sur une approche semi-automatique, nécessitant une phase d’annotation manuelle des rapport pour leur exploitation. Grâce aux développements récents de LLMs, nous avons pu développer \gls{nlmyo}, un ensemble de quatre outils supplémentaires pour permettre d’exploiter de manière totalement automatique et rapide un grand nombre de rapports textuels.

Le premier outil (Anonymizer) permet d’anonymisation des données. Comme nous travaillons sur des données de santé, il est important de s’assurer que nous ne traitons que des données anonymisées. Ce travail d’anonymisation manuel est long et fastidieux. Ainsi, grâce aux \gls{llms}, nous pouvons détecter automatiquement les nom, prénom ainsi que dates de naissance dans les documents et nous pouvons les censurer avant le traitement de ces données.

Le second outil (MyoExtract) permet l’extraction des métadonnées d’un rapport histopathologique. Pour faciliter la numérisation des rapports, il est utile d’extraire de manière automatique des informations communes à chaque rapport. De fait, à partir du texte du rapport, nous sommes capables d’utiliser les \gls{llms} pour extraire automatiquement le numéro de biopsie, l’âge du patient, le muscle prélevé et le diagnostic final. Grâce à une instruction spécifique donnée au \gls{llms}, nous extrayons ces informations dans un format informatique standard (JSON) qui permet son traitement de manière automatique pour pré-remplir les champs des formulaires informatiques utilisés pour numériser les données de patients, par exemple dans \gls{impatient}.

En troisième outil (MyoClassify), nous avons exploré la possibilité de prédire un diagnostic de manière totalement automatique sans aucune annotation humaine à partir du texte brut du rapport et avons obtenu une exactitude de classification de 65 \% (versus 35 \% pour le hasard). Cette classification est réalisée grâce à des techniques de vectorisation de phrases (\textit{embedding}). L’\textit{embedding} correspond à la transformation par \gls{ia} d’un texte en un unique vecteur numérique de grande taille (plusieurs centaines voire milliers de dimensions) capable de capturer le sens sémantique du texte. En appliquant cette méthode sur un corpus de rapport élargis (149 rapports) avec des diagnostics différents, nous avons pu entrainer un modèle d’\gls{ia} capable de prédire le diagnostic associé à un rapport uniquement à partir de son \textit{embedding}.

Enfin, en quatrième outil (MyoSearch), nous avons utilisé ces techniques d’\textit{embedding} pour développer un véritable moteur de recherche intelligent de patient. Dans cet outil, l’utilisateur peut formuler une requête en texte libre pour rechercher par exemple des patients ayant un symptôme spécifique. L’\textit{embedding} de cette requête sera comparé à l’\textit{embedding} de l’ensemble des phrases contenues dans les rapports d’histologiques, et par calcul de similarité, les rapports avec les meilleurs scores seront présentés en premier comme correspondant à la requête. Cet outil permet de référencer et de rapidement trouver des patients ayant un profil symptomatique spécifique.

L’intégration future de ces méthodes automatiques dans \gls{impatient} pourrait permettre de faciliter la numérisation des patients et gagner un temps important lors de l’annotation de ces patients dans la base de données.


\paragraph{\textbf{Vers une génération de rapports automatique à partir d’imagerie avec MyoQuant}}\mbox{}\\

Grâce aux outils présentés précédemment, nous sommes en mesure d’exploiter les informations contenues dans les rapports histologiques de patients. Cependant, ces rapports sont rédigés à la main après observation de coupe de biopsie musculaire au microscope par un biologiste ou médecin. L’expertise manuelle des images de biopsie musculaire est couteuse en temps et elle n’est que qualitative : par exemple pour la centralisation nucléaire, un marqueur pathologique typique des \gls{mc}, il sera noté qu’il y en a peu ou beaucoup, mais sans valeur numérique, car le comptage des fibres individuelles serait trop couteux en temps. Il est donc intéressant d’avoir des outils capables de réaliser ce travail de comptage de façon automatique pour améliorer la précision de l’évaluation des biopsies musculaires et réduire le travail manuel nécessaire.

Nous avons développé \gls{myoquant}, en collaboration avec l’équipe du Dr. Jocelyn Laporte de l’IGBMC à Strasbourg. Cet outil permet de réaliser la quantification automatique de marqueurs pathologiques détectés en routine dans les biopsies musculaires lors du diagnostic des \gls{mc}. Actuellement \gls{myoquant} est capable de quantifier automatiquement des marqueurs pathologiques dans trois des cinq techniques de coloration réalisée en routine lors de la biopsie musculaire (\gls{he}, ATPase, \gls{sdh}) grâce à des systèmes d’\gls{ia}.

Pour la coloration \gls{he} qui met en évidence les noyaux cellulaires, souvent centralisés dans les \gls{cnm}, nous avons développé un algorithme capable d’évaluer le niveau de centralisation de chaque noyau dans les fibres musculaires. Dans une fibre musculaire saine, les noyaux sont localisés en périphérie des fibres. Grâce aux outils existants CellPose et Stardist, qui permettent respectivement de segmenter les fibres et les noyaux cellulaires d’une coupe histologique, nous pouvons calculer pour chaque noyau un score d’excentricité, représentatif de son niveau de centralisation. En appliquant cette procédure à l’ensemble de la coupe, nous pouvons compter automatiquement le nombre de noyaux internalisés ou centralisés. Dans le même temps, l’outil est aussi en mesure d’évaluer la taille des fibres musculaires et de compter le nombre de fibres atrophiques.

Pour la coloration ATPase, qui colore de façon différentielle les fibres de type 1 et les fibres de type 2 et dont l'équilibre est souvent modifié dans les \gls{mc}, nous avons développé une méthode de \gls{ml} capable de définir automatiquement un ou plusieurs seuils d’intensité qui permet de différencier et compter les fibres de chaque catégorie.

Enfin, pour la coloration au \gls{sdh} qui met en évidence l’activité oxydative des fibres, souvent anormale dans les \gls{com}, nous avons développé un réseau de neurones de type Resnet50 capable de détecter les fibres ayant une répartition mitochondriale anormale. Entrainé sur un total de 17 000 fibres musculaires issues de 17 souris modèles de \gls{mc}, notre réseau de neurones obtient une exactitude de classification de 93 \%.

L’ensemble de ces techniques permet de quantifier rapidement et automatiquement plusieurs marqueurs pathologiques des \gls{mc}. Nous souhaitons à présent étendre le champ de détection de \gls{myoquant} en développant des méthodes pour détecter les agrégats protéiques des colorations au \gls{tg} ainsi que les \textit{cores} dans la coloration NADH. Ces méthodes pourraient permettre à terme d’être capable de générer automatiquement un rapport de biopsie musculaire plus précis. 


\paragraph{\textbf{Conclusions et Perspectives}}\mbox{}\\

Dans le cadre de cette thèse, j’ai eu l’occasion de développer plusieurs outils permettant d’exploiter des données multimodales de patients par approches \gls{ia}. Avec \gls{impatient}, j’ai créé une plateforme d’annotation et d’exploration de rapports histologiques de patients qui a permis de numériser une centaine de rapports de patients. La base de données d’\gls{impatient} a été utilisée pour évaluer le performance de plusieurs approches \gls{xai} pour la prédiction des \gls{mc}. Ensuite, avec \gls{nlmyo}, j’ai exploré comment les récentes avancées en traitement de langage naturel grâce aux \gls{llms} pouvaient faciliter et accélérer l’annotation et la classification de ces rapports textuels. Enfin, avec \gls{myoquant}, j’ai développé un outil qui permet d’accélérer le processus d’évaluation des images de biopsies musculaires avec comme but à terme, d’être capable de générer un rapport histologique de façon automatique.

Les avancées en \gls{ia} pavent le chemin pour une amélioration du diagnostic des maladies rares. Les outils que j’ai développés dans le cadre de cette thèse sont un exemple de science translationnelle. Sur le plan diagnostic, ces outils peuvent permettre un gain de temps pour les praticiens via l’automatisation de l’extraction d’information lors de l’évaluation des résultats d’analyse. Sur le plan de la recherche, ils peuvent permettre la découverte de nouveaux critères de diagnostic potentiellement important. Il serait maintenant intéressant de compiler l’ensemble de ces outils dans une plateforme unique, cohérente et clé en main utilisable par la communauté pour démocratiser l’usage de ces outils. De plus l’intégration de méthodes d’analyse de données génomique complèterait la palette d’outil mis à disposition.


\paragraph{\textbf{Publications et communications}}\mbox{}\\

\underline{Papier soumis :} Meyer, C., Romero, N., Evangelista, T., Cadot, B., Laporte, J., Jeannin-Girardon, A., Collet, P., Chennen, K., \& Poch, O. (2022). IMPatienT: An Integrated web application to digitize, process and explore Multimodal PATIENt daTa. (p. 2022.04.08.487635). bioRxiv. https://doi.org/10.1101/2022.04.08.487635

\underline{Poster:} IMPatienT: an integrated web application to digitize, process and explore multimodal patient data – ED Days 2022, Strasbourg, France

\underline{Poster et communication orale (demo):} IMPatienT: an integrated web application to digitize, process and explore multimodal patient data – JOBIM, Strasbourg, France

\underline{Communication orale:}

MyoQuant: a tool to automatically quantify pathological features in muscle fiber histology images – JSFM 2022, Toulouse, France

MyoQuant: a tool to automatically quantify pathological features in muscle fiber histology images, Journée Fondation Maladie Rare 2022, Strasbourg, France

MyoQuant: a tool to automatically quantify pathological features in muscle fiber histology images, ED Days 2023, Strasbourg France
