\chapter{Résumé de Thèse}
\paragraph{\textbf{L’avènement des \textit{Big Data} dans le domaine biomédical}}\mbox{}\\

Le terme "\textit{Big Data}" fait référence à des ensembles de données extrêmement vastes, complexes et hétérogènes qui dépassent la capacité des outils de traitement de données traditionnels. Les \textit{Big Data} sont caractérisées par les "5 V" : le volume, la variété, la vélocité, la véracité et la valeur. Les données biomédicales sont un exemple concret de \textit{Big Data}. Elles sont multimodales : regroupant l’ensemble des données relatives à la santé humaine telles que les données génétiques, les dossiers cliniques, les images médicales ou les rapports d’analyses. Elles sont massives (ex. : taille des données de séquençage, imagerie à l’échelle du giga pixel, milliers de comptes rendus en texte libre…) et ont un flux important grâce à l’amélioration des techniques d’acquisition d’images, d’analyse et de séquençage. Ces données sont utilisées pour améliorer la compréhension des maladies et pour poser des diagnostics pour les patients. L’augmentation exponentielle des volumes, modalités et complexités des données biomédicales rend impossible leur traitement manuel et requiert donc l’utilisation d’outils adaptés pour traiter et extraire des informations pertinentes dans le cadre de la recherche médicale, comme les méthodes basées sur l’\gls{ia}.

\paragraph{\textbf{Nouvelle génération d’\gls{ia} pour le traitement du \textit{Big Data} grâce aux réseaux de neurones}}\mbox{}\\

L’\gls{ia} représente un ensemble de techniques permettant de créer des programmes simulant l’intelligence humaine. Une sous branche de l’\gls{ia} nommée \gls{ml} regroupe des algorithmes permettant à un programme informatique d’accomplir une tâche en apprenant d’un jeu de données. Cependant, les techniques de \gls{ml} traditionnelles ne peuvent apprendre que de données sous forme de tableaux, fermant la porte à l’exploitation de données plus complexes, comme les images ou les textes libres.

Cette dernière décennie, la popularisation des \gls{dnn}, reposant sur le concept bio-inspiré de neurones, a permis l’exploitation de données complexes sans avoir besoin de connaissance à priori sur les données, c’est-à-dire sans devoir définir des descripteurs pertinents manuellement. Par exemple, grâce aux architectures de \gls{cnn}, des modèles d’analyse d’images biomédicales ont été développés et mis à disposition de la communauté. Plus récemment encore, grâce aux architectures à base de modules d’attention (\textit{Transformers}), des réseaux de neurones ont été entrainés pour comprendre du langage naturel (texte libre) et en extraire de l’information. L’année 2023 représente une année clé dans l’histoire du \gls{nlp} avec le développement et la mise à disposition de \gls{llms} généraux performants et accessibles tel que \textit{GPT-3.5-turbo} (souvent nommé à tort \textit{ChatGPT}) ou \textit{LLAMA}. Ces méthodes permettent d’explorer de façon rétrospective et multimodale, l’ensemble des données biomédicales acquises sur des patients sans avoir besoin de réaliser un travail manuel d’annotation trop important.

\paragraph{\textbf{L’exemple des myopathies congénitales et la difficulté du diagnostic}}\mbox{}\\

Les \gls{mc} sont une famille de maladies rares et génétiques. Cette maladie peut être causée par une mutation sur un panel de 35 gènes différents et présente une prévalence d’environ 1,5 pour 100 000, soit environ 1000 patients en France. Actuellement, les \gls{mc} sont différenciées en cinq sous-types : \gls{nm}, \gls{com}, \gls{cnm}, \gls{cftd} et sans précision.

L’examen principal permettant la différenciation de ces sous-types de \gls{mc} est l’histopathologie du muscle. Cet examen donne lieu à la rédaction d’un rapport d’analyse et permet de poser un diagnostic pour orienter le test génétique vers un groupe de gènes candidats. Cependant encore aujourd’hui, ce diagnostic est compliqué en raison de l’hétérogénéité des manifestations au niveau du muscle entre patients atteints d’un même sous-type de \gls{mc}. Mais aussi en raison d’un chevauchement important des manifestations phénotypiques entre des sous-types de \gls{mc} différents. La triple hétérogénéité des myopathies sur le plan clinique, histologique et génétique rendent difficile le diagnostic et l’orientation du test génétique. En raison de cette hétérogénéité et de la rareté de la maladie, 50 \% des patients atteints de \gls{mc} n’ont pas de diagnostic génétique à ce jour. 

\paragraph{\textbf{IMPatienT : un outil d’annotation et d’exploration de données multimodales de patients}}\mbox{}\\

Pendant ma thèse, en collaboration avec l’Institut de Myologie de Paris, j’ai développé une plateforme en ligne nommée \gls{impatient} permettant de numériser et d’explorer les données de patients atteints de \gls{mc}. Plusieurs centaines de rapports de biopsies musculaires ont été générés ces dernières décennies et cette masse de documents contient des informations expertisées sur les critères de différenciation des sous-types de \gls{mc}. Cependant, ces documents sont sous la forme de texte libre semi-structuré. Nous avons donc utilisé une approche ontologique pour détecter et extraire les concepts clés dans ces rapports d’histologie et être capables d’en faire l’analyse statistique. La plateforme \gls{impatient} a été conçue en quatre modules reliés à une base de données unique. 

Le premier module permet aux utilisateurs de créer leur propre vocabulaire standard (arborescence de termes ou de concepts) qui sera ensuite utilisé par le module 2. Au fur et à mesure que la base de données intègre des données patients, la définition des termes est enrichie avec les gènes et les diagnostics associés et les autres termes qui co-occurrent chez les patients. Le deuxième module permet de numériser des rapports d’histologie textuels et d’annoter automatiquement les concepts présents dans le rapport. Ces annotations peuvent être affinées par l’expert, ou en ajoutant des métadonnées liées aux rapports comme des symptômes cliniques, un gène muté, une variation et un diagnostic final. Enfin, un système d’aide à l’a décision est disponible afin suggérer un diagnostic sur la base de la similarité (méthodes bayésiennes) du profil du patient avec les patients déjà enregistrés dans la base de données d’\gls{impatient}. Le troisième module permet la segmentation automatique d’images histologiques, et leur annotation avec des termes issus du vocabulaire standard. Enfin, le dernier module correspond au tableau de bord de visualisation automatique. Il génère automatiquement des graphiques et tableaux permettant l’exploration statistique en temps réel des données enregistrées dans la base de données. 

Développée avant la révolution des \gls{llms}, \gls{impatient} est une plateforme permettant l’annotation semi-automatique et l’exploration de données multimodales de patients atteints de \gls{mc}. Les méthodes de détection des termes dans les rapports accélèrent le travail d'annotation, mais requièrent encore un travail manuel pour affiner les annotations réalisées. Ainsi, par la suite, j'ai exploré comment les \gls{llms} peuvent automatiser ce travail d'annotation pour faciliter l'exploration des données.


\paragraph{\textbf{Analyse de la base de données d’IMPatienT : classification des rapports par IA Explicable (xAI)}}\mbox{}\\

Le concept d’\gls{xai} se réfère à la capacité de comprendre et d’expliquer le fonctionnement des systèmes d’\gls{ia} de manière claire et compréhensible pour les êtres humains. C’est une caractéristique importante des modèles \gls{ia} notamment dans le domaine du diagnostic, car il est préférable pour l’Homme d’être en mesure de comprendre sur quels critères une prédiction est réalisée. Les 89 rapports annotés via \gls{impatient} ont été utilisés pour entrainer différents algorithmes d'\gls{ia} dont les \gls{lcs}, considérés comme un système de référence en termes d’explicabilité. Nous avons comparé leurs performances sur des données réelles et avons obtenu une exactitude de classification de 83\% pour la différenciation entre \gls{nm}, \gls{com} et \gls{cnm}. Nous avons aussi exploré de nouvelles façons de visualiser les connaissances contenues dans ces \gls{lcs} pour faciliter l’extraction de connaissances.

\paragraph{\textbf{NLMyo : Traitement de rapport textuel par modèles linguistiques de grande taille}}\mbox{}\\

Grâce aux développements récents de \gls{llms}, nous avons pu développer \gls{nlmyo}, intégrant quatre outils pour permettre d’exploiter de manière totalement automatique et rapide un grand nombre de rapports textuels.

Le premier outil (\textit{Anonymiser}) permet l’anonymisation des données, une étape essentielle pour travailler sur des données de santé. Grâce aux \gls{llms}, nous pouvons détecter automatiquement les noms, prénoms et dates de naissance dans les documents et les censurer avant le traitement de ces données. Le second outil (\textit{MyoExtract}) permet l’extraction automatique des métadonnées d’un rapport. Nous utilisons les \gls{llms} pour extraire automatiquement le numéro de biopsie, l’âge du patient, le muscle prélevé et le diagnostic final. Nous extrayons ces informations dans un format standard (JSON) qui permet son traitement de manière automatique pour pré-remplir les champs des formulaires utilisés pour numériser les données de patients, par exemple dans \gls{impatient}. Avec le troisième outil (\textit{MyoClassify}), nous avons exploré la possibilité de prédire un diagnostic de manière totalement automatique sans aucune annotation humaine à partir du texte brut du rapport et avons obtenu une justesse de classification de 65 \% (versus 35 \% pour le hasard). Cette classification est réalisée grâce à des techniques de vectorisation de phrases (\textit{embedding}). L’\textit{embedding} correspond à la transformation d’un texte en un unique vecteur numérique de grande taille (plusieurs centaines voire milliers de dimensions) capable de capturer le sens sémantique du texte. En appliquant cette méthode sur un corpus élargi de 192 rapports, nous avons pu entrainer un modèle d’\gls{ia} capable de prédire le diagnostic associé à un rapport uniquement à partir de son \textit{embedding}. Enfin, le quatrième outil (\textit{MyoSearch}) exploite ces techniques d’\textit{embedding} pour fournir un véritable moteur de recherche intelligent de patient. L’utilisateur peut formuler une requête en texte libre pour rechercher par exemple des patients ayant un symptôme spécifique. L’\textit{embedding} de cette requête sera comparé à l’\textit{embedding} de l’ensemble des phrases contenues dans les rapports histologiques, et les rapports avec les meilleurs scores seront présentés en premier. Cet outil permet de référencer et de rapidement trouver des patients ayant un diagnostic ou un profil symptomatique spécifique.

L’intégration future de ces méthodes dans \gls{impatient} facilitera la numérisation des données patients et permettra de gagner un temps important lors de l’annotation de ces patients dans la base de données.

\paragraph{\textbf{Vers une génération de rapports automatique à partir d’imagerie avec MyoQuant}}\mbox{}\\

Grâce aux outils présentés précédemment, nous sommes en mesure d’exploiter les informations contenues dans les rapports histologiques de patients. Cependant, ces rapports sont rédigés à la main après observation de coupes de biopsie musculaire au microscope par un biologiste ou médecin. L’expertise manuelle des images de biopsie musculaire est couteuse en temps et elle n’est que qualitative : par exemple pour la centralisation nucléaire, un marqueur pathologique typique des \gls{mc}, il sera noté qu’il y en a peu ou beaucoup, mais sans valeur numérique, car le comptage des fibres individuelles serait trop couteux en temps. 

Nous avons développé \gls{myoquant}, en collaboration avec l’équipe du Dr Jocelyn Laporte de l’IGBMC à Strasbourg, afin d’automatiser ce travail de comptage. Actuellement, \gls{myoquant} peut quantifier des marqueurs pathologiques dans trois des cinq techniques de coloration réalisées en routine lors de la biopsie musculaire grâce à des systèmes d’\gls{ia}. Pour la coloration \gls{he} qui met en évidence les noyaux cellulaires, l’algorithme est capable d’évaluer le niveau de centralisation de chaque noyau dans les fibres musculaires. Dans une fibre musculaire saine, les noyaux sont localisés en périphérie des fibres. En segmentant les fibres et les noyaux cellulaires d’une coupe histologique, nous calculons pour chaque noyau un score d’excentricité, représentatif de son niveau de centralisation, pour ensuite compter automatiquement le nombre de noyaux internalisés ou centralisés. Pour la coloration ATPase, qui colore de façon différentielle les fibres de type 1 et de type 2 et dont l'équilibre est modifié dans les \gls{mc}, nous avons développé une méthode de \gls{ml} capable de définir automatiquement un ou plusieurs seuils d’intensité qui permet de différencier et compter les fibres de chaque catégorie. Enfin, pour la coloration au \gls{sdh} qui met en évidence l’activité oxydative des fibres, souvent anormale dans les \gls{com}, nous avons développé un réseau de neurones capable de détecter les fibres ayant une répartition mitochondriale anormale. Entrainé sur un total de 17 000 fibres musculaires issues de 17 souris modèles de \gls{mc}, notre réseau de neurones obtient une exactitude de classification de 93 \%.

Nous souhaitons à présent étendre le champ de détection de \gls{myoquant} en développant des méthodes pour détecter les agrégats protéiques des colorations au \gls{tg} ainsi que les \textit{cores} dans la coloration NADH, pour permettre à terme de générer automatiquement un rapport de biopsie musculaire plus précis. 

\paragraph{\textbf{Conclusions et Perspectives}}\mbox{}\\

Dans le cadre de cette thèse, j’ai eu l’occasion de développer plusieurs outils permettant d’exploiter des données multimodales de patients par approches \gls{ia}. Avec \gls{impatient}, j’ai créé une plateforme d’annotation et d’exploration de rapports histologiques de patients qui a permis de numériser une centaine de rapports de patients. La base de données d'\gls{impatient} a été utilisée pour évaluer les performances de plusieurs approches \gls{xai} pour la prédiction des \gls{mc}. Ensuite, avec \gls{nlmyo}, j’ai exploré comment les récentes avancées en \gls{nlp} grâce aux \gls{llms} pouvaient faciliter et accélérer l’annotation et la classification de ces rapports textuels. Enfin, avec \gls{myoquant}, j’ai développé un outil qui permet d’accélérer le processus d’évaluation des images de biopsies musculaires avec comme but, à terme, d’être capable de générer un rapport histologique de façon automatique.

Les avancées en \gls{ia} ouvrent la voie pour une amélioration du diagnostic des maladies rares. Les outils que j’ai développés sont un exemple de science translationnelle. Sur le plan diagnostic, ces outils peuvent permettre un gain de temps pour les praticiens via l’automatisation de l’extraction d’information lors de l’évaluation des résultats d’analyse. Sur le plan de la recherche, ils peuvent permettre la découverte de nouveaux critères de diagnostic potentiellement importants. Il serait maintenant intéressant de compiler l’ensemble de ces outils dans une plateforme unique, cohérente et clé en main utilisable par la communauté pour démocratiser l’usage de ces outils. De plus, l’intégration de méthodes d’analyse de données génomique complèterait la palette d’outils mis à disposition.

\paragraph{\textbf{Publications et communications}}\mbox{}\\

\underline{Papier soumis dans Journal of Neuromuscular Diseases:} Meyer, C., Romero, N., Evangelista, T., Cadot, B., Laporte, J., Jeannin-Girardon, A., Collet, P., Chennen, K., \& Poch, O. (2022). IMPatienT: An Integrated web application to digitize, process and explore Multimodal PATIENt daTa. (p. 2022.04.08.487635). bioRxiv. https://doi.org/10.1101/2022.04.08.487635. 

\underline{Poster:} 

IMPatienT: an integrated web application to digitize, process and explore multimodal patient data – Meyer, C., Romero, N., Evangelista, T., Cadot, B., Laporte, J., Jeannin-Girardon, A., Collet, P., Chennen, K., \& Poch, O. - ED Days 2022 - Strasbourg France

Methods for the exploitation of multimodal patient data using artificial intelligence: application to congenital myopathies - Meyer, C., Giraud Q., Vernay B., Romero, N., Evangelista, T., Cadot, B., Laporte, J., Jeannin-Girardon, A., Collet, P., Chennen, K., \& Poch, O. – ISMB/ECCB 2023 - Lyon France

\underline{Communication orale (demo) et Poster:} IMPatienT: an integrated web application to digitize, process and explore multimodal patient data - Meyer, C., Romero, N., Evangelista, T., Cadot, B., Laporte, J., Jeannin-Girardon, A., Collet, P., Chennen, K., \& Poch, O. – JOBIM 2022 - Rennes France

\underline{Communication orale:}

MyoQuant: a tool to automatically quantify pathological features in muscle fiber histology images – Meyer, C., Giraud Q., Vernay B., Romero, N., Evangelista, T., Cadot, B., Laporte, J., Jeannin-Girardon, A., Collet, P., Chennen, K., \& Poch, O. - JSFM 2022 - Toulouse France

MyoQuant: a tool to automatically quantify pathological features in muscle fiber histology images - Meyer, C., Giraud Q., Vernay B., Romero, N., Evangelista, T., Cadot, B., Laporte, J., Jeannin-Girardon, A., Collet, P., Chennen, K., \& Poch, O. - Journée Fondation Maladie Rare 2022 - Strasbourg France

MyoQuant: a tool to automatically quantify pathological features in muscle fiber histology images - Meyer, C., Giraud Q., Vernay B., Romero, N., Evangelista, T., Cadot, B., Laporte, J., Jeannin-Girardon, A., Collet, P., Chennen, K., \& Poch, O. - ED Days 2023 - Strasbourg France
