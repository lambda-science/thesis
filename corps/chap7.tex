\chapter{Vers une génération de rapports automatique à partir d’imagerie avec MyoQuant}

\section{Contexte}
\section{Méthodes d'analyses de la position de noyaux cellulaires}
\subsection{Algorithme}
\subsection{Exemple d'application}
\section{Méthodes de classification basé sur l'intensité de coloration}
\subsection{Algorithme}
\subsection{Exemple d'application}
\section{Méthodes de classification basé sur intelligence artificielle}
\subsection{IA}
\subsection{Exemple d'application}
\section{Déploiement de la plateforme}
\subsection{Outil en ligne de commande}
\subsection{Démo en ligne}
\section{Limitations}
\section{Perspectives futures de développement}