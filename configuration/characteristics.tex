% Auteur de la thèse : prénom (1er argument obligatoire), nom (2e argument
% obligatoire) et éventuel courriel (argument optionnel). Les éventuels accents
% devront figurer et le nom /ne/ doit /pas/ être saisi en capitales
\author[]{Corentin}{Meyer}
%
% Titre de la thèse dans la langue principale (argument obligatoire) et dans la
% langue secondaire (argument optionnel)
\title[]{Méthodes d’exploitation de données multimodales de patients par intelligence artificielle : cas d’application dans les  myopathies congénitales.}
%
% (Facultatif) Sous-titre de la thèse dans la langue principale (argument
% obligatoire) et dans la langue secondaire (argument optionnel)
% \subtitle[]{}
%
% Champ disciplinaire dans la langue principale (argument obligatoire) et dans
% la langue secondaire (argument optionnel)
\academicfield[]{Sciences de la vie et de la santé}
%
% (Facultatif) Spécialité dans la langue principale (argument obligatoire) et
% dans la langue secondaire (argument optionnel)
\speciality[]{Bioinformatique et biologie des systèmes}
%
% Date de la soutenance, au format {jour}{mois}{année} donnés sous forme de
% nombres
\date{01}{09}{2023}
%
% (Facultatif) Date de la soumission, au format {jour}{mois}{année} donnés sous
% forme de nombres
%\submissiondate{}{}{}
%
% (Facultatif) Sujet pour les méta-données du PDF
\subject[]{}
%
% (Facultatif) Nom (argument obligatoire) de la ComUE
\pres[logo=logo/unistra.png,url=https://www.unistra.fr/]{}
%
% Nom (argument obligatoire) de l'institut (principal en cas de cotutelle)
\institute[logo=logo/ed414.png,url=https://www.unistra.fr/]{Université de Strasbourg}
%
% (Facultatif) En cas de cotutelle (normalement, seulement dans le cas de
% cotutelle internationale), nom (argument obligatoire) du second institut
% \coinstitute[logo=]{}
%
% (Facultatif) Nom (argument obligatoire) de l'école doctorale
\doctoralschool[url=http://ed.vie-sante.unistra.fr/]{École doctorale des sciences de la vie et de la santé}
%
% Nom (1er argument obligatoire) et adresse (2e argument obligatoire) du
% laboratoire (ou de l'unité) où la thèse a été préparée, à utiliser /autant de
% fois que nécessaire/
\laboratory[
logo=logo/icube.png,
telephone=0368854554,
email=contact@icube.unistra.fr,
url=https://icube.unistra.fr/
]{ICube - UMR 7357}{%
  \\
  \\
  \\
  \\
  \\
  }
%
% Directeur(s) de thèse et membres du jury, saisis au moyen des commandes
% \supervisor, \cosupervisor, \comonitor, \referee, \committeepresident,
% \examiner, \guest, à utiliser /autant de fois que nécessaire/ et /seulement
% si nécessaire/. Toutes basées sur le même modèle, ces commandes ont
% 2 arguments obligatoires, successivement les prénom et nom de chaque
% personne. Si besoin est, on peut apporter certaines précisions en argument
% optionnel, essentiellement au moyen des clés suivantes :
% - « professor », « seniorresearcher », « associateprofessor »,
%   « associateprofessor* », « juniorresearcher », « juniorresearcher* » (qui
%   peuvent ne pas prendre de valeur) pour stipuler le corps auquel appartient
%   la personne ;
% - « affiliation » pour stipuler l'institut auquel est affiliée la personne ;
% - « female » pour stipuler que la personne est une femme pour que certains
%   mots clés soient accordés en genre.
%
\supervisor[,affiliation=Directeur de recherche CNRS]{Olivier}{Poch}
\supervisor[,affiliation=Professeur Université de Strasbourg]{Pierre}{Collet}
% \cosupervisor[,affiliation=]{}{}
% \comonitor[,affiliation=]{}{}
\referee[,affiliation=]{John}{Doe}
\referee[,affiliation=]{John}{Doe}
\committeepresident[,affiliation=]{John}{Doe}
\examiner[,affiliation=]{John}{Doe}
\examiner[,affiliation=]{John}{Doe}
% \guest{}{}
%
% (Facultatif) Mention du numéro d'ordre de la thèse (s'il est connu, ce numéro
% est à spécifier en argument optionnel)
% \ordernumber[]
%
% Préparation des mots clés dans la langue principale (1er argument) et dans la
% langue secondaire (2e argument)
%%%%%%%%%%%%%%%%%%%%%%%%%%%%%%%%%%%%%%%%%%%%%%%%%%%%%%%%%%%%%%%%%%%%%%%%%%%%%%%
\keywords{mot clé1, mot clé2}{keyword 1, keyword 2}

